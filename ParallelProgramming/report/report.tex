%! TEX program = xelatex
\documentclass{report}
% provides basic settings for ctex document
\usepackage[UTF8, heading=true]{ctex}
\usepackage{fancyhdr}
\usepackage{tocloft}
\usepackage[margin=1in]{geometry}
\usepackage{metalogo}                   % \XeLaTeX
\usepackage{float}                      % figure H flag
\usepackage{microtype}                  % break long words
\usepackage[hidelinks]{hyperref}
\usepackage{tabularx}
\usepackage{amsmath}
\usepackage{lmodern}                    % allow fonts to scale
\usepackage{placeins}
\usepackage{multirow}                   % multirow, multicolumn support
\usepackage{booktabs}                   % toprule, cmidrule support
\usepackage{caption}

% make chapter stay in the same page
%\makeatletter
%\renewcommand\chapter{\thispagestyle{plain}%
%\global\@topnum\z@
%\@afterindentfalse
%\secdef\@chapter\@schapter}
%\makeatother

\fancyhead{}
\renewcommand{\sectionmark}[1]{\markleft{#1}}
\renewcommand{\partmark}[1]{\markright{#1}}
\lhead{\tiny \leftmark}
\rhead{\tiny \rightmark}
% see $(texdoc ctex) for details
\ctexset{
    chapter = {
        name = {实验},
        format += \flushleft,
        number = \arabic{chapter},
    },
    section = {
        format += \flushleft,
    },
    appendix = {
        number = \Alph{chapter},
        name = {附录},
    },
}

\pagestyle{fancy}
%\setlength\cftaftertoctitleskip{2em}

% provides code input support
\usepackage{xparse}                     % newcommand multiple optional arguments
\usepackage{listings}                   % code
\usepackage{fontspec}
\usepackage{lmodern}

%\newfontfamily\codeF{Fira Code}

\setmonofont[
    Contextuals={Alternate},
    ItalicFont = Fira Code      % to avoid font warning
]{Fira Code}

% usage: \inputCode{[language] <path>}
% if language is not explicitly set, it's defaulted to c
\DeclareDocumentCommand{\inputCode}{ O{c} m }{
    {
        \lstinputlisting[
            basicstyle=\small\ttfamily,
            language={#1},
            tabsize=4,
            showstringspaces=false,
            breaklines=true,
            frame=shadowbox,
            framexleftmargin=10mm,
            rulesepcolor=\color{black},
            numbers=left,
            xleftmargin=4em,
        ]{#2}
    }
}

\DeclareDocumentCommand{\inputCodeSetLanguage}{ m }{
    \lstset{
        basicstyle=\small\ttfamily,
        language={#1},
        tabsize=4,
        showstringspaces=false,
        breaklines=true,
        frame=shadowbox,
        framexleftmargin=10mm,
        rulesepcolor=\color{black},
        numbers=left,
        xleftmargin=4em,
    }
}

\DeclareDocumentCommand{\inputCodeNoNumberSetLanguage}{ m }{
    \lstset{
        basicstyle=\small\ttfamily,
        language={#1},
        tabsize=4,
        showstringspaces=false,
        breaklines=true,
        frame=shadowbox,
        rulesepcolor=\color{black},
    }
}


\usepackage{xparse}
\usepackage{tcolorbox}
\usepackage{mdframed}

\tcbuselibrary{breakable}
\NewDocumentCommand{\exercise}{ m +m }{
    {
        \edef\originalParIndent{\the\parindent}
        \begin{tcolorbox}[breakable,arc=0mm,boxrule=0.4pt]
            \setlength{\parindent}{\originalParIndent}
            \noindent
            \textbf{\large Exercise #1}
            \indent
            #2
        \end{tcolorbox}
    }
}

% environment style of exercise, to feed special requirements
\NewDocumentEnvironment{exerciseEnv}{m}{
    \edef\originalParIndent{\the\parindent}
    \begin{tcolorbox}[breakable,arc=0mm,boxrule=0.4pt]
        \setlength{\parindent}{\originalParIndent}
        \noindent
        \textbf{\large Exercise #1}
        \indent
} {
    \end{tcolorbox}
}

\NewDocumentEnvironment{questionEnv}{}{
    \edef\originalParIndent{\the\parindent}
    \begin{tcolorbox}[breakable,arc=0mm,boxrule=0.4pt]
        \setlength{\parindent}{\originalParIndent}
        \noindent
        \textbf{\large Question}
        \indent
} {
    \end{tcolorbox}
}

% a raised rule
\NewDocumentCommand{\raisedrule}{ O{0em} m }{\leaders\hbox{\rule[#1]{1pt}{#2}}\hfill}

\NewDocumentEnvironment{exerciseSolution}{m}{
    {\noindent \textbf{\large Exercise #1 实验过程 \raisedrule[0.3em]{0.6pt}}}
} {
    \par
    {\noindent \textbf{\large \raisedrule[0.3em]{0.6pt} Exercise #1 实验过程}}
    \vspace{1em}
}

\NewDocumentEnvironment{answer}{}{
    {\noindent \textbf{\large Answer \raisedrule[0.3em]{0.6pt}}}
} {
    \par
    {\noindent \textbf{\large \raisedrule[0.3em]{0.6pt} Answer}}
    \vspace{1em}
}


% provides algorithm input support
\usepackage{algorithm}                  % algorithm pseudo code
\usepackage[noend]{algpseudocode}
\usepackage{caption}

% usage:
% \begin{simpleAlgorithm}{<algorithm-name>}
%     \Procedure{<procedure-name>}{$<parameters>$}
%     \State <statements>
%     \For{<condition>}
%     ...
%     \EndFor
%     \If{<condition>}
%         ...
%     \ElsIf{<condition>}
%         ...
%     \Else
%         ...
%     \EndIf
%     \State \Return <return val>
%     \EndProcedure
% \end{simpleAlgorithm}
%
% see $(texdoc algorithmic for details)
\NewDocumentEnvironment{simpleAlgorithm}{m}
{
    \captionsetup[algorithm]{aboveskip=0pt,belowskip=1pt}
    \vspace{0.5em}
    \hrule height 1.2pt
    \captionof{algorithm}{#1}
    \hrule height 0.5pt
    \begin{algorithmic}[1]
} {
    \end{algorithmic}
    \hrule height 1.2pt \vspace{1em}
}


%%%%%%%%%%%%%%%%%%%%%%%%%%%%%%%%
\graphicspath{{./res/}}

% report content %%%%%%%%%%%%
%%%%%%%%%%%%%%%%%%%%%%%%%%%%%
\begin{document}

% cover page
\begin{titlepage}
    \addtolength{\topmargin}{1cm}
    \centering
    \includegraphics[width=0.6\textwidth]{hust.jpg}\par
    \vspace{0.5cm}
    {\Huge \heiti 编译原理实验报告}\par
    \vspace{10cm}
    {
        \large
        \begin{tabular}{r m{8em}}
            \makebox[6em][s]{学生姓名}:& 胡思勖 \\ \cline{2-2}
            \makebox[6em][s]{学号}:& U201514898\\ \cline{2-2}
            \makebox[6em][s]{专业}:& 计算机科学与技术\\ \cline{2-2}
            \makebox[6em][s]{班级}:& 计卓1501\\ \cline{2-2}
            \makebox[6em][s]{指导教师}:& 徐丽萍\\ \cline{2-2}
        \end{tabular}
    }
    \vfill
    2018-06-23
\end{titlepage}

\setcounter{tocdepth}{1}
\pagenumbering{Roman}
\tableofcontents

\newpage
\pagenumbering{arabic}
\setcounter{page}{1}



% chapter 1
\chapter{Building a Cache Simulator}
\label{cha:building_a_cache_simulator}
\par 整个实验分为两个部分,在第一部分中,需要实现一个缓存模拟器。通过valgrind中的lackey工具\footnote{\url{http://valgrind.org/docs/manual/lk-manual.html}}可以得到一个程序几乎所有的内存访问情况,使用如下命令即可获得这些信息:
\inputCodeNoNumberSetLanguage{bash}
\begin{lstlisting}[numbers=none]
linux> valgrind --log-fd=1 --tool=lackey --trace-mem=yes <program>
\end{lstlisting}

\par 一个样例输出如下:
\begin{lstlisting}[numbers=none]
I 0400d7d4,8
 M 0421c7f0,4
 L 04f6b868,8
 S 7ff0005c8,8
\end{lstlisting}

\par 输出的格式为``操作\ 地址,大小'',I表示指令加载,L表示数据加载,S表示数据存储,M表示数据修改,而数据修改应该被当做一次数据加载加上一次数据存储。内存地址以十六进制的形式给出。

\section{实验要求}
\label{sec:shi_yan_yao_qiu_}

\par 此实验要求编写一个Cache模拟器,其输入为Valgrind输出的内存访问轨迹,输出为与csim-ref相同的统计数据。

\par 实验要求:
\begin{itemize}
    \item 模拟器必须在输入参数s、E、b设置为任意值时均能正确工作——即需要使用malloc函数(而不是代码中固定大小的值)来为模拟器中数据结构分配存储空间。
    \item 由于实验仅关心数据Cache的性能,因此模拟器应忽略所有指令cache访问(即轨迹中“I”起始的行)
    \item 假设内存访问的地址总是正确对齐的,即一次内存访问从不跨越块的边界——因此可忽略访问轨迹中给出的访问请求大小
    \item main函数最后必须调用printSummary函数输出结果,并如下传之以命中hit、缺失miss和淘汰/驱逐eviction的总数作为参数:
\end{itemize}

\par 在编写完成后,使用test-csim程序进行测试以及评分。Cache模拟器使用的策略应为LRU替换策略。

\section{实验设计}
\label{sec:shi_yan_she_ji_}

\subsection{总体设计}
\label{sub:zong_ti_she_ji_}

\par 需要修改的文件为csim.c。由于已经给出了框架,首先观察框架中的代码,其中包含以下函数:
\inputCodeSetLanguage{c}
\begin{lstlisting}
accessData(mem_addr_t addr)
freeCache()
initCache()
main(int argc,char * argv[])
printUsage(char * argv[])
replayTrace(char * trace_fn)
\end{lstlisting}

\par 根据函数名和注释可以得知,initCache和freeCache用于初始化和释放cache,printUsage用于打印帮助信息,而replayTrace则直接被主函数调用并调用accessData来模拟Cache的替换过程。而需要修改的函数就是accessData函数。

\par 接下来观察Cache是如何在C语言中被组织并模拟的。首先,文件中定义了如下的结构,来描述cache line,并定义其指针以及指针的指针分别为cache set和cache。
\begin{lstlisting}
typedef struct cache_line {
    char valid;
    mem_addr_t tag;
    unsigned long long int lru;
}
typedef cache_line_t *cache_set_t;
typedef cache_set_t *cache_t;
\end{lstlisting}

\par 而通过initCache中的初始化过程可以得知,文件中描述了一个如图\ref{fig:cacheStructure}所示的cache:

\begin{figure}[htb]
    \centering
    \includegraphics[width=0.72\linewidth]{cacheStructure.png}
    \caption{Cache 结构}
    \label{fig:cacheStructure}
\end{figure}
\FloatBarrier

\par 了解模拟cache在内存中的结构后,每一次的内存访问过程如下:首先判断此内存地址是否能够在对应位置cache hit,如果cache hit则将hit计数加一,否则判定为cache miss,将miss计数加一,然后判断是否需要替换。如果需要替换则按照LRU规则进行替换,然后将eviction计数加一。由于cache line是以数组的形式存在于内存中的,因此在实现LRU的队列结构的时候需要注意对于结构体中lru字段的操作。

\subsection{详细设计}
\label{sub:xiang_xi_she_ji_}
\par 接下来根据LRU的替换规则设计accessData中具体的cache替换流程。首先,根据传入的地址计算组索引以及tag,并根据组索引获得cache组。接下来对于组中的所有cache line进行遍历,若某一个cache line的tag字段与事先计算的tag字段相等,则增加一次cache hit计数,然后更新此cache set中所有cache line的lru字段。若遍历完成都没有cache hit,则一定出现了cache miss,此时将cache miss计数加一。但此时依然需要分类讨论:若此cache set未被填满,则不需要进行替换,直接将新的记录插入cache set中并更新所有cache line的lru字段,否则还需要进行cache替换。替换方法为:首先遍历次cache set中所有的cache line,找出其中lru值最大的cache line(存在时间最长的),将其替换为新的地址后更新其他所有cache line的lru字段。具体的替换流程如图\ref{fig:cacheSub}所示。

\begin{figure}[htb]
    \centering
    \includegraphics[width=0.7\linewidth]{cacheSub.png}
    \caption{cache更新流程}
    \label{fig:cacheSub}
\end{figure}

\par 在上述三种不同的情况中,三种lru的更新方法是不同的:在cache hit时,遍历所有cache line,将lru字段小于命中的cache的lru字段值,并且valid字段为1的所有cache line的lru值加一,然后将命中的cache line的lru值置为0。
\par 在cache miss但未发生cache eviction时,首先遍历所有cache line,找出第一个valid字为0的cache line,将这一cache line的valid字段置为1,lru字段置为0并将其他所有valid字段为1的cache line的lru值加一。
\par 若出现cache miss eviction,则遍历所有cache line,找出lru值最大的cache line,将其lru值置为0并将这一cache set中其他所有cache line的lru值加一。则这三种替换过程如\ref{fig:cacheSub123}所示。

\begin{figure}[htb]
    \centering
    \includegraphics[width=0.9\linewidth]{cacheSub123.png}
    \caption{三种情况下的cache替换过程}
    \label{fig:cacheSub123}
\end{figure}

\par 除了上述对于核心部分的设计之外,还需要考虑其他的部分:题目要求实现与csim-ref完全相同的功能,而csim-ref的参数可以指定-h、-v、-s、-E、-b、-t。其中-h在所给出的代码框架中已经得以实现,而-s、-E、-b、-t全部是为核心部分的cache模拟提供支持的。-v参数则需要添加额外的实现。使用-v参数得到的一个样例输出如下:
\inputCodeNoNumberSetLanguage{bash}
\begin{lstlisting}
L 10,1 miss
M 20,1 miss hit
L 22,1 hit
S 18,1 hit
L 110,1 miss eviction
L 210,1 miss eviction
M 12,1 miss eviction hit
hits:4 misses:5 evictions:3
\end{lstlisting}

\par 从输出中可以看出,-v对于除了I型指令之外的内存操作均进行了输出,输出格式与valgrind的输出格式类似,但在每行后面添加了cache hit/miss/eviction的情况。

\section{实验过程}
\label{sec:shi_yan_guo_cheng_}

\subsection{实验环境}
\label{sub:shi_yan_huan_jing_}
\begin{table}[htb]
    \centering
    \caption{实验环境配置}
    \label{tab:label}
    \begin{tabular}{r l}
        \toprule
        操作系统        & Archlinux x64 2018-04-11 更新\\
        编译器          & gcc 7.3.1 \\
        Makefile管理器  & gnu make 4.2.1 \\
        内存调试工具    & valgrind 3.13.0 \\
        版本管理工具    & git 2.17.0 \\
        \bottomrule
    \end{tabular}
\end{table}

\subsection{详细步骤}
\label{sub:shi_yan_guo_cheng_}

\par 有了上述设计后,代码的实现就较为容易了。首先,补全freeCache程序中的代码。根据initCache中的代码可知源程序是在cache这一二维数组的两个维度进行了分配,而从逻辑上可以推断出,在整个程序的运行过程中不需要对于内存进行新的分配,因此,在freeCache时只需要对应的将initCache中分配的内存逐个释放即可。实现的代码如下:
\inputCodeSetLanguage{c}
\begin{lstlisting}
void freeCache() {
    int i;
    for (i = 0; i < S; ++i)
        free(cache[i]);
    free(cache);
}
\end{lstlisting}

\par 接下来实现核心部分代码,也就是accessData模拟cache更新这一部分。首先通过如下代码计算组编号以及tag值,并根据组编号获得cache组。
\begin{lstlisting}
mem_addr_t set_index = (addr >> b) & set_index_mask;
mem_addr_t tag = addr >> (s + b);
cache_set_t cache_set = cache[set_index];
\end{lstlisting}

\par 然后对于cache set进行第一次遍历。在每一次循环中,对于是否命中进行判断,判断的依据为valid是否为1且tag是否与当前内存地址的tag相等。若命中则增加命中计数并进行上述替换过程1,然后直接在返回。此外,还需要注意对于-v参数的支持:如果verbosity flag为1,则需要输出``hit''提示。
\begin{lstlisting}
for (i = 0; i < E; ++i) {
        /* hit */
        if (cache_set[i].valid && cache_set[i].tag == tag) {
            if (verbosity)
                printf("hit ");
            ++hit_count;

            /* update entry whose lru is less than the current lru (newer) */
            for (int j = 0; j < E; ++j)
                if (cache_set[j].valid && cache_set[j].lru < cache_set[i].lru)
                    ++cache_set[j].lru;
            cache_set[i].lru = 0;
            return;
        }
    }
}
\end{lstlisting}

\par 如果上述循环执行完成而没有返回,表明没有发生cache hit,因此使用以下代码将miss技术加一,并在verbosity flag为1时打印``miss''提示。
\begin{lstlisting}
if (verbosity)
    printf("miss ");
++miss_count;
\end{lstlisting}

\par 接下来需要分情况讨论是否会发生cache eviction。经过考虑后发现,可以将不发生eviction情况下寻找最大lru条目的过程与发生eviction情况下寻找第一个invalid cache line的过程合并在同一个循环中,以提高cache模拟器的性能,合并后的循环如下:
\begin{lstlisting}
int j, maxIndex = 0;
unsigned long long maxLru = 0;
for (j = 0; j < E && cache_set[j].valid; ++j) {
    if (cache_set[j].lru >= maxLru) {
        maxLru = cache_set[j].lru;
            maxIndex = j;
        }
    }
}
\end{lstlisting}

\par 在上述循环结束后,通过$j$的值来判断是否发生了cache eviction。由于上述循环在遇到invalid cache line时会跳出,因此当$j$小于E时表明未发生cache eviction。也就是说,如果发生cache eviction,则$j==E$一定成立。因此通过以下代码进行cache的更新。
\begin{lstlisting}
if (j != E) {
    for (int k = 0; k < E; ++k)
        if (cache_set[k].valid)
            ++cache_set[k].lru;
    cache_set[j].lru = 0;
    cache_set[j].valid = 1;
    cache_set[j].tag = tag;
} else {
    if (verbosity)
        printf("eviction ");
    ++eviction_count;
    for (int k = 0; k < E; ++k)
        ++cache_set[k].lru;
    cache_set[maxIndex].lru = 0;
    cache_set[maxIndex].tag = tag;
}
\end{lstlisting}

\par 在上述代码中,$j\neq E$部分为不发生eviction的情况,$j==E$的部分为发生eviction的情况,分别按照章节\ref{sub:xiang_xi_she_ji_}中的方法进行cache的更新。

\par 在accessData中核心部分的代码完成后,需要完成replayTrace中的代码调用accessData完成对于内存访问过程中cache变化的模拟。使用fscanf读入每一行,然后根据访问类型进行对于accessData的调用:忽略I型访问,对于L型和S型访问调用accessData一次,而对于M型访问调用accessData两次,此外还需要注意verbosity flag对于输出的影响。代码如下:
\begin{lstlisting}
while (fscanf(trace_fp, " %c %llx,%d", buf, &addr, &len) > 0) {
    if (verbosity && buf[0] != 'I')
        printf("%c %llx,%d ", buf[0], addr, len);
    switch (buf[0]) {
        case 'I':
            break;
        case 'L':
        case 'S':
            accessData(addr);
            break;
        case 'M':
            accessData(addr);
            accessData(addr);
            break;
        default:
            break;
    }
    if (verbosity && buf[0] != 'I')
        putchar('\n');
}
\end{lstlisting}

\par 至此,所有需要填写的代码已补充完成。

\subsection{测试与分析}
\label{sub:jie_guo_fen_xi_}

\par 对于完成的代码进行测试:首先使用make对代码进行编译,然后直接运行./test-csim命令。实验的测试程序给出的测试样例如表\ref{tab:example}所示,其输出结果如图\ref{fig:result1}所示。

\begin{center}
    \captionof{table}{测试样例}
    \label{tab:example}
    \begin{longtable}{r c c c c c c}
        \toprule
        \multicolumn{1}{c}{\textbf{测试文件}} &
        \multicolumn{1}{c}{\textbf{组索引位数 s}} &
        \multicolumn{1}{c}{\textbf{关联度 E}} &
        \multicolumn{1}{c}{\textbf{块偏移位数 b}} &
        \multicolumn{1}{c}{\textbf{hit}} &
        \multicolumn{1}{c}{\textbf{miss}} &
        \multicolumn{1}{c}{\textbf{eviction}}            \\
        \cmidrule(lr){1-1} \cmidrule(lr){2-4} \cmidrule(lr){5-7}
        yi2.trace   & 1 & 1 & 1 & 9      & 8     & 6     \\
        yi.trace    & 4 & 2 & 4 & 4      & 5     & 2     \\
        dave.trace  & 2 & 1 & 4 & 2      & 3     & 1     \\
        trans.trace & 2 & 1 & 3 & 167    & 71    & 67    \\
        trans.trace & 2 & 2 & 3 & 201    & 37    & 29    \\
        trans.trace & 2 & 4 & 3 & 212    & 26    & 10    \\
        trans.trace & 5 & 1 & 5 & 213    & 7     & 0     \\
        long.trace  & 5 & 1 & 5 & 265189 & 21775 & 21743 \\
        \bottomrule
    \end{longtable}
\end{center}

\begin{figure}[htb]
    \centering
    \includegraphics[width=0.8\linewidth]{result1.png}
    \caption{test-csim输出结果}
    \label{fig:result1}
\end{figure}

\par 从图中可以看出,所有的测试通过。接下来对于-v选项进行测试。首先对于yi.trace的进行测试,然后对比处理其他trace文件的输出与csim-ref处理相同文件的输出。运行的结果如\ref{fig:result2}所示。

\begin{figure}[htb]
    \centering
    \includegraphics[width=0.95\linewidth]{result2.png}
    \caption{对于-v选项的测试}
    \label{fig:result2}
\end{figure}

\par 可以看出,csim的输出与csim-ref的输出完全相同。至此,所有的测试完成,csim能够完全复现csim-ref的功能。





% chapter 2
\chapter{基于pthread的形态学图像处理}
\section{实验目的与要求}
\begin{itemize}
    \item 掌握使用pthread的基本的并行编程设计方法以及调优方法;
    \item 掌握并行编程中基本的数据分块以及任务分解的方法。
    \item 使用pthread实现并行的形态学图像处理。
    \item 简要分析以及总结处理的结果。
\end{itemize}

\section{算法描述}
\par 使用多个线程对于一个图像进行蚀刻以及膨胀的算法如下,算法为一个线程的流程,而有多个这样的线程同时进行。
\begin{simpleAlgorithm}{pthread并行处理算法(一个线程)}
    \Procedure{PthreadParallel}{$blocks$}
    \While{true}
        \State lock\((blocks)\)
        \State get first block \(blk\) from \(blocks\)
        \If{\(blocks\).empty()}
            \State unlock\((blocks)\)
            \State \Return
        \EndIf
        \State unlock\((blocks)\)
        \State \Call{ErodeAndDilate}{$blk, kernel_e, kernel_d$}
    \EndWhile
    \EndProcedure
\end{simpleAlgorithm}
\par 算法中,\(blocks\)参数为一个工作队列,队列中的工作为原预处理过后的图片的子图片。在每个线程的每个循环中,首先锁住队列,从队列中获取一个子图片\(blk\)、解锁队列然后使用上一章中的ErodeAndDilate过程进行处理。如果\(blocks\)中没有子图片,说明处理完成,则此线程退出。
\par 主线程的流程如图\ref{fig:pthreadMain}所示。在进行预处理过后启动多个线程,然后等待所有线程竞争子图像、处理然后结束即可,最后保存处理的结果即可。
\begin{figure}[htpb]
    \centering
    \includegraphics[width=0.95\linewidth]{pthreadMain.png}
    \caption{主线程流程}
    \label{fig:pthreadMain}
\end{figure}

\par 由于是使用pthread的并行算法,每一个线程处理一个部分,因此首先需要将数据分块(即分为算法中的\(blocks\))。分块方式如图\ref{fig:partition}所示。每块大小一样,在边缘部分如果块大小不符则按照原图的边缘进行裁减。因此,在进行处理时需要对于边缘部分进行考虑。
\begin{figure}[htpb]
    \centering
    \includegraphics[width=0.76\linewidth]{partition.png}
    \caption{分块方法}
    \label{fig:partition}
\end{figure}

\section{实验方案}
\par 所有的开发与运行环境见附录\ref{cha:env},表\ref{tab:env},此后实验的开发与运行环境均相同,不再赘述。根据算法描述、分块方法以及主线程的流程编写程序并运行,然后观察结果并与串行的程序比较。经过多轮的比较以及参数调试后得出一个较好的效果。

\section{实验结果与分析}
\par 功能上,程序处理后的图片与串行处理后的图片一致,此处不再给出。4线程,分块大小为128的情况下程序的运行时间如图\ref{fig:pthreadOutput}所示。在4个线程的情况下,运行三次的平均运行时间为11.7s,相比于串行算法,程序的加速比为\(44.2\div 11.7 = 3.77\),已经十分接近理想加速比4。
\begin{figure}[htpb]
    \centering
    \includegraphics[width=0.9\linewidth]{pthreadOutput.png}
    \caption{pthread程序运行时间}
    \label{fig:pthreadOutput}
\end{figure}

\par 经过8组、每组3次的测试,加速比随线程变化的曲线如图\ref{fig:pthreadTrend}所示。可以看出,在线程数为1\textasciitilde 4时加速比随着线程数几乎呈线性变化,而在线程数为1时加速比为1.006,overhead所占用的时间几乎可以不计。在线程数达到4时由于物理内核已经被占满,因此后面加速比不再增加,随着线程数量的进一步增大,由于线程调度的开销,因此程序的加速比不再增加,反而有所下降。
\begin{figure}[htpb]
    \centering
    \includegraphics[width=0.8\linewidth]{pthreadTrend.png}
    \caption{加速比随线程数变化}
    \label{fig:pthreadTrend}
\end{figure}

\par 对于分块大小而言,加速比随着分块大小的变化如图\ref{fig:pthreadTrend2}所示,在分块大小较小时,加速比随着分块大小的变化并不大,只在分块大小过小时由于线程调度导致一点性能开销。当分块大小大于原图的一半时总时间则取决于分到最大分块线程所用的时间,因此在这个区间内性能随分块大小呈下降趋势。
\begin{figure}[htpb]
    \centering
    \includegraphics[width=0.8\linewidth]{pthreadTrend2.png}
    \caption{加速比随分块大小变化}
    \label{fig:pthreadTrend2}
\end{figure}




% chapter 3
\chapter{User Environment}
\label{cha:user_environment}

\section{User Environments and Exception Handling}
\par inc/env.h中包含了基本的环境定义。在kern/env.c中,可以看到内核维护了3个全局变量来存储环境。
\begin{itemize}
    \item struct Env *envs = NULL; //ALL environments
    \item struct Env *curenv = NULL; The current env
    \item static struct Env *env\_free\_list; //Free environment list
\end{itemize}
\par jOS开始运行以后,env指针将指向一个存放系统各种环境的Env结构体数组。jOS内核最大支持NENV个同时活动的环境。jOS内核使用env\_free\_list维护所有不同的Env结构体,类似于空闲链表。内核使用curenv来表示当前运行的环境。内核启动前这个变量是NULL。

\subsection{Environment State}
\par Env结构体在inc/env中定义:
\inputCodeSetLanguage{c}
\begin{lstlisting}
struct Env {
    struct Trapframe env_tf;	// Saved registers
    struct Env *env_link;		// Next free Env
    envid_t env_id;			    // Unique environment identifier
    envid_t env_parent_id;		// env_id of this env's parent
    enum EnvType env_type;		// Indicates special system environments
    unsigned env_status;		// Status of the environment
    uint32_t env_runs;		    // Number of times environment has run

    // Address space
    pde_t *env_pgdir;		    // Kernel virtual address of page dir
};
\end{lstlisting}
\par 其中:
\begin{itemize}
    \item env\_tf:定义在inc/trap.h中,用于存放环境停止运行时寄存器的值。切换为内核模式的时候也会保存寄存器的值。
    \item env\_link:指向env\_free\_list中的下一个Env,env\_free\_list空闲链表的第一个env环境。
    \item env\_id:内核储存env\_id环境的父用户环境id。
    \item env\_type:用来特定环境。
    \item env\_status:这个变量可能为以下几种值:
        \begin{itemize}
            \item ENV\_FREE:这个Env是不活跃的,也就是说在env\_free\_list中。
            \item ENV\_RUNNABLE:这个Env正在等待被处理器运行。
            \item ENV\_RUNNING:这个Env结构体代表了正在运行的环境。
            \item ENV\_NOT\_RUNNABLE:当前环境是活跃的但是不准备运行。比如等待其他环境进行进程间通信。
            \item ENV\_DYING:Env是一个僵尸环境,这个环境下一次进入内核的时候会被释放。
        \end{itemize}
    \item env\_pgdir:这个变量存放这个环境的页目录的虚拟地址。
\end{itemize}
\par 类似于Unix,一个jOS环境中结合了``线程''和``地址空间''的概念。线程是由保护寄存器定义的,而地址空间是由env\_pgdir指向的页目录和页表定义。

\subsection{Allocating the Environments Array}
\par 在lab2中修改了mem\_init()内为pages数组分配了空间,而现在需要进一步修改mem\_init()来为Env分配一个相似的结构envs。
\exercise{1}{
    \par 修改kern/pmap.c中的mem\_init()来为envs分配空间并建立映射。这个数组应该正好包含NENV个Env结构。并且这个envs应该映射到用户制度的UENVS,这样用户进程可以读取。可以使用check\_kern\_pgdir()来检查代码是否正确。
}
\begin{exerciseSolution}{1}
    \par 首先是分配数组,在分配pages的代码后添加为envs分配空间的代码:
    \inputCodeSetLanguage{c}
    \begin{lstlisting}
envs = (struct Env*)boot_alloc(NENV*sizeof(struct Env));
memset(envs, 0, NENV*sizeof(struct Env));
    \end{lstlisting}
    \par 在分配完内存空间之后,接下来准备映射。因此在对于pages的映射之后对envs进行映射,添加如下代码:
    \begin{lstlisting}
boot_map_region(kern_pgdir, UENVS, PTSIZE, PADDR(envs), PTE_U);
    \end{lstlisting}

    \par 修改完成以后,重新编译运行,结果如图\ref{fig:lab3/exercise1_1}所示。可以看到,显示check\_kern\_pgdir() succeeded!,也就是说exercise1实验成功。
    \begin{figure}[htb]
        \centering
        \includegraphics[width=0.8\linewidth]{lab3/exercise1_1.png}
        \caption{修改mem\_init后的运行结果}
        \label{fig:lab3/exercise1_1}
    \end{figure}
    \FloatBarrier
\end{exerciseSolution}

\subsection{Creating and Running Environments}
\par 现在需要完善kern/env.c使之能够运行一个用户环境。由于没有文件系统,因此必须将内核设置为能够加载内核中的静态二进制程序映像文件。
\par Lab3里面的GNUMakefile文件在obj/user目录下生成了一系列二进制文件。通过kern/Makefrag能够将这些二进制文件直接链接到可执行文件中。通过链接器中的-b binary选项能够使这些文件被作为二进制文件链接到内核之后。
\exercise{2}{
    \par 在env.c中,完成以下函数:
    \begin{itemize}
        \item env\_init()
            \par 初始化所有在envs数组中的Env结构,并将其加入env\_free\_list中。此外还需要调用env\_init\_percput来配置段式内存管理硬件来将所有的分段分为0级(内核)以及3级(用户)。
        \item env\_setup\_vm()
            \par 分配页目录,初始化用户环境地址空间中和内核相关的部分。
        \item region\_alloc()
            \par 为用户环境分配物理空间。
        \item load\_icode()
            \par 像boot loader一样分析一个ELF文件,并将它的内容加载到用户环境下。
        \item env\_create()
            \par 使用env\_alloc和load\_icode函数分配空间并加载一个ELF文件到用户环境中。
        \item env\_run()
            \par 在用户模式下开始一个用户环境。
    \end{itemize}
}
\begin{exerciseSolution}{2}
    \par 对于env\_init函数而言,遍历envs数组中的Env结构体,把每一个Env的end\_id置0。实现的env\_init的代码如下:
    \begin{lstlisting}
void env_init(void) {
    int counter;
    env_free_list = NULL;
    for (counter = NENV - 1; counter >= 0; --counter) {
        envs[counter].env_id = 0;
        envs[counter].env_status = ENV_FREE;
        envs[counter].env_link = env_free_list;
        env_free_list = &envs[counter];
    }
    // Per-CPU part of the initialization
    env_init_percpu();
}
    \end{lstlisting}

    \par 然后填写env\_setup\_vm部分。env\_setup\_vm的函数的作用是初始化新的用户环境页目录表,但是只设置夜幕里表中和内核相关的页目录,而不映射用户目录。因此可以使用kern\_pgdir来设置env\_pgdir中的内容。最终补充的代码如下:
    \begin{lstlisting}
++p->pp_ref;
e->env_pgdir = (pde_t *)page2kva(p);
memcpy(e->env_pgdir, kern_pgdir, PGSIZE);
    \end{lstlisting}

    \par 接下来补充为用户环境分配len字节的空间的函数,然后映射到环境中的虚拟地址va,根据提示,va向下对齐,va+len向上对齐。
    \begin{lstlisting}
static void region_alloc(struct Env *e, void *va, size_t len) {
    struct PageInfo *page = NULL;
    va = ROUNDDOWN(va, PGSIZE);
    void *end = (void *)ROUNDUP(va + len, PGSIZE);
    for (; va < end; va += PGSIZE) {
        if (!(page = page_alloc(ALLOC_ZERO)))
            panic("region_alloc: alloc failed.");
        if (page_insert(e->env_pgdir, page, va, PTE_U | PTE_W))
            panic("region_alloc: page mapping failed.");
    }
}
    \end{lstlisting}

    \par load\_icode需要加载ELF二进制到用户内存。参考boot/main.c中的boot loader加载内核到内存,首先验证ELF文件的合法性,然后加载ph->
p\_type = ELF\_PROG\_LOAD的字段,在加载前需要注意使用lcr3切换到用户态的页目录,否则不能够正确的加载到用户内存空间。在加载完成并将多余位清零后,映射初始栈的一个页,最终完成的代码如下:
    \begin{lstlisting}
static void load_icode(struct Env *e, uint8_t *binary) {
    struct Elf *elf_header = (struct Elf *)binary;
    if (elf_header->e_magic != ELF_MAGIC)
        panic("load_icode: illegal ELF format.");
    lcr3(PADDR(e->env_pgdir));
    struct Proghdr *ph = (struct Proghdr *)((uint8_t *)(elf_header) + elf_header->e_phoff);
    struct Proghdr *eph = ph + elf_header->e_phnum;
    for (; ph < eph; ++ph) {
        if (ph->p_type == ELF_PROG_LOAD) {
            region_alloc(e, (void *)ph->p_va, ph->p_memsz);
            memmove((void *)ph->p_pa, binary + ph->p_offset, ph->p_filesz);
            memset((void *)(ph->p_pa + ph->p_filesz), 0, ph->p_memsz - ph->p_filesz);
        }
    }
    e->env_tf.tf_eip = elf_header->e_entry;
    lcr3(PADDR(kern_pgdir));
    region_alloc(e, (void *)(USTACKTOP - PGSIZE), PGSIZE);
}
    \end{lstlisting}

    \par 对于env\_create首先使用env\_alloc创建一个env,然后调用load\_icode来加载elf二进制镜像,最后设置env\_type。值得注意的是env的父id应该设置为0,其实现如下:
    \begin{lstlisting}
void env_create(uint8_t *binary, enum EnvType type) {
    struct Env *environment;
    if(env_alloc(&environment, 0))
        panic("env_create: env_alloc failed.");
    load_icode(environment, binary);
    environment->env_type = type;
}
    \end{lstlisting}

    \par 对于env\_run而言,首先切换判断当前环境是否为空,环境状态是否为ENV\_RUNNING,如果是则将环境设置为ENV\_RUNNABLE,然后将curenv设置为当前环境。设置状态ENV\_RUNNING,更新env\_runs计数器后切换到它的地址空间。使用env\_pop\_tf换源环境寄存器然后进入用户模式。实现的代码如下:
    \begin{lstlisting}
void env_run(struct Env *e) {
    if(curenv && curenv->env_status == ENV_RUNNING)
        curenv->env_status = ENV_RUNNABLE;
    curenv = e;
    curenv->env_status = ENV_RUNNING;
    ++curenv->env_runs;
    lcr3(PADDR(curenv->env_pgdir));
    env_pop_tf(&curenv->env_tf);
}
    \end{lstlisting}
\end{exerciseSolution}

\par 用户环境的代码被调用前,操作系统一共按顺序执行了以下几个函数:
\begin{itemize}
    \item start (kern/entry.S)
    \item i386\_init (kern/init.c)
        \begin{itemize}
            \item cons\_init
            \item mem\_init
            \item env\_init
            \item trap\_init (still incomplete at this point)
            \item env\_create
            \item env\_run
                \begin{itemize}
                    \item env\_pop\_tf
                \end{itemize}
        \end{itemize}
\end{itemize}
\par 完成上述函数的代码后重新编译运行,系统会进入用户空间并且开始执行hello程序,直到系统调用int指令。这个指令不能成功执行,因为jOS还没有设置相关硬件来实现从用户态向内核态转换的功能。当CPU发现它不能处理这种中断时会触发一个异常,然后发现这个异常也无法处理,直到产生第三个异常,但仍旧不能解决,因此将其叫做"triple fault"。我们可以使用调试器检查我们是否进入了用户模式。使用make qemu-gdb并在env\_pop\_tf处设置一个断点,然后单步执行,处理器会在执行完iret指令以后进入用户模式。该进入用户模式的第一条指令是一个cmp指令。然后使用b *0x...设置一个在obj/user/hello.asm中的断点中的sys\_cputs函数的int \$0x30处。这个int指令是一个系统调用,用来向控制台输出一个字符。如果你的程序不能运行到int指令说明程序有错误。
\par 按照上述过程进行调试,程序停止在了int \$0x30处,如图\ref{fig:lab3/exercise2_1}所示。说明程序功能正常。
\begin{figure}[htb]
    \centering
    \includegraphics[width=0.9\linewidth]{lab3/exercise2_1.png}
    \caption{程序停止在int \$0x30处}
    \label{fig:lab3/exercise2_1}
\end{figure}

\subsection{Handling Interrupts and Exceptions}
\par 现在需要一个异常处理及系统调用处理机制来使系统从用户态切换到内核态。
\exercise{3}{
    \par 阅读\textit{Chapter 9, Exceptions and Interrupts }\footnote{\url{https://pdos.csail.mit.edu/6.828/2017/readings/i386/c09.htm}}
}
%\begin{exerciseSolution}{3}
%    \par 从文中可以知道,中断分为可屏蔽中断以及不可屏蔽中断;异常分为处理器检测异常以及程序触发的异常。通过NMI、IF、RF以及修改SS可以使能或屏蔽中断。中断描述符表储存中断处理程序的入口地址,而中断的处理流程如图\ref{fig:lab3/exercise3_1}所示,通过IDT与GDT共同决定用于处理的程序。
%    \begin{figure}[htb]
%        \centering
%        \includegraphics[width=0.6\linewidth]{lab3/exercise3_1.png}
%        \caption{中断处理流程}
%        \label{fig:lab3/exercise3_1}
%    \end{figure}
%\end{exerciseSolution}

\subsection{Basics of Protected Control Transfer}
\par 异常和中断都是保护控制转移,让处理器从用户模式切换到内核模式。这样用户代码不会对内核造成任何影响。在intel处理器中,中断通常是外部设备引起的异步的保护控制转移,而异常则是由当前运行的代码引起的同步的保护控制转移。
\par 为了能够保证这些控制转移真的能被保护,处理器的中断/异常机制通常为用户态代码无权选择内核代码的执行起点,处理器只有在某些条件下才能进入内核态。在x86上,有2中机制配合来提供这种保护:
\begin{enumerate}
    \item 中断向量表:
        \par 处理器保证撞断和异常只能导致内核进入一些预先定好的入口。x86处理器可以有最多256个不同的中断和异常,而每一个都对应一个唯一的中断向量。一个中断向量的值是根据中断的来源决定的。CPU将使用这个向量作为中断向量表的索引,而这个表又是内核设置的。通过表项处理器会加载:
        \begin{itemize}
            \item 加载到EIP寄存器的值,也就是指向处理这种类型异常的内核代码指针。
            \item 加载到CS寄存器的值,包含特权级别0\textasciitilde 1。
        \end{itemize}
    \item 任务状态段:
        \par 处理器需要存放中断异常发生之前的旧的处理器状态,包括原EIP和CS值,以在中断处理之后能够还原到之前的状态。保存的这个位置必须要受到保护,不能随意被修改。
        \par 因此,处理西在处理中断时会导致特权级别由用户转级为内核级,将堆切换到内核内存中。处理器将SS, ESP, EFLAGS, CS, EIP和可选的错误码压入堆栈中,然后从中断描述符中加载CS和EIP,设置ESP和SS指向新的堆栈。
\end{enumerate}

\subsection{Types of Exceptions and Interrupts}
\par 所有的x86处理器内部产生的议程向量是0\textasciitilde 31之间的整数,也映射到了IDT的0\textasciitilde 31项。大于31的项只被软件中断所使用,也就是说可以被int触发,或者是异步的硬件中断。
\par 这一节将要扩展jOS的功能使之能够处理0\textasciitilde 31号的内部异常。下一节会让jOS处理48号软件中断。

\subsection{An Example}
\par 在这个例子中,假设处理器遇到了除0的问题。
\begin{enumerate}
    \item 处理器切换到TSS的SS0和ESP0对应的堆栈,在jOS中,这两个字段是GD\_KG和KSTACKTOP。
    \item 处理器将异常参数压入内核堆栈,并放在KSTACKTOP中。
        \inputCodeSetLanguage{bash}
        \begin{lstlisting}[numbers=none]
+--------------------+ KSTACKTOP
| 0x00000 | old SS   |     " - 4
|      old ESP       |     " - 8
|     old EFLAGS     |     " - 12
| 0x00000 | old CS   |     " - 16
|      old EIP       |     " - 20 <---- ESP
+--------------------+
        \end{lstlisting}
   \item 由于处理器错误在x86上是0号中断向量,因此去读IDT的第0项并设置CS:EIP指向中断处理程序。
   \item 处理函数接过控制权并处理异常。
\end{enumerate}
\par 对于确定类型的x86异常,除了上面标准的5个压栈元素外还有一个错误码。当处理器将错误码压栈时,栈是这样的:
\begin{lstlisting}[numbers=none]
+--------------------+ KSTACKTOP
| 0x00000 | old SS   |     " - 4
|      old ESP       |     " - 8
|     old EFLAGS     |     " - 12
| 0x00000 | old CS   |     " - 16
|      old EIP       |     " - 20
|     error code     |     " - 24 <---- ESP
+--------------------+
\end{lstlisting}

\subsection{Nested Exceptions and Interrupts}
\par 处理器在内核模式和用户模式都可以处理异常和中断。但是当内核从用户态进入内核态的时候,x86处理器会在压入旧的寄存器之前自动切换栈并通过IDT触发异常处理。如果当中断或异常发生时处理器已经在内核态了,那么CPU会在同一个栈上压入更多的值。这样,内核就能够处理嵌套中断;了。如果处理器已经在内核模式且正在处理嵌套异常,就不会保存SS和ESP寄存器,因此堆栈如下:
\begin{lstlisting}[numbers=none]
+--------------------+ <---- old ESP
|     old EFLAGS     |     " - 4
| 0x00000 | old CS   |     " - 8
|      old EIP       |     " - 12
+--------------------+
\end{lstlisting}
\par 如果处理器在内核模式处理异常,但栈空间不足,不能将旧的状态压入堆栈,那么处理器之后就不能恢复,只能重启。内核应该被设计为不允许这种事情发生。

\subsection{Setting Up the IDT}
\par 现在可以设置IDT表并处理JOS的内部异常了(中断向量0\textasciitilde 31)。最终需要实现的代码效果如下:
\inputCodeSetLanguage{bash}
\begin{lstlisting}[numbers=none]
       IDT              trapentry.S       trap.c
+----------------+
|   &handler1    |---> handler1:        trap (struct Trapframe *tf)
|                |        // do stuff    {
|                |        call trap        // handle the exception/interruput
|                |        // ...         }
+----------------+
|   &handler2    |---> handler2:
|                |       // do stuff
|                |       call trap
|                |       // ...
+----------------+
        ...
+----------------+
|   &handlerX    |---> handlerX:
|                |        // do stuff
|                |        call trap
|                |        // ...
+----------------+
\end{lstlisting}
\par 每一个中断或者异常结构都有它的中断处理函数,定义在trapentry.S中。trap\_init()初始化IDT表。每个处理函数都应该构建一个在Trapframe堆栈上的结构体,并调用trap()函数指向它。trap()则处理异常/中断。

\exercise{4}{
    \par 编辑trap.S以及trap.c并实现上述功能。对于每一个定义在inc/trap.h中的trap,都应该有一个函数入口应该被加在trapentry.S中。应该提供一个\_alltraps供TRAPHANDLER宏引用。要初始化idt表需要修改trap\_init函数,使表中的每一项指向定义在 trapentry.S 中的入口指针。实现的\_alltraps函数应该:
    \begin{enumerate}
        \item 将值压入堆栈,使堆栈看起来像一个Trapframe
        \item 加载GD\_KD进入\%ds以及\%es
        \item 使用pusl \%esp给Trapframe传递指针,作为trap()的参数
        \item 调用trap
    \end{enumerate}
}
\begin{exerciseSolution}{4}
    \par 首先,trapentry.S 中的宏定义TRAPHANDLER以及TRAPHANDLER\_NOEC定义了发生中断或异常时用于初始处理的函数。因为有些中断有错误码,有些没有,因此需要两个函数。通过参考\textit{80386 Programmer’s Manual 9.10 Error Code Summary}\footnote{\url{https://pdos.csail.mit.edu/6.828/2016/readings/i386/s09_10.htm}}可以知道哪些中断有错误码。因此trapentry.S修改如下:

\inputCodeSetLanguage{[x86masm]Assembler}
\begin{lstlisting}
TRAPHANDLER_NOEC(handler_divide,  T_DIVIDE)
TRAPHANDLER_NOEC(handler_debug,   T_DEBUG)
TRAPHANDLER_NOEC(handler_nmi,     T_NMI)
TRAPHANDLER_NOEC(handler_brkpt,   T_BRKPT)
TRAPHANDLER_NOEC(handler_oflow,   T_OFLOW)
TRAPHANDLER_NOEC(handler_bound,   T_BOUND)
TRAPHANDLER_NOEC(handler_illop,   T_ILLOP)
TRAPHANDLER_NOEC(handler_device,  T_DEVICE)
TRAPHANDLER_NOEC(handler_simderr, T_SIMDERR)
TRAPHANDLER_NOEC(handler_fperr,   T_FPERR)
TRAPHANDLER_NOEC(handler_mchk,    T_MCHK)
TRAPHANDLER_NOEC(handler_syscall, T_SYSCALL)
TRAPHANDLER(handler_dblflt, T_DBLFLT)
TRAPHANDLER(handler_tss,    T_TSS)
TRAPHANDLER(handler_segnp,  T_SEGNP)
TRAPHANDLER(handler_stack,  T_STACK)
TRAPHANDLER(handler_gpflt,  T_GPFLT)
TRAPHANDLER(handler_pgflt,  T_PGFLT)
TRAPHANDLER(handler_align,  T_ALIGN)

_alltraps:
    pushl %ds
    pushl %es
    pushal
    movw $GD_KD, %eax
    movw %ax, %ds
    movw %ax, %es
    pushl %esp
    call trap
\end{lstlisting}
\par 然后在trap.c中完成trap\_init,对于系统的IDT表进行初始化:
\inputCodeSetLanguage{c}
\begin{lstlisting}
void handler_divide();
void handler_debug();
void handler_nmi();
void handler_brkpt();
void handler_oflow();
void handler_bound();
void handler_illop();
void handler_device();
void handler_simderr();
void handler_fperr();
void handler_mchk();
void handler_syscall();
void handler_dblflt();
void handler_tss();
void handler_segnp();
void handler_stack();
void handler_gpflt();
void handler_pgflt();
void handler_align();

void
trap_init(void)
{
    extern struct Segdesc gdt[];

    SETGATE(idt[T_DIVIDE],  0, GD_KT, handler_divide,  0);
    SETGATE(idt[T_DEBUG],   0, GD_KT, handler_debug,   0);
    SETGATE(idt[T_NMI],     0, GD_KT, handler_nmi,     0);
    SETGATE(idt[T_BRKPT],   0, GD_KT, handler_brkpt,   3);
    SETGATE(idt[T_OFLOW],   0, GD_KT, handler_oflow,   0);
    SETGATE(idt[T_BOUND],   0, GD_KT, handler_bound,   0);
    SETGATE(idt[T_ILLOP],   0, GD_KT, handler_illop,   0);
    SETGATE(idt[T_DEVICE],  0, GD_KT, handler_device,  0);
    SETGATE(idt[T_SIMDERR], 0, GD_KT, handler_simderr, 0);
    SETGATE(idt[T_FPERR],   0, GD_KT, handler_fperr,   0);
    SETGATE(idt[T_MCHK],    0, GD_KT, handler_mchk,    0);
    SETGATE(idt[T_SYSCALL], 0, GD_KT, handler_syscall, 3);
    SETGATE(idt[T_DBLFLT],  0, GD_KT, handler_dblflt,  0);
    SETGATE(idt[T_TSS],     0, GD_KT, handler_tss,     0);
    SETGATE(idt[T_SEGNP],   0, GD_KT, handler_segnp,   0);
    SETGATE(idt[T_STACK],   0, GD_KT, handler_stack,   0);
    SETGATE(idt[T_GPFLT],   0, GD_KT, handler_gpflt,   0);
    SETGATE(idt[T_PGFLT],   0, GD_KT, handler_pgflt,   0);
    SETGATE(idt[T_ALIGN],   0, GD_KT, handler_align,   0);

    // Per-CPU setup
    trap_init_percpu();
}
\end{lstlisting}
\par 完成后使用make grade进行测试,可以发现divzero,softint以及badsegment测试通过,如图\ref{fig:lab3/exercise4_1}所示,说明IDT初始化正确。
\begin{figure}[htb]
    \centering
    \includegraphics[width=0.8\linewidth]{lab3/exercise4_1.png}
    \caption{使用make grade进行测试}
    \label{fig:lab3/exercise4_1}
\end{figure}
\FloatBarrier
\end{exerciseSolution}

\begin{questionEnv}
    \begin{enumerate}
        \item 对于每一个中断/异常都设置一个独立的处理函数的意义是什么?
        \item 你做了让user/softint正确执行的工作吗?grade script希望它产生一个general protection falt(trap 13),但是softint中为int \$14。为什么产生了中断向量13?如果系统允许int \$14调用kernel page fault处理函数?
    \end{enumerate}
\end{questionEnv}
\begin{answer}
    \begin{enumerate}
        \item 因为不同的中断可能需要不同的处理方式,比如有些中断需要返回,有些中断则不需要,还有些中断需要做额外的工作。
        \item 应为当先系统在用户态,而int为特权级别为0的指令,此时不能直接调用int指令,会引发general protection exception。如果允许int \$14处理,那么会导致用户态程序可能得到0级特权,造成保护失效。
    \end{enumerate}
\end{answer}

\section{Page Faults, Breakpoints Exceptions, and System Calls}
\subsection{Handling Page Faults}
\par 当处理器产生一个缺页异常时,它会将因此缺页异常的线性地址(虚拟的)存入CR2中。在trap.c中我们提供了一个特殊的函数page\_fault\_handler()用于处理缺页异常。
\exercise{5}{
    \par 修改trap\_dispatch函数使系统能够把缺页异常分发到page\_fault\_handler上。修改完成后运行make grade应该可以成功通过faultread, faultreadkernel, faultwrite, 以及faultwritekernel检查。
}
\begin{exerciseSolution}{5}
    \par trap\_dispatch是一个分发函数,通过Trapframe指针tf中的tf\_trapno来判断这个中断是什么中断。而在这一个exercise中,如果中断是缺页中断则调用page\_fault\_handler函数。考虑到之后可能要添加的中断类型,在trap\_dispatch中添加的代码如下:
    \inputCodeSetLanguage{c}
    \begin{lstlisting}
switch(tf->tf_trapno){
    case T_PGFLT:
        page_fault_handler(tf);
        break;
}
    \end{lstlisting}
    \par 重新编译运行后,通过了题目描述中的4个测试,如图\ref{fig:lab3/exercise5_1}所示。说明这部分代码实现成功。
    \begin{figure}[htb]
        \centering
        \includegraphics[width=0.8\linewidth]{lab3/exercise5_1.png}
        \caption{重新编译运行后的make grade输出}
        \label{fig:lab3/exercise5_1}
    \end{figure}
\end{exerciseSolution}

\subsection{The Breakpoint Exception}
\par 异常编号为3的断点异常能够让调试器给程序加上断点,也就是将要加断点的语句用一个int3指令替换,然后在执行到int3时触发中断。在jOS中需要将这个中断变为任何用户环境都能调用的伪系统调用。
\exercise{6}{
    \par 修改trap\_dispatch,使断点异常发生时能够调用kernel monitor。修改完成后重新make grade应该能够通过breakpoint测试。
}
\begin{exerciseSolution}{6}
    \par 与exercise 5类似,但是这里处理的是T\_BRKPT。调用kernel monitor需要使用kern/monitor.c中的monitor函数。修改后的trap\_dispatch中的前半部分内容如下:
    \inputCodeSetLanguage{c}
    \begin{lstlisting}
switch(tf->tf_trapno){
    case T_PGFLT:
        page_fault_handler(tf);
        break;
    case T_BRKPT:
        monitor(tf);
        break;
}
    \end{lstlisting}
    \par 修改完成后,成功通过breakpoint测试,如图\ref{fig:lab3/exercise6_1}所示。
    \begin{figure}[htb]
        \centering
        \includegraphics[width=0.8\linewidth]{lab3/exercise6_1.png}
        \caption{程序通过breakpoint测试输出}
        \label{fig:lab3/exercise6_1}
    \end{figure}
\end{exerciseSolution}

\begin{questionEnv}
    \begin{enumerate}
        \setcounter{enumi}{2}
        \item breakpoint exeception测试用例会征程一个breakpoint 异常或者general protection 错误,依赖于如何初始化IDT中的breakpoint entry。为什么?要怎么做才能让breakpoint exception正常工作?怎样的错误设置会导致触发general protection error?
        \item 这些机制有什么意义?尤其是对于user/softint中的测试程序而言?
    \end{enumerate}
\end{questionEnv}
\begin{answer}
    \begin{enumerate}
        \setcounter{enumi}{2}
        \item 如果在IDT中设置breakpoint exception时将DPL字段设置为0则会触发breakpoint exception,设置为3则会触发general protection exception。DPL字段为段描述符优先级,如果当前程序为用户态但是尝试调用内核态的指令的时候就会触发general protection exception。只有当前程序的优先级小于或等于段描述符优先级才能触发正确的breakpoint exception。
        \item 这些机制保证用户环境不能随意访问内核态的代码和内存,保护内核不受用户程序的破坏。
    \end{enumerate}
\end{answer}

\subsection{System calls}
\par 用户程序会要求内核通过系统调用的方式帮其完成一些任务。当用户程序触发系统调用时,处理器进入内核态并保存用户的处理状态。内核处理完成后返回用户程序,但具体的细节随系统的不同而不同。
\par jOS使用int来处理系统调用。特别的,使用int \$0x30作为系统调用中断。常量T\_SYSCALL就是0x30。0x30不能被外部硬件产生,因此没有任何歧义。
\par 应用程序会把系统调用和参数放到寄存器中,通过这种方法内核就不需要查询用户程序的堆栈了。系统调用号存放到\%eax中,参数则存放在 \%edx, \%ecx, \%ebx, \%edi, 和 \%esi中。返回值存放到\%eax@s中。lib/syscall.c中已有了触发系统调用的方法。
\exercise{7}{
    \par 通过编辑kern/trapentry.S以及kern/trap.c的trap\_init(),给T\_SYSCALL添加一个中断向量处理函数。同时trap\_dispatch也需要被修改,通过调用syscall的方法来处理系统调用。最后,需要在kern/syscall.c中首先实现syscall函数。如果系统调用号不合法,需要syscall返回-E\_INVAL。
    \par 通过make run-hello运行user/hello,qemu应该打印处hello, world,并触发一个page fault。并且make grade应该能够通过testbss测试。
}
\begin{exerciseSolution}{7}
    \par 在用户态执行系统调用时,首先产生了中断30,因此在kern/trapentry.S中添加一个处理函数声明TRAPHANDLER\_NOEC(handler\_syscall, T\_SYSCALL),并在trap\_init中添加handler\_syscall的声明以及在trap\_init中通过SETGATE(idt[T\_SYSCALL], 0, GD\_KT, t\_syscall, 3);将其注册到IDT,这些在exercise 5中已经完成,此时系统已经能够正确捕捉int 30了。
    \par 观察lib/syscall.c的syscall,发现其就是执行了int指令并取回了返回值,而对于这条int指令的处理,则是在kern/syscall.c中进行的。首先,完成kern/trap.c中对于中断的分发,也就是在switch中加入如下几行:
    \inputCodeSetLanguage{c}
    \begin{lstlisting}
case T_SYSCALL:
    tf->tf_regs.reg_eax = syscall(
            tf->tf_regs.reg_eax,
            tf->tf_regs.reg_edx,
            tf->tf_regs.reg_ecx,
            tf->tf_regs.reg_ebx,
            tf->tf_regs.reg_edi,
            tf->tf_regs.reg_esi);
    break;
    \end{lstlisting}
    \par 然后在kern/syscall.c中完成对于syscall的实现,从而完成对于整个int指令的调用:
    \begin{lstlisting}
int32_t syscall(uint32_t syscallno, uint32_t a1,
        uint32_t a2, uint32_t a3, uint32_t a4, uint32_t a5) {
    switch (syscallno) {
        case SYS_cputs:
            sys_cputs((char *)a1, a2);
            return 0;
        case SYS_cgetc:
            return sys_cgetc();
        case SYS_getenvid:
            return sys_getenvid();
        case SYS_env_destroy:
            return sys_env_destroy(a1);
        default:
            return -E_INVAL;
    }
}
    \end{lstlisting}
    \par 填写完成后,重新编译并运行qemu,输出hello world并触发缺页中断。如图\ref{fig:lab3/exercise7_1}所示。
    \begin{figure}[htb]
        \centering
        \includegraphics[width=0.8\linewidth]{lab3/exercise7_1.png}
        \caption{输出hello world并触发缺页中断}
        \label{fig:lab3/exercise7_1}
    \end{figure}
    \par 运行make grade,成功通过testbss测试,如图\ref{fig:lab3/exercise7_2}所示,说明系统调用实现成功。
    \begin{figure}[htb]
        \centering
        \includegraphics[width=0.8\linewidth]{lab3/exercise7_2.png}
        \caption{成功通过testbss测试}
        \label{fig:lab3/exercise7_2}
    \end{figure}
\end{exerciseSolution}

\subsection{User-mode startup}
\par 用户模式开始运行的地方是lib/entry.S。在该文件中首先进行一些设置,然后调用libmain。应该修改libmain()来初始化全局指针thisenvz指向envs数组中的Env结构体。
\par 然后libmain调用main,也就是user/hello.c中被调用的函数。在之前的实验中发现hello.c只能打印hello world并报出page fault异常,而其原因就是this->env\_id语句。如果正确初始化了thisenv就不会报错了。
\exercise{8}{
    \par 补全用户库中的代码并启动内核,user/hello应该打印出hello, world然后打印i am environment 00001000。通过调用sys\_env\_destroy(),user/hello会尝试退出。由于内核仅仅支持一个用户环境,因此它应该显示用户环境已被销毁的信息,然后退回kernel monitor。完成后make grade应该能够通过hello test测试。
}
\begin{exerciseSolution}{8}
    \par 在libmain中,修thisenv让其指向env当前环境的env即可。使用sys\_getenvid来获得当前的环境id。因此将libmain中的thisenv修改为如下即可。
    \inputCodeSetLanguage{c}
    \begin{lstlisting}
thisenv = envs + ENVX(sys_getenvid());
    \end{lstlisting}
    \par 重新编译运行,并运行make grade,输出如图\ref{fig:lab3/exercise8_1}所示,此时能够通过hello测试。
    \begin{figure}[htb]
        \centering
        \includegraphics[width=0.8\linewidth]{lab3/exercise8_1.png}
        \caption{程序通过hello测试}
        \label{fig:lab3/exercise8_1}
    \end{figure}
\end{exerciseSolution}

\subsection{Page faults and memory protection}
\par 内存保护是操作系统的一个非常重要的特性,能够保证bug不能够损坏其他的程序或者操作系统。操作系统通常依赖于硬件来完成这一功能。操作系统能够让硬件知道那笑虚拟地址是有效的。当程序尝试访问一个无效地址或者越权操作时就会触发异常。如果异常是可以修复的,那么就会修复异常并继续运行程序,否则就不会继续运行。
\par 在许多操作系统中,内核在初始情况下只会分配一个内核堆栈。如果程序想要访问这个堆栈之外的堆栈空间,就会触发异常,内核会自动分配页个程序然后继续让程序运行。
\par 但是系统调用需要存在问题,大部分系统系统调用接口让用户传递一个指向用户缓冲区的指针给内核,但是:
\begin{enumerate}
    \item 内核中的page fault比用户中的page fault严重。如果内核出现page fault,那么这是内核bug,而且异常处理会中断内核的执行。但是当内核解引用用户程序的时候,它需要一种方法标记这些page fault确实是由用户引起的。
    \item 内核通常比用户程序由更高级别的权限。用户程序可能会传递一个内核可读写但是用户不行的指针。此时内核不能对其进行解引用,否则可能泄露内核信息。
\end{enumerate}

\begin{exerciseEnv}{9}
    \par 修改kern/trap.c,使其能够检测内核模式下的page fault发生并发出kernel panic。阅读kern/pmap.c中的use\_mem\_assert并实现其中的use\_mem\_check。修改ker/syscall.c检查输入参数。
    \par 启动内核并运行user/buggyhello。环境应该被摧毁,但是不应该出现kernel panic。应该能够看到:
    \inputCodeSetLanguage{bash}
    \begin{lstlisting}[numbers=none]
[00001000] user_mem_check assertion failure for va00000001
[00001000] free env 00001000
Destroyed the only environment - nothing more to do!
    \end{lstlisting}
    \par 最后,修改kern/kdebug.c中的debuginfo\_eip来运行user\_mem\_check检查use, stabs和stabstr。如果现在运行user/breakpoint,应该能够从内核监视器中运行backtrace来检查在page fault之前检查lib/libmain.c。是什么导致了page fault?
\end{exerciseEnv}

\begin{exerciseSolution}{9}
    \par 首先检查page fault是否在内核模式,即检查Trap Frame中的tf\_cs,在page\_fault\_handler中添加的代码如下:
    \inputCodeSetLanguage{c}
    \begin{lstlisting}
if(tf->tf_cs == GD_KT)
    panic("page_fault_handler: kernel page fault");
    \end{lstlisting}
    \par pmap.c中的user\_mem\_check。阅读user\_mem\_assert, 可以发现它调用了user\_mem\_check。然而user\_mem\_check的功能当前永辉态程序是否有对于$[va, va+len)$的perm|PTE\_P的访问权限。因此在user\_mem\_check中查看用户态程序中的页表项,然后检查其perm|PTE\_P。最终实现的程序如下:
    \begin{lstlisting}
int user_mem_check(struct Env *env, const void *va, size_t len, int perm) {
    uint32_t start = (uint32_t)ROUNDDOWN(va, PGSIZE);
    uint32_t end = (uint32_t)ROUNDUP(va + len, PGSIZE);
    pte_t *page;
    for (; start < end; start += PGSIZE) {
        page = pgdir_walk(env->env_pgdir, (void *)start, 0);
        if (!page || start > ULIM || ((uint32_t)(*page) & perm) != perm ) {
            if (start <= (uint32_t)va)
                user_mem_check_addr = (uintptr_t)va;
            else
                user_mem_check_addr = (uintptr_t)start;
            return -E_FAULT;
        }
    }
    return 0;
}
    \end{lstlisting}
    \par 接下来对于kern/syscall.c进行补全。通过观察发现需要补全的是sys\_cputs函数。通过注释可以发现需要用户程序检查用户对于虚拟地址空间$[s, s+len)$是否具有访问权限,而这个则可以用上面实现的user\_mem\_assert实现。这个函数补全后如下:
    \begin{lstlisting}
static void sys_cputs(const char *s, size_t len) {
    user_mem_assert(curenv, s, len, 0);
    cprintf("%.*s", len, s);
}
    \end{lstlisting}
    \par 最后修改kern/kdegbug.c中的debuginfo\_eip。添加如下代码:
    \begin{lstlisting}
if(user_mem_check(curenv, usd, sizeof(struct UserStabData), PTE_U))
    return -1;
    \end{lstlisting}
    \par 运行make run-breakpoint,显示如图\ref{fig:lab3/exercise9_1}所示。可以看到,输入backtrace能够显示backtrace之前能够追踪进入libmain.c
    \begin{figure}[htb]
        \centering
        \includegraphics[width=0.7\linewidth]{lab3/exercise9_1.png}
        \caption{运行user/breakpoint后的结果}
        \label{fig:lab3/exercise9_1}
    \end{figure}
    \par 通过gdb进行追踪,发现是由于执行mon\_backtrace时进行追踪时到达了用户栈顶,然后在打印参数时访问的第六个参数超过了用户栈的大小,如图\ref{fig:lab3/exercise9_2}所示。
    \begin{figure}[htb]
        \centering
        \includegraphics[width=0.9\linewidth]{lab3/exercise9_2.png}
        \caption{使用gdb进行追踪}
        \label{fig:lab3/exercise9_2}
    \end{figure}
    \par 最后,使用make grade 进行测试,发现能够正常通过所有测试,如图\ref{fig:lab3/exercise9_3}所示。
    \begin{figure}[htb]
        \centering
        \includegraphics[width=0.6\linewidth]{lab3/exercise9_3.png}
        \caption{make grade通过所有测试}
        \label{fig:lab3/exercise9_3}
    \end{figure}
    \FloatBarrier
\end{exerciseSolution}

\begin{exerciseEnv}{10}
    \par 启动你的内核并运行user/evilhello。环境应该被摧毁并且内核不应该panic。你应该能够看到:
    \inputCodeSetLanguage{bash}
    \begin{lstlisting}[numbers=none]
[00000000] new env 00001000
...
[00001000] user_mem_check assertion failure for va f010000c
[00001000] free env 00001000
    \end{lstlisting}
\end{exerciseEnv}
\begin{exerciseSolution}{10}
    \par 运行evilhello,结果如图\ref{fig:lab3/exercise10_1}所示。环境被摧毁且没有发生kernel panic。
    \begin{figure}[htb]
        \centering
        \includegraphics[width=0.6\linewidth]{lab3/exercise10_1.png}
        \caption{evilhello运行结果}
        \label{fig:lab3/exercise10_1}
    \end{figure}
    \FloatBarrier
\end{exerciseSolution}




% chapter 4
\chapter{Preemptive Multitasking}
\label{cha:preemptive_multitasking}

\section{Multiprocessor Support and Cooperative Multitasking}
\par 这一部分首先要向jOS扩展到一个多处理器的系统上,然后实现一些内核调用来允许用户级别的环境来创建新的环境。还需要实现轮训调度,在进程不使用CPU使允许内核切换到另一个进程。

\subsection{Multiprocessor Support}
\par 我们将使jOS支持对称多处理(SMP)系统。这种系统能够使所有的CPU对内存能和I/O够有相同的访问权限。虽然CPU在对称多处理的情况下以相同的方式工作,但是在启动过程中可以被分为2个类型:引导处理器(BSP)以及应用处理器(AP)。引导处理器用户初始化以及启动操作系统,而应用处理器被引导处理器启动。哪个处理器作为引导处理器由硬件决定。
\par 在对称多处理系统中,每个CPU都有一个本地APIC(LAIPC)单元。LAIPC单元用于传递中断并给它连接的CPU一个唯一的id。在这个lab中将使用LAIPC以下功能:
\begin{itemize}
    \item 从BSP读取LAPIC标识符来区分当前代码在那个CPU上运行。
    \item 从BSP发送STARTUP跨处理器中断到AP来启动其他CPU。
    \item 使用LAPIC的内置计时器编程来触发时钟中断以支持抢占式多任务处理。
\end{itemize}
\par 处理器通过应设在内存上的I/O(MMIO)来访问LAPIC。在MMIO中,物理内存的一部分被链接到I/O设备的寄存器。因此load/store指令可以用于访问设备的寄存器。LAPIC起始于物理地址0xfe000000。对于以前的直接到KERNBASE的映射来说这太高了。jOS在虚拟地址MMIOBASE留了4MB的空间,因此我们由空间来映射这一类的设备。

\exercise{1}{
    \par 实现kern/pmap.c中的mmio\_map\_region。查看kern/lapic.c来查看如何使用它。在对这一exercise进行测试之前需要先完成下一个exercise。
}
\begin{exerciseSolution}{1}
    \par 首先完成mmio\_map\_region。调用boot\_map\_region建立需要的映射。根据注释,需要注意使用的权限为PTE\_PCD|PCE\_PWT来防止CPU来cache这部分内存,此外还需要注意每次都需要更新base,以及超出内存的部分需要发出panic。
    \inputCodeSetLanguage{c}
    \begin{lstlisting}
void * mmio_map_region(physaddr_t pa, size_t size) {
    static uintptr_t base = MMIOBASE;

    if(base + ROUNDUP(size, PGSIZE) > MMIOLIM)
        panic("mmio_map_region: mmio overflow.");
    boot_map_region(kern_pgdir, base, ROUNDUP(size, PGSIZE), pa, PTE_W|PTE_PCD|PTE_PWT);
    uintptr_t temp = base;
    base += ROUNDUP(size, PGSIZE);
    return (void *)temp;
}
    \end{lstlisting}
\end{exerciseSolution}

\subsubsection{Application Processor Bootstrap}
\par 在启动AP之前,BSP理器首先需要收集处理器系统的信息,包括CPU总数,CPU的APIC id,LAPIC的MMIO地址。kern/mpconfig.c中的mp\_init()通过读取BIOS中的MP配置来获得这些信息。APs从实模式开始
\par boot\_aps函数启动了APs的引导。AP的音高过程类似于boot/boot.S中的bootloader启动过程。因此boot\_aps()将AP的入口复制到实模式可以访问的区域MPENTRY\_PADDR。然后boot\_aps()通过发送STARTUP跨处理器中断到LAPIC单元,逐个激活APs。激活时,首先初始化AP的CS:IP让其从入口执行代码,然后开启分页进入保护模式,然后调用mp\_main(),最后等待AP的CPU\_STARTED 信号,并开始激活下一个。

\exercise{2}{
    \par 阅读kern/init.c中的boot\_aps()以及mp\_main,以及kern/mpentry.S中的汇编。然后修改kern/pmap.c中的page\_init()实现,来避免将MPENTRY\_PADDR加入空闲链表,一次确保能够安全的复制AP的启动代码到这个物理地址并运行。代码应该能够通过更新过的check\_page\_free\_list测试。(但可能不会通过check\_kern\_pgdir测试)
}
\begin{exerciseSolution}{2}
    \par 将从MPENTRY\_PADDR开始的一个物理页标记为已使用,并能够将其加入链表。而MPENTRY\_PADDR=0x7000,在低地址处,因此将处理低地址的代码的循环中加入一个判断条件即可,即修改为如下代码:
    \inputCodeSetLanguage{c}
    \begin{lstlisting}
for (i = 1; i < npages_basemem; ++i) {
    if(i == MPENTRY_PADDR/PGSIZE){
        pages[i].pp_ref = 1;
        continue;
    }
    pages[i].pp_ref = 0;
    pages[i].pp_link = page_free_list;
    page_free_list = &pages[i];
}
    \end{lstlisting}
    \par 修改完成后重新编译并运行qemu,其显示输出如图\ref{fig:lab4/exercise2_1}所示。
    \begin{figure}[htb]
        \centering
        \includegraphics[width=0.8\linewidth]{lab4/exercise2_1.png}
        \caption{通过check\_page\_free\_list测试}
        \label{fig:lab4/exercise2_1}
    \end{figure}
\end{exerciseSolution}

\begin{questionEnv}
    \begin{enumerate}
        \item 比较kern/mpentry.S以及boot/boot.S。记住kern/mpentry.S的目标是向内核一样在KERNBASE之上运行。宏MPBOOTPHYS的作用是什么?为什么在kern/mpentry.S中它是必要的但是boot/boot.S中它不是必要的?也就是说,如果kern/mpentry.S中不写MPBOOTPHYS会发生什么?
    \end{enumerate}
\end{questionEnv}
\begin{answer}
    \begin{enumerate}
        \item 根据MPBOOTPHYS的内容可知,起作用是将高地址转换为绝对地址址。因为此时还出于实模式,但是代码中的地址为高地址,因此需要进行转换。
    \end{enumerate}
\end{answer}

\subsubsection{Per-CPU State and Initialization}
\par 在写一个多处理器操作系统时,区分CPU的私有状态以及全局状态非常关键。kern/cpu.h定义了大部分的私有状态。应该注意的私有状态有:
\begin{itemize}
    \item Per-CPU 内核栈。
        \par 多个CPU可能同时陷入内核状态,因此需要给每个处理器一个独立于其他CPU的的内核栈。数组percpu\_kstacks[NCPU][KSTKSIZE]为NCPU个内核栈保留了空间。在Lab2中BSP的内核栈映射到了KSTACKTOP下方,而在lab4中需要把每个CPU的内核栈都映射到这个区域。每个栈之间留下一个空也作为缓冲区避免溢出。
    \item Per-CPU TSS和TSS描述符
        \par 为了标识每个CPU内核栈的位置,需要任务状态段(TSS)
    \item Per-CPU 当前环境指针
        \par 每个CPU能够同时运行各自的用户进程因此重新定义了curenv。
    \item Per-CPU 系统寄存器
        \par 所有的寄存器包括系统寄存器都是CPU私有的,因此初始化寄存器的指令需要在每个CPU执行一次。
\end{itemize}
\exercise{3}{
    \par 修改kern/pmap.c中的mem\_init\_mp()来映射开始于KSTACKTOP的per-CPU内核栈。每个栈的大小是KSTKSIZE加上KSTKGAP字节的不被映射的空间。代码应该能够通过新的check\_kern\_pgdir()测试。
}
\begin{exerciseSolution}{3}
    \par 在这一部分代码中,只需要像lab2一样逐个映射每一个栈的内存地址即可。在这一部分中,虽然BSP的栈地址之前已经映射过,但是更换物理地址不会增删页面引用,因此不修改之前的引用也不会出现问题。
    \inputCodeSetLanguage{c}
    \begin{lstlisting}
static void mem_init_mp(void) {
    uint32_t i;
    uintptr_t start = KSTACKTOP-KSTKSIZE;
    for(i=0; i < NCPU; ++i){
        boot_map_region(kern_pgdir, (uintptr_t)(start), KSTKSIZE, PADDR(percpu_kstacks[i]), PTE_W);
        start -= (KSTKSIZE + KSTKGAP);
    }
}
    \end{lstlisting}
    \par 补完这一部分代码后重新编译运行qemu,结果如图\ref{fig:lab4/exercise3_1}所示,能够成功通过check\_kern\_pgdir()测试。
    \begin{figure}[htb]
        \centering
        \includegraphics[width=0.7\linewidth]{lab4/exercise3_1.png}
        \caption{通过check\_kern\_pgdir测试}
        \label{fig:lab4/exercise3_1}
    \end{figure}
    \FloatBarrier
\end{exerciseSolution}

\exercise{4}{
    \par kern/trap.c中的trap\_init\_percpu()代码为了BSP初始化TSS以及TSS描述符。在Lab3中它是工作正确的,但是在运行其他CPU是运行不正确。修改这部分代码使它能在所有CPU上运行。
}
\begin{exerciseSolution}{4}
    \par 根据注释中的提示我们可以知道thiscpu总是指向当前cpu的信息,而第i个CPU的TSS描述符为$gdt[(GD\_TSS0 >> 3) + i]$,再参考用于初始化BSP的代码,修改后的代码如下:
    \inputCodeSetLanguage{c}
    \begin{lstlisting},
void trap_init_percpu(void) {
    struct Taskstate *thists = &thiscpu->cpu_ts;

    thists->ts_esp0 = KSTACKTOP - thiscpu->cpu_id * (KSTKSIZE + KSTKGAP);
    thists->ts_ss0 = GD_KD;
    thists->ts_iomb = sizeof(struct Taskstate);

    gdt[(GD_TSS0 >> 3) + thiscpu->cpu_id] =
        SEG16(STS_T32A, (uint32_t) (thists), sizeof(struct Taskstate) - 1, 0);
    gdt[(GD_TSS0 >> 3) + thiscpu->cpu_id].sd_s = 0;

    ltr(GD_TSS0 + (thiscpu->cpu_id << 3));
    lidt(&idt_pd);
}
    \end{lstlisting}
    \par 修改完成后按照指导输入make qemu CPUS=4,输出结果如图\ref{fig:lab4/exercise4_1}所示。说明
    \begin{figure}[htb]
        \centering
        \includegraphics[width=0.8\linewidth]{lab4/exercise4_1.png}
        \caption{使用CPUS=4启动qemu界面}
        \label{fig:lab4/exercise4_1}
    \end{figure}
\end{exerciseSolution}

\subsubsection{Locking}
\par 当前代码在初始化AP之后就会开始自旋,在让AP进一步执行之前,首先需要解决多个CPU运行内核的地址竞态。最简单的方法是使用big kernel lock,也就是使用一个全局的锁来锁住整个内核,并在环境返回用户模式时释放这个锁。此时用户环境可以运行在多个CPU上但是内核只能运行在一个CPU上。
\par kern/spinlock.h声明了这个big kernel lock,名称为kernel\_lock。它也提供了lock\_kernel以及unlock\_kernel来加锁和解锁。在以下四个地方应该加锁:
\begin{itemize}
    \item i386\_init()中,BSP唤醒其他CPU之前。
    \item mp\_main()中,在初始化AP之后加锁,并调用sched\_yield来在这个AP上运行用户环境。
    \item 在trap()中,从用户模式进入内核模式需要加锁。检查tf\_cs的较低比特来判断当前是在用户环境中还是在内核中。
    \item 在env\_run()中,在切换到用户模式前解锁。不要太晚或太早解锁,否则可能产生死锁或竞态。
\end{itemize}

\exercise{5}{
    \par 按照上述描述使用big kernel lock。使用lock\_kernel()以及unlock\_kernel()在适当的地方加锁和解锁。
}
\begin{exerciseSolution}{5}
    \par 按照提示在这一节所描述的3个地方(i386\_init, mp\_main以及trap中)加锁即可。注意的是unlock\_kernel()应该置于env\_pop\_tf(\&curenv->env\_tf);之前而不是其他位置。重新编译并运行代码可以发现exercise4中出现的kernel panic消失了,但也由于锁的原因并没有进入user mode。如图\ref{fig:lab4/exercise5_1}所示。
    \begin{figure}[htb]
        \centering
        \includegraphics[width=0.8\linewidth]{lab4/exercise5_1.png}
        \caption{加上big kernel lock后运行qemu结果}
        \label{fig:lab4/exercise5_1}
    \end{figure}
\end{exerciseSolution}

\begin{questionEnv}
    \begin{enumerate}
        \setcounter{enumi}{1}
    \item 看来使用大内核锁能够保证一次只有一个CPU能够运行内核代码。为什么我们还是需要为每个CPU准备一个不同的内核栈呢?使用``即使加上了big kernel lock,共享内核栈依然会出错''来描述原因。
    \end{enumerate}
\end{questionEnv}
\begin{answer}
    \begin{enumerate}
        \setcounter{enumi}{1}
    \item 因为在进入内核之前也可能对内核栈进行了操作,例如trap之前向内核中压入寄存器信息等。
    \end{enumerate}
\end{answer}

\subsection{Round-Robin Scheduling}
\par 这一节的任务是修改内核使之能够进行轮询调度。轮训调度按照如下规则工作:
\begin{itemize}
    \item kern/shed.c中的sched\_yield函数负责选择一个新的环境运行。它按圈顺序搜索envs[]数组(如果之前没有运行的环境则选择数组的开头),选择第一个有ENV\_RUNNABLE状态的环境,并调用env\_run()跳转到那个环境中。
    \item sched\_yield()不能同时在两个CPU上运行同一个环境。如果环境已经在某一个CPU上运行,则其状态会变为ENV\_RUNNING。
    \item 程序中已经实现了一个新的系统调动sys\_yield(),用户进程可以调用它来放弃CPU使用权限。
\end{itemize}

\begin{exerciseEnv}{6}
    \par 在sched\_yield()中实现一个轮询式调度,并修改syscall()来分发sys\_yield()。
    \par 确保在mp\_main()中调用了sched\_yield()。
    \par 修改kern/init.c来创建三个或更多的环境来运行用户程序user/yield.c
    \par 运行make qemu,应该能看到环境在结束前来回切换5次,输出如下。
    \par 使用多个CPU进行测试:make qemu CPUS=2
    \inputCodeSetLanguage{bash}
    \begin{lstlisting}[numbers=none]
...
Hello, I am environment 00001000.
Hello, I am environment 00001001.
Hello, I am environment 00001002.
Back in environment 00001000, iteration 0.
Back in environment 00001001, iteration 0.
Back in environment 00001002, iteration 0.
Back in environment 00001000, iteration 1.
Back in environment 00001001, iteration 1.
Back in environment 00001002, iteration 1.
...
    \end{lstlisting}
    \par 在yield程序退出后,将会没有可运行的环境。调度器应该调用jOS kernel monitor。如果没有发生则表明需要修改代码。
\end{exerciseEnv}
\begin{exerciseSolution}{6}
    \par 首先在kern/sched.c中实现轮询调度,按照题目描述使用按圈轮询的方式,实现的调度代码如下
    \inputCodeSetLanguage{c}
    \begin{lstlisting}
void sched_yield(void) {
    int counter;
    if (curenv) {
        for (counter = ENVX(curenv->env_id) + 1;
                counter != ENVX(curenv->env_id);
                counter = (counter + 1) % NENV)
            if (envs[counter].env_status == ENV_RUNNABLE)
                env_run(envs + counter);
    } else {
        for (counter = 0; counter < NENV; ++counter)
            if (envs[counter].env_status == ENV_RUNNABLE)
                env_run(envs + counter);
    }

    // sched_halt never returns
    sched_halt();
}
    \end{lstlisting}

    \par 然后,在syscall中添加一个case来使用sys\_yield系统调用:
    \begin{lstlisting}
case SYS_yield:
    sys_yield();
    return 0;
    \end{lstlisting}

    \par 最后,修改kern/init.c中的用户进程。在sched\_yield();前添加:
    \begin{lstlisting}
    ENV_CREATE(user_yield, ENV_TYPE_USER);
    ENV_CREATE(user_yield, ENV_TYPE_USER);
    ENV_CREATE(user_yield, ENV_TYPE_USER);
    \end{lstlisting}

    \par 修改完成后,重新编译,运行make qemu CPUS=2,结果如图\ref{fig:lab4/exercise6_1},程序切换5次后退出,与预想的一致。
    \begin{figure}[htb]
        \centering
        \includegraphics[width=0.8\linewidth]{lab4/exercise6_1.png}
        \caption{运行3份user/yield的输出}
        \label{fig:lab4/exercise6_1}
    \end{figure}
\end{exerciseSolution}

\begin{questionEnv}
    \begin{enumerate}
        \setcounter{enumi}{2}
        \item 在的env\_run实现中应该调用了lcr3()。在调用lcr3之前和之后代码进行了对于变量env\_run的参数e的引用。在加载\%cr3寄存器时,MMU使用的地址环境被改变了。但是虚拟地址的意义与给定的地址上下文有关联:地址上下文指定了虚拟地址映射到的物理地址。那么指针为什么在地址切换前后都能够正常解引用?
        \item 任何时候,如果内核从一个环境切换到了另一个,它必须保证老的内核环境寄存器被保存了,以保证她们能够被正确的还原。为什么?这在那里发生的?
    \end{enumerate}
\end{questionEnv}

\begin{answer}
    \begin{enumerate}
        \setcounter{enumi}{2}
        \item 在lab3中env\_setup\_vm以内核的页目录作为模板初始化了用户环境。因此两个环境的e映射到了同一个物理地址,因此才能够正确的解引用。
        \item 保存在kern/trapentry.S中的\_alltraps处,而恢复发生在kern/env.c中的env\_pop\_tf处。只有保证内核调用前后栈的内容没有变化才能使用户程序继续正确执行。
    \end{enumerate}
\end{answer}

\subsection{System Calls for Environment Creation}
\par 虽然内核可以运行并切换多个用户环境了,但是现在只能运行内核设置好的程序。现在需要实现一个新的系统调用,以允许用户创建并开始新的用户环境。
\par Unix提供了fork系统调用来创建进程,它会复制父进程的地址空间来创建子进程。唯一的区别是进程ID。在父进程中,fork返回子进程的id,而在子进程中返回0。父子进程的地址空间是独立的。
\par 现在需要创建一个更原始的jOS系统调用来创建一个用户环境。在这个系统调用中需要实现一个类Unix的fork()。加上一些其他的用户环境创建。新的jOS系统调用如下:
\begin{itemize}
    \item sys\_exofork:
        \par 这个系统调用创建一个空白进程,在其用户空间中不映射物理内存,并且它是不可运行的。在刚开始它会和父进程有相同的寄存器状态。sys\_exofork还会返回子进程的envid\_t,子进程返回0。
    \item sys\_env\_set\_status:
        \par 设置指定进程的状态为ENV\_RUNNABLE或者ENV\_NOT\_RUNNABLE。这个系统调用通常用来标记在地址空间和寄存器状态被初始化之后的新环境就绪。
    \item sys\_page\_alloc:
        \par 分配物理地址空间并将其映射到用户环境空间的虚拟地址。
    \item sys\_page\_map:
        \par 复制一个用户环境的页映射到另一个用户环境,也就是共享内存。
    \item sys\_page\_unmap:
        \par 删除用户环境的虚拟内存映射。
\end{itemize}
\par 对于所有接收环境id的系统调用,jOS内核支持从值0到``当前用户环境''的转换。这个转换被envid2env实现。

\exercise{7}{
    \par 在kern/syscall.c实现上述系统调用。
    \par 实现kern/syscall.c中的上述系统调用。你需要kern/pmap.c以及kern/env.c中的不同函数,尤其是envid2env。现在无论在什么时候调用envid2env,都需要传递一个checkperm参数。使用user/dumbfork来测试内核。
}
\begin{exerciseSolution}{7}
    \par 首先实现sys\_exofork函数,该函数分配一个新的进程但是不做内存复制处理。其关键是如何处理父进程返回子进程id而子进程返回0的问题。于是子进程复制父进程的TrapFrame并将TrapFrame中eax的值设置为0,而函数本身返回子进程的id。这样在运行env\_run时就可以获得不同的返回值了。最后实现的代码如下:
    \inputCodeSetLanguage{c}
    \begin{lstlisting}
static envid_t sys_exofork(void) {
    struct Env *environment;
    int res;
    if((res = env_alloc(&environment, curenv->env_id)) < 0)
        return res;
    environment->env_status = ENV_NOT_RUNNABLE;
    environment->env_tf = curenv->env_tf;
    environment->env_tf.tf_regs.reg_eax = 0;
    return environment->env_id;
}
    \end{lstlisting}

    \par 然后是sys\_page\_alloc函数,该函数在环境的目标地址va分配一个页面并设置权限为perm。在实现这个函数时,首先需要检查目标地址以及权限是否合法。根据注释提示,如果page\_insert失败了还需要释放页面。
    \begin{lstlisting}
static int sys_page_alloc(envid_t envid, void *va, int perm) {
    if(!(perm & PTE_U) || !(perm & PTE_U) ||
            (perm & (~PTE_SYSCALL)) ||
            va > (void *)UTOP ||
            va != ROUNDDOWN(va, PGSIZE))
        return -E_INVAL;
    struct PageInfo *pginfo = page_alloc(ALLOC_ZERO);
    if(!pginfo)
        return -E_NO_MEM;
    struct Env *environment;
    if(envid2env(envid, &environment, 1) < 0)
        return -E_BAD_ENV;
    if(page_insert(environment->env_pgdir, pginfo, va, perm) < 0){
        page_free(pginfo);
        return -E_NO_MEM;
    }
    return 0;
}
    \end{lstlisting}
    \par 接下来是sys\_page\_map函数。简单的来说,这个函数就是建立跨进程的映射。较为重要的是根据注释对于参数逐个进行检查以及建立正确的映射。
    \begin{lstlisting}
static int
sys_page_map(envid_t srcenvid, void *srcva,
         envid_t dstenvid, void *dstva, int perm) {
    if((uint32_t)srcva >= UTOP || PGOFF(srcva) ||
            (uint32_t)dstva >= UTOP || PGOFF(dstva) ||
            !(perm & PTE_U) || !(perm & PTE_U) ||
            (perm & (~PTE_SYSCALL)) )
        return -E_INVAL;
    struct Env *src_environemt, *dst_environment;
    if(envid2env(srcenvid, &src_environemt, 1) < 0 ||
            envid2env(dstenvid, &dst_environment, 1) < 0)
        return -E_BAD_ENV;
    pte_t *pte;
    struct PageInfo *page = page_lookup(src_environemt->env_pgdir, srcva, &pte);
    if(!page || (!(*pte & PTE_W) && (perm & PTE_W)))
        return -E_INVAL;
    if(page_insert(dst_environment->env_pgdir, page, dstva, perm) < 0)
        return -E_NO_MEM;
    return 0;
}
    \end{lstlisting}

    \par 然后是sys\_page\_unmap函数,用于取消sys\_page\_map建立的映射。

    \begin{lstlisting}
sys_page_unmap(envid_t envid, void *va) {
    if((uint32_t)va >= UTOP || PGOFF(va))
        return -E_INVAL;
    struct Env *environment;
    if(envid2env(envid, &environment, 1) < 0)
        return -E_BAD_ENV;
    page_remove(environment->env_pgdir, va);
    return 0;
}
    \end{lstlisting}

    \par 然后实现sys\_env\_set\_status函数,在子进程内存映射结束后设置状态。与上面几个函数相同,需要注意的是对于参数的检查。
    \begin{lstlisting}
static int
sys_env_set_status(envid_t envid, int status) {
    if(status != ENV_RUNNABLE && status != ENV_NOT_RUNNABLE)
        return -E_INVAL;
    struct Env *environment;
    if(envid2env(envid, &environment, 1) < 0)
        return -E_BAD_ENV;
    environment->env_status = status;
    return 0;
}
    \end{lstlisting}

    \par 最后,需要在syscall函数中添加新的系统调用的分发,否则不能正常执行这些系统调用,即在switch中添加如下几行:
    \begin{lstlisting}
case SYS_exofork:
    return sys_exofork();
case SYS_env_set_status:
    return sys_env_set_status(a1, a2);
case SYS_page_alloc:
    return sys_page_alloc(a1, (void *) a2, a3);
case SYS_page_map:
    return sys_page_map(a1, (void *) a2, a3, (void *) a4, a5);
case SYS_page_unmap:
    return sys_page_unmap(a1, (void *) a2);
    \end{lstlisting}

    \par 运行make run-dumbfork,输出如图\ref{fig:lab4/exercise7_1}所示。父进程在10次迭代以后退出,而子进程在20次迭代以后退出。
    \begin{figure}[htb]
        \centering
        \includegraphics[width=0.6\linewidth]{lab4/exercise7_1.png}
        \caption{运行make run-dumbfork结果}
        \label{fig:lab4/exercise7_1}
    \end{figure}
    \FloatBarrier
\end{exerciseSolution}
\par 至此第一部分结束,运行make grade以后第一部分能够正常通过,如图\ref{fig:lab4/exercise7_2}所示。
\begin{figure}[htb]
    \centering
    \includegraphics[width=0.8\linewidth]{lab4/exercise7_2.png}
    \caption{运行make grade的部分结果}
    \label{fig:lab4/exercise7_2}
\end{figure}

\section{Copy-on-Write Fork}
\par Part A中实现的类Unix系统调用较为原始。xv6将父进程的页复制新的进程页来创建子进程。而这个复制的过程则是整个fork最为昂贵的操作。然而调用了fork之后子进程通常会调用exec()将子进程的内存更换为新的程序,在这种情况下复制父进程的内存这个操作就被浪费了。
\par 因此后来的Unix系统让父子进程桐乡同一片物理区域,直到某个进程修改了内存。这个技术叫做copy-on-write。要实现它就让fork()只复制页面的映射关系而补复制内容,同时将页面标记为只读。这样当父进程或子进程写入这个内存值会触发page fault。这时Unix系统才会分配一个新的可写内存给这个进程。这个操作让连续的fork和exec操作开销大大降低,因为在执行exec之前只需要复制一个页面。
\par 在这一部分中将要实现一个``正确''的类Unix的fork,实现copy-on-write。在用户空间实现copy-on-write支持使得内核维持简洁和正确。

\subsection{User-level page fault handling}
\par 用户级的copy-on-write fork需要知道向写保护的页写入时的page fault。通常需要设置一个用户空间,这样page fault就能指示何时需要进行一些操作。 例如,大多数Unix内核最初只映射新进程的堆栈区域中的单个页面,随着进程的堆栈消耗增加,并在尚未映射的堆栈地址上发生页面错误,并按需分配额外的页面。例如一个栈的page fault会分配并映射一个新的页。一个BSS的也错误会分配一个新的页,初始化为0,再映射。
\par 有很多需要内核跟踪的信息。与传统Unix方法不同,在jOS中设计者将决定如何处理用户空间中的页面错误。这种设计方式允许程序定义存储区时有较大的灵活性。

\subsubsection{Setting the Page Fault Handler}
\par 为了处理自己的页面错误用户环境需要在jOS注册一个page fault handler entrypoint。用户环境通过sys\_env\_set\_pgfault\_upcall来注册自己的entrypoint,并在Env结构中增加env\_pgfault\_upcall来记录这一信息。

\exercise{8}{
    \par 实现sys\_env\_set\_pgfault\_upcall系统调用。确保在查找环境id时使用了权限检查,因为这是一个``危险''的系统调用。
}
\begin{exerciseSolution}{8}
    \par 按照说明查找用户环境,并实现系统调用即可。实现的代码如下。最后,需要在syscall中添加用于分发的case。
    \inputCodeSetLanguage{c}
    \begin{lstlisting}
static int
sys_env_set_pgfault_upcall(envid_t envid, void *func) {
    struct Env *environment;
    if(envid2env(envid, &environment, 1))
        return -E_BAD_ENV;
    environment->env_pgfault_upcall = func;
    return 0;
}
    \end{lstlisting}
    \par 在syscall中添加的分发代码如下:
    \begin{lstlisting}
case SYS_env_set_pgfault_upcall:
    return sys_env_set_pgfault_upcall(a1, (void *) a2);
    \end{lstlisting}
\end{exerciseSolution}

\subsubsection{Normal and Exception Stacks in User Environments}
\par 在正常运行时,jOS的进程会运行在正常用户栈上,ESP在开始指向USTACKTOP,栈向下增长,数据存放在USTACKTOP - PGSIZE到USTACKTOP - 1中。当出现页错误时,内核会把进程在一个新的栈上重启,并运行指定的用户级页面错误处理函数,这个栈称为用户异常栈。这个过程完成了进程的栈切换,与从用户态陷入内核态相似。
\par jOS的用户异常栈也是一个页这么大,其顶部为虚拟地址UXSTACKTOP,因此用户异常栈的有效字节在UXSTACKTOP - PGSIZE到UXSTACKTOP - 1之间。当在这个用户异常栈上运行时,用户页面错误处理程序可以使用jOS的普通系统调用来映射页面或调整映射,以解决由页面错误导致的问题。然后用户页面错误处理程序通过汇编语言返回原始栈上的错误代码。

\subsubsection{Invoking the User Page Fault Handler}
\par 现在需要修改kern/trap.c来支持用户级别的页错误处理。如果没有page fault handler,jOS会直接销毁用户环境,否则内核会初始化一个TrapFrame来记录寄存器的状态,在异常栈上处理页面错误。在inc/trap.h中,UTrapframe如下:
\inputCodeSetLanguage{bash}
\begin{lstlisting}
<-- UXSTACKTOP
trap-time esp
trap-time eflags
trap-time eip
trap-time eax       start of struct PushRegs
trap-time ecx
trap-time edx
trap-time ebx
trap-time esp
trap-time ebp
trap-time esi
trap-time edi       end of struct PushRegs
tf_err (error code)
fault_va            <-- %esp when handler is run
\end{lstlisting}
\par 内核会重新安排用户环境并开始执行。fault\_va是导致页面错误的虚拟地址。如果用户环境在发生异常时已经在用户异常栈上运行,则页面错误处理程序应该出错,在这种情况下应该在tf->tf\_esp而不是UXSTACKTOP上构造新管道栈。应该先压入一个32位字,然后才是UTrapframe。
\par 要测试tf->tf\_esp是否在用户异常栈中,需要检查它是否在UXSTACKTOP - PGSIZE到UXSTACKTOP - 1的范围。

\exercise{9}{
    \par 实现kern/trap.c中的page\_fault\_handler来分发页面错误到用户模式的处理函数。确保在写入异常栈时使用了正确的措施。
}
\begin{exerciseSolution}{9}
    \par 在实现用户异常栈调用时,需要考虑这个page fault是否由另一个page fault引起,如果是的话需要递归式上行调用。也就是说,需要将page fault分为两种情况:
    \begin{itemize}
        \item 用户进程发生page fault,此时栈的切换顺序为用户栈\textrightarrow 内核栈\textrightarrow 用户异常栈。
        \item 在page fault时又发生了page fault。此时已经在用户异常栈了,但是还是会重复一遍上面的步骤,此处压入UTrapframe需要留4个字节的空位,而栈的切换顺序为用户异常栈\textrightarrow 内核栈\textrightarrow 用户异常栈。
    \end{itemize}
    \par 此外还需要考虑权限以及地址是否用完等问题,因此在page\_fault\_handler中添加的代码如下:

    \inputCodeSetLanguage{c}
    \begin{lstlisting}
if(curenv->env_pgfault_upcall){
    struct UTrapframe *utf;
    uintptr_t addr;         // addr of utf
    if(USTACKTOP - PGSIZE <= tf->tf_esp && tf->tf_esp < USTACKTOP)
        addr = tf->tf_esp - sizeof(struct UTrapframe) - 4;
    else
        addr = UXSTACKTOP - sizeof(struct UTrapframe);
    user_mem_assert(curenv, (void *)addr, sizeof(struct UTrapframe), PTE_W);
    utf = (struct UTrapframe *)addr;
    utf->utf_fault_va = fault_va;
    utf->utf_err = tf->tf_err;
    utf->utf_regs = tf->tf_regs;
    utf->utf_eip = tf->tf_eip;
    utf->utf_eflags = tf->tf_eflags;
    utf->utf_esp = tf->tf_esp;

    tf->tf_eip = (uint32_t)curenv->env_pgfault_upcall;
    tf->tf_esp = addr;
    env_run(curenv);
}
    \end{lstlisting}
\end{exerciseSolution}

\subsubsection{User-mode Page Fault Entrypoint}
\par 接下来需要实现汇编程序来调用C的页面错误处理程序,并在发生错误的地方恢复执行。这个汇编程序就是用sys\_env\_set\_pgfault\_upcall()向内核注册的处理程序。

\exercise{10}{
    \par 实现lib/pfentry.S中的\_pgfault\_upcall。有趣的是讷河将会直接返回到导致页面错误的用户代码中,而不需要通过内核。同时切换堆栈并重加载EIP是较为困难的。
}
\begin{exerciseSolution}{10}
    \par 将返回的错误地址填到之前预先保留的4个字节处,然后在恢复寄存器,从而达到同时切换esp和eip的效果。实现的代码如下:
    \inputCodeSetLanguage{[x86masm]Assembler}
    \begin{lstlisting}
.text
.globl _pgfault_upcall
_pgfault_upcall:
	// Call the C page fault handler.
	pushl %esp			// function argument: pointer to UTF
	movl _pgfault_handler, %eax
	call *%eax
	addl $4, %esp			// pop function argument

    movl 0x28(%esp), %eax
    subl $0x4, 0x30(%esp)
    movl 0x30(%esp), %edx
    movl %eax, (%edx)
    addl $0x8, %esp

    popal

    addl $0x4, %esp
    popfl

    popl %esp

    ret
    \end{lstlisting}
\end{exerciseSolution}

\exercise{11}{
    \par 完成lib/pgfault.c中的set\_pgfault\_handler()代码。
}
\begin{exerciseSolution}{11}
    \par 这一函数是用户用于指定缺页异常处理方式的函数,需要补充的代码用于初始化\_pgfault\_upcall和\_pgfault\_handler,其中\_pgfault\_upcall就是lib/pgentry.S 中的上行调用入口,\_pgfault\_handler在\_pgfault\_upcall中被调用。补充完整后的代码如下:
    \inputCodeSetLanguage{c}
    \begin{lstlisting}
void set_pgfault_handler(void (*handler)(struct UTrapframe *utf)) {
	int r;
	if (_pgfault_handler == 0) {
        envid_t envid = sys_getenvid();
        if(sys_page_alloc(envid, (void *)(UXSTACKTOP-PGSIZE), PTE_U | PTE_W | PTE_P) < 0)
            panic("set_pgfault_handler: sys_page_alloc failed. ");
        if(sys_env_set_pgfault_upcall(thisenv->env_id, _pgfault_upcall) < 0)
            panic("set_pgfault_handler: sys_env_set_pgfault_upcall failed.");
	}
	_pgfault_handler = handler;
}
    \end{lstlisting}
\end{exerciseSolution}

\subsection{Implementing Copy-on-Write Fork}
\par 现在我们已经有了在用户控件实现copy-on-write fork的条件。在lib/fork.c中提供了fork的框架。像dumbfork一样,fork应该能够创建一个新的环境,然后扫描父进程的整个地址空间并创建对应的子进程空间的调用。不同的是,dumbfork复制了整个页而fork只会创建映射。fork的基本流程如下:
\begin{enumerate}
    \item 父进程将pgfault设置为page fault的处理函数。
    \item 父进程使用sys\_exofork创建一个子进程。
    \item 对于每一个UTOP之下可写的页面以及标记了copy-on-write的页面,父进程使用suppage将其映射到子进程,同时将权限修改为只读。
        \par 异常栈有不同的分配方式。需要在子进程中分配一个新的页面,因为page fault handlerhi向异常栈中写入内容并在异常栈上运行。如果异常栈页面用copy-on-write,就不能执行复制过程了。
    \item 父进程会为子进程设置user page fault entrypoint。
    \item 子进程现在已经就绪,父进程将其标记为runnable。
\end{enumerate}
\begin{exerciseEnv}{12}
    \par 实现lib/fork.c中的fork, duppage以及pgfault。
    \par 使用forktree测试代码,这个程序除了new env, free env和exiting gracefully以外应该产生如下信息(有可能不是以这个顺序产生):
    \inputCodeSetLanguage{bash}
    \begin{lstlisting}[numbers=none]
1000: I am ''
1001: I am '0'
2000: I am '00'
2001: I am '000'
1002: I am '1'
3000: I am '11'
3001: I am '10'
4000: I am '100'
1003: I am '01'
5000: I am '010'
4001: I am '011'
2002: I am '110'
1004: I am '001'
1005: I am '111'
1006: I am '101'
    \end{lstlisting}
\end{exerciseEnv}
\begin{exerciseSolution}{12}
    \par 首先实现fork。参考user/dumbfork.c中的dumbfork,但是需要修改的是设置page fault handler,且fork不需要复制内核映射,此外还需要为子进程设置user page fault entrypoint。实现后的fork如下:
    \inputCodeSetLanguage{c}
    \begin{lstlisting}
envid_t fork(void) {
    // LAB 4: Your code here.
    set_pgfault_handler(pgfault);
    envid_t envid = sys_exofork();
    if (envid < 0)
        panic("fork: sys_exofork failed.");
    if (envid == 0) {
        thisenv = envs + ENVX(sys_getenvid());
        return 0;
    }

    uint32_t addr;
    for (addr = 0; addr < USTACKTOP; addr += PGSIZE)
        if ((uvpd[PDX(addr)] & PTE_P) && (uvpt[PGNUM(addr)] & PTE_P) )
            duppage(envid, PGNUM(addr));

    if (sys_page_alloc(envid, (void *)(UXSTACKTOP - PGSIZE), PTE_U | PTE_W | PTE_P) < 0)
        panic("fork: sys_page_alloc failed.");

    sys_env_set_pgfault_upcall(envid, _pgfault_upcall);

    if (sys_env_set_status(envid, ENV_RUNNABLE) < 0)
        panic("fork: sys_env_set_status failed.");

    return envid;
}
    \end{lstlisting}

    \par 然后实现duppage函数,用于复制父子用户环境之间的页面映射。
    \begin{lstlisting}
static int duppage(envid_t envid, unsigned pn) {
	int r;
    void *addr = (void *)(pn * PGSIZE);
    if((uvpt[pn] & PTE_W) || (uvpt[pn] & PTE_COW)){
        if(sys_page_map(0, addr, envid, addr, PTE_U | PTE_COW | PTE_P) < 0)
            panic("duppage: parent->child sys_page_map failed.");
        if(sys_page_map(0, addr, 0, addr, PTE_U | PTE_COW | PTE_P) < 0)
            panic("duppage: self sys_page_map failed.");
    } else {
        if(sys_page_map(0, addr, envid, addr, PTE_P | PTE_U) < 0)
            panic("duppage: single sys_page_map failed.");
    }
	return 0;
}
    \end{lstlisting}

    \par 最后实现\_pgfault\_upcall中调用的页面错误处理函数。在diayon在调用前父子进程的错误地址都在同一个物理内存,而这一函数的作用是分配一个新的物理页面使得两者独立。最终实现的函数如下:
    \begin{lstlisting}
static void pgfault(struct UTrapframe *utf) {
	void *addr = (void *) utf->utf_fault_va;
	uint32_t err = utf->utf_err;
	int r;

    if(!(err & FEC_WR) || !(uvpt[PGNUM(addr)] & PTE_COW))
        panic("pgfault: invalid UTrapFrame");

    envid_t envid = sys_getenvid();
    addr = ROUNDDOWN(addr, PGSIZE);
    if(sys_page_alloc(envid, PFTEMP, PTE_U | PTE_W | PTE_P) < 0)
        panic("pgfault: sys_page_alloc failed.");
    memcpy(PFTEMP, addr, PGSIZE);
    if(sys_page_map(envid, PFTEMP, envid, addr, PTE_U | PTE_W | PTE_P) < 0)
        panic("pgfault: sys_page_map failed.");
    if(sys_page_unmap(envid, PFTEMP) < 0)
        panic("pgfault: sys_page_unmap failed.");
}
    \end{lstlisting}
\end{exerciseSolution}。

\section{Preemptive Multitasking and Inter-Process communication (IPC)}
\subsection{Clock Interrupts and Preemption}
\par 运行user/spin,这个测试程序运行一个子环境,获得CPU控制权后它会死循环,父环境和内核都不会获得CPU。这显然不利于保护内核免受用户环境的影响。因此需要允许内核抢占正在运行的环境并强行重新获得CPU控制权。此时需要扩展jOS以支持来自时钟的硬件中断。

\subsubsection{Interrupt discipline}
\par 外部中断(IRQ)一共由16种,在picirg.c中将IRQ\_OFFSET映射到了IDT。在inc/trap.h中IRQ\_OFFSET被定义为32,因此IDT[32]包含了时钟中断的入口地址。
\par 在jOS中,相对于xv6做了一个关键的简化:在内核态禁用外部中断。外部中断使用\%eflag寄存器的FL\_IF位控制。当该位为1时开启中断。由于jOS的这一简化,只需要在进入以及离开内核时需要对这个位进行修改。
\par 内核需要确保在用户态时FL\_IF为1,从而使得有中断发生时内核能够处理。bootloader的第一条指令cli关闭了中断,从那之后就再未开启过。

\exercise{13}{
    \par 修改kern/trapentry.S以及kern/trap.c来初始化IDT表项并为IRQ0\textasciitilde 15 提供处理函数。然后修改kern/env.c中的env\_alloc()来保证用户环境总是中断使能的。同样,上述sched\_halt()中的sti指令的注释,使得空闲CPU不屏蔽中断。
}
\begin{exerciseSolution}{13}
    \par 参考lab3中设置trap的方法,在kern/trapentry.S中加入:
    \inputCodeSetLanguage{c}
    \begin{lstlisting}
TRAPHANDLER_NOEC(handler_timer,    IRQ_OFFSET + IRQ_TIMER)
TRAPHANDLER_NOEC(handler_kbd,      IRQ_OFFSET + IRQ_KBD)
TRAPHANDLER_NOEC(handler_serial,   IRQ_OFFSET + IRQ_SERIAL)
TRAPHANDLER_NOEC(handler_spurious, IRQ_OFFSET + IRQ_SPURIOUS)
TRAPHANDLER_NOEC(handler_ide,      IRQ_OFFSET + IRQ_IDE)
TRAPHANDLER_NOEC(handler_error,    IRQ_OFFSET + IRQ_ERROR)
    \end{lstlisting}
    \par 并在kern/trap.c中加入6个函数的声明,然后在trap\_init()中加入如下代码:
    \begin{lstlisting}
SETGATE(idt[IRQ_OFFSET + IRQ_TIMER],    0, GD_KT, handler_timer,    0);
SETGATE(idt[IRQ_OFFSET + IRQ_KBD],      0, GD_KT, handler_kbd,      0);
SETGATE(idt[IRQ_OFFSET + IRQ_SERIAL],   0, GD_KT, handler_serial,   0);
SETGATE(idt[IRQ_OFFSET + IRQ_SPURIOUS], 0, GD_KT, handler_spurious, 0);
SETGATE(idt[IRQ_OFFSET + IRQ_IDE],      0, GD_KT, handler_ide,      0);
SETGATE(idt[IRQ_OFFSET + IRQ_ERROR],    0, GD_KT, handler_error,    0);
    \end{lstlisting}

    \par 最后,在kern/env.c的env\_alloc()中加入e->env\_tf.tf\_eflags |= FL\_IF; 使能用户中断,以使得中断发生时内核能够正确处理,并取消kern/sched.c中sti的注释。
\end{exerciseSolution}

\subsubsection{Handling Clock Interrupts}
\par 在user/spin程序中,子进程开启以后就陷入死循环,此后kernel无法再次获得控制权。我们需要让硬件周期性地产生时钟中断,将控制权交给kernel,使得我们能够切换到其他的进程。

\exercise{14}{
    \par 修改内核的trap\_dispatch函数,让它调用sched\_yield()来在时钟中断发生时查找并运行一个不同的环境。
}
\begin{exerciseSolution}{14}
    \par 要处理时钟中断,只需要在trap\_dispatch中添加如下代码即可。注意需要使用lapic\_eoi()来接受中断。
    \inputCodeSetLanguage{c}
    \begin{lstlisting}
if (tf->tf_trapno == IRQ_OFFSET + IRQ_TIMER) {
    lapic_eoi();
    sched_yield();
}
    \end{lstlisting}
\end{exerciseSolution}

\subsection{Inter-Process communication (IPC)}
\par 前几部分一致在关注操作系统的隔离问题,这造成了每个程序都拥有整个CPU的错觉。操作系统的一个重要的功能是允许服务程序在彼此需要是进行通讯。进程间通讯有很多模型,但在jOS中只会实现一个简单的IPC机制。

\subsubsection{IPC in JOS}
\par 这一部分将实现两个系统调用sys\_ipc\_recv以及sys\_ipc\_try\_send,然后将他们封装为2个函数库,并通过ipc\_recv以及ipc\_send进行通信。IPC的消息由两部分组成:一个32位值以及一个映射。

\subsubsection{Sending and Receiving Messages}
\par 要接收消息,调用函数sys\_ipc\_recv。这一系统调用取消环境的调度,并在接收到消息之前一直阻塞,此时任何其他环境都可以向其发送消息。也就是说Part A中的权限检查不适用于IPC,因为IPC的系统调用是安全的。
\par 要发送消息,需要调用函数sys\_ipc\_try\_send。如果指定的环境正在接收消息,则将发送消息并返回0,否则返回-E\_IPC\_NOT\_RECV来标识目标环境不在接收消息。库函数ipc\_send将重复调用sys\_ipc\_trap\_send直到发送成功。

\subsubsection{Transferring Pages}
\par 当进程调用sys\_ipc\_recv并踢动dstva虚拟地址时,用户环境表明它愿意接收页面映射。如果发送者发送了一个页面,那么页面应该映射到dstva中。如果接受者已经有了一个映射到dstva的页面,那么这一页面就不会映射。
\par 当环境调用sys\_ipc\_try\_send并使用srcva作为参数时,表明发送者想要将映射到srcva的页面以perm的权限发送给接收者。在一个成功的IPC之后,发送方保留原页面的映射,接收方在指定的地址空间获得相同的页面。
\par 如果发送方或接收方指示应该传递页面,则不传递页面。IPC之后内核将接收方的Env中的env\_ipc\_perm设置为接收到的页面的权限,如果没有收到则为0。

\subsubsection{Implementing IPC}
\exercise{15}{
    \par 实现kern/syscall.c中的sys\_ipc\_recv以及sys\_ipc\_try\_send。实现之前先阅读注释,因为它们需要共同工作。当调用envid2时,应该设置checkperm为0,这意味着任何环境都被允许发送IPC消息到任何环境,并且内核除了检查envid是否有效外不做特殊的权限检查。然后实现lib/ipc.c中的ipc\_recv以及ipc\_send。
}
\begin{exerciseSolution}{15}
    \par 首先实现sys\_ipc\_recv。需要注意的是,如果参数指示的虚拟地址大于UTOP时不能报错退出,而需要忽略,因为这只说明接受者只需要接收值而不需要共享页面。实现的代码如下:
    \inputCodeSetLanguage{c}
    \begin{lstlisting}
static int sys_ipc_recv(void *dstva) {
    if((uint32_t)dstva < UTOP && PGOFF(dstva))
        return -E_INVAL;
    curenv->env_ipc_recving = true;
    curenv->env_ipc_dstva = dstva;
    curenv->env_status = ENV_NOT_RUNNABLE;
    sys_yield();
	return 0;
}
    \end{lstlisting}

    \par 接下来实现sys\_ipc\_try\_send,需求与sys\_page\_map类似,但是不能直接调用sys\_page\_map,因为sys\_page\_map会检查env的权限。实现的sys\_ipc\_try\_send如下:
    \begin{lstlisting}
static int sys_ipc_try_send(envid_t envid,
        uint32_t value, void *srcva, unsigned perm) {
    struct Env *env;
    if(envid2env(envid, &env, 0) < 0)
        return -E_BAD_ENV;
    if(!env->env_ipc_recving)
        return -E_IPC_NOT_RECV;

    pte_t *pte;
    int res;
    struct PageInfo *page = page_lookup(curenv->env_pgdir, srcva, &pte);
    if((uint32_t)srcva < UTOP && (
                PGOFF(srcva) || !page ||
                (perm & PTE_W) != (*pte & PTE_W) ||
                !(perm & PTE_U) || !(perm & PTE_P) ||
                (perm & (~PTE_SYSCALL))) )
            return -E_INVAL;
    if((uint32_t)srcva < UTOP && (res = page_insert(env->env_pgdir, page, env->env_ipc_dstva, perm)) < 0)
        return res;

    env->env_ipc_recving = 0;
    env->env_ipc_from = curenv->env_id;
	env->env_ipc_perm = perm;
	env->env_ipc_value = value;
	env->env_status = ENV_RUNNABLE;
    return 0;
}
    \end{lstlisting}

    \par 接下来实现对于上面两个系统调用的封装。首先是ipc\_recv,值得注意的是当不需要共享页面时将虚拟地址设置为大于或等于UTOP的值。ipc\_recv实现的代码如下:
    \begin{lstlisting}
int32_t ipc_recv(envid_t *from_env_store, void *pg, int *perm_store) {
    if(!pg)
        pg = (void *)UTOP;
    int res;
    if((res = sys_ipc_recv(pg)) < 0){
        if(from_env_store)
            *from_env_store = 0;
        if(perm_store)
            *perm_store = 0;
        return res;
    }
    if(from_env_store)
        *from_env_store = thisenv->env_ipc_from;
    if(perm_store)
        *perm_store = thisenv->env_ipc_perm;
    return thisenv->env_ipc_value;
	return 0;
}
    \end{lstlisting}

    \par 然后是ipc\_send,同样,当pg为NULL时设置虚拟地址为大于或等于UTOP的值。ipc\_send实现如下:
    \begin{lstlisting}
void ipc_send(envid_t to_env, uint32_t val, void *pg, int perm) {
    if(!pg)
        pg = (void *)UTOP;
    int res;
    while(1){
        res = sys_ipc_try_send(to_env, val, pg, perm);
        if(!res)
            return;
        if(res != -E_IPC_NOT_RECV)
            panic("ipc_send: not receiving");
        sys_yield();
    }
}
    \end{lstlisting}
	\par 最后,将这两个syscall加入syscall函数分支中:
    \begin{lstlisting}
case SYS_ipc_try_send:
    return sys_ipc_try_send(a1, a2, (void *)a3, a4);
case SYS_ipc_recv:
    return sys_ipc_recv((void *)a1);
    \end{lstlisting}
\end{exerciseSolution}
\par 至此,Lab4全部完成,运行make grade,能够通过全部测试,如图 \ref{fig:lab4/exercise15_1}所示。
\begin{figure}[htb]
    \centering
    \includegraphics[width=0.65\linewidth]{lab4/exercise15_1.png}
    \caption{运行make grade输出结果}
    \label{fig:lab4/exercise15_1}
\end{figure}




% chapter 5
\chapter{基于CUDA的形态学图像处理}
\section{实验目的与要求}
\begin{enumerate}
    \item 深入理解GPGPU的架构并掌握CUDA编程模型
    \item 使用CUDA实现形态图像处理操作的并行算法
    \item 对程序执行结果进行简单的分析和总结
    \item 根据执行结果和硬件环境提出优化解决方案
    \item 将其与Lab2,Lab3和Lab4的结果进行比较
\end{enumerate}

\section{算法描述}
\par 与上述并行算法不同的是,CUDA使用的是异构的并行算法,由于GPU本身就比较适合进行图像处理,因此预计相对于串行算法的加速比会比较高。同样,按照pthread实验中的方法对于图片进行分块,然后将每一个图片块分配给不同的CUDA block,将块中的不同像素分配给不同的CUDA Thread进行处理。如图\ref{fig:cudaScheme}所示,线程的处理过程与算法ErodeAndDilate相同。
\begin{figure}[htpb]
    \centering
    \includegraphics[width=0.4\linewidth]{cudaScheme.png}
    \caption{CUDA块与线程分配}
    \label{fig:cudaScheme}
\end{figure}

\section{实验方案}
\par 编写CUDA程序并进行测试,在编译时需要使用CUDA的包装编译器nvcc进行编译,,在运行时需要保证Nvidia驱动已被正确加载。

\section{实验结果与分析}
\par 图像处理的结果与串行图像处理的结果相同,性能方面,程序在使用分块大小为128,每个像素使用一个线程处理的情况下,处理时间如图\ref{fig:cudaOutput}所示,可以看出,程序只使用了989ms就完成了1000次处理,相对与并行程序加速比达到了44.7,由此可见CUDA是较为适合并行处理图片的。

\begin{figure}[htpb]
    \centering
    \includegraphics[width=0.9\linewidth]{cudaOutput.png}
    \caption{CUDA程序运行结果}
    \label{fig:cudaOutput}
\end{figure}
\par 由于CUDA架构与仅使用CPU并行的架构完全因此不同,因此不具备与CPU程序在性能上的可比性。在不同的块大小情况下,CUDA程序的最优时间为686ms,相对于串行程序的加速比达到了64.31,由此可见,对于数据易于分块,数据间无依赖的简单数据使用GPU进行处理占有绝对优势。


\chapter{基于并行回溯算法的数独求解算法}
\section{实验目的与要求}
\begin{enumerate}
    \item 掌握并行化和改进程序的方法
    \item 了解并行粒度与性能之间的关系
    \item 掌握如何分割数据和分解复杂算法的任务
\end{enumerate}

\section{算法描述}
\par 数独的规则见\url{https://zh.wikipedia.org/zh-sg/%E6%95%B8%E7%8D%A8}。一个未求解和已经进行求解的数独如图\ref{fig:unsolved}和\ref{fig:solved}所示。
\begin{figure}[htpb]
    \centering
    \begin{minipage}{0.45\linewidth}
        \includegraphics[width=0.9\linewidth]{unsolved.png}
        \caption{未求解的数独}
        \label{fig:solved}
    \end{minipage}
    \hfill
    \begin{minipage}{0.45\linewidth}
        \includegraphics[width=0.9\linewidth]{solved.png}
        \caption{已经求解的数独}
        \label{fig:unsolved}
    \end{minipage}
\end{figure}
\par 首先,需要一个串行的数独求解程序进行参考,串行的程序使用回溯的方法对于数独进行求解,同样,为了便于时间的统计,因此需要增加求解的数据量,此时,对于100个数独进行求解。
\par 然后需要设计一个并行的数独求解程序。由于数独前后的逻辑依赖性较强,因此不便于进行并行。在本实验中采用的数独求解方法为并行化的测试一个cell中的所有case,以求获得最小的可能的case的时间。此时,case的选择可以在不同的遍历层级进行。递归的层级越高则粒度越低。

\section{实验方案}
\par 按照算法描述中的算法编写OpenMP程序,在小于特定层级时将所有的可能性如队列,然后在这一特定的层级使用OpenMP进行并行,在大于这一层级不再使用并行算法,以避免线程数量指数爆炸。
\par 需要注意的是在一个线程找到结果后需要及时停止其他的线程,而在停止时不可以使用C++ exception,经过性能测试,exception带来的开销约为20\%,极大的开销会为程序带来负优化。线程的错误处理一定因此通过返回值的方式进行。

\section{实验结果与分析}
\par 串行程序解决100个数独的结果如图\ref{fig:sudoSerial}所示。数独题目由qqwing程序生成,所有答案经过验证均正确。从图中可以看出,串行的程序需要使用25.8s进行求解。
\begin{figure}[htpb]
    \centering
    \includegraphics[width=0.9\linewidth]{sudoSerial.png}
    \caption{串行数独求解}
    \label{fig:sudoSerial}
\end{figure}
\par 在递归的第二层进行4线程并行的时间如图\ref{fig:sudoParallel}所示,从图中可以看出,需要2.96s进行求解,相比于串行程序,加速比为8.71,达到了较为理想的加速比。

\begin{figure}[htbp]
    \centering
    \includegraphics[width=0.9\linewidth]{sudoParallel.png}
    \caption{4线程并行数独求解}
    \label{fig:sudoParallel}
\end{figure}
\par 加速比与粒度的关系如图\ref{fig:sudoTrend}所示。可以看出,粒度越细加速比越低,这是由于前边的并行部分增大以及线程竞争引起的。

\begin{figure}[htbp]
    \centering
    \includegraphics[width=0.8\linewidth]{sudoTrend.png}
    \caption{加速比随粒度的变化关系}
    \label{fig:sudoTrend}
\end{figure}

\appendix
\chapter{开发与运行环境}
\label{cha:env}

\par 所有lab的开发与运行的环境的平台、版本如表\ref{tab:env}所示。
\begin{table}[htpb]
    \centering
    \caption{开发与运行环境}
    \label{tab:env}
    \begin{tabular}{ c r l }
        \toprule
        \multicolumn{1}{c}{} &
        \multicolumn{1}{c}{项目} &
        \multicolumn{1}{c}{版本} \\
        \cmidrule(lr){1-1} \cmidrule(lr){2-2} \cmidrule(lr){3-3}
        \multirow{8}{*}{软件} & 操作系统    & ArchLinux,  Kernel version 4.17.5-1\\
                              & gcc             & 8.1.1     \\
                              & GNU Make        & 4.2.1     \\
                              & OpenMP          & 6.0.1     \\
                              & OpenMPI         & 3.1.0-1   \\
                              & CUDA            & 9.2.148-1 \\
                              & Nvidia driver   & 396.24-15 \\
                              & shell           & fish 2.7.1\\
        \cmidrule(lr){1-1} \cmidrule(lr){2-2} \cmidrule(lr){3-3}
        \multirow{3}{*}{硬件} & CPU             & Intel Core i7-7700K CPU @ 4.20GHz \\
                              & GPU             & NVIDIA Corporation GP102 [GeForce GTX 1080 Ti] \\
                              & RAM             & Kingston DDR4 16GiB (\(\times 2\))\\
        \bottomrule
    \end{tabular}
\end{table}

\par 目录结构如下
\inputCodeSetLanguage{bash}
\begin{lstlisting}
./                          project根目录
├── README.md               README文件(本文件)
├── Makefile                主Makefile,用于同时编译/清除所有lab
├── data                    数据目录
│   ├── Lenna.png           lab1 ~ lab5 所用图片
│   └── questions.txt       lab6 所用数独数据
├── lab1                    串行形态学图像处理
│   ├── Makefile            lab1 Makefile
│   └── main.cpp            lab1 源码
├── lab2                    使用pthread加速的多线程图像处理
│   ├── Makefile            lab2 Makefile
│   └── main.cpp            lab2 源码
├── lab3                    使用openmp加速的多线程图像处理
│   ├── Makefile            lab3 Makefile
│   └── main.cpp            lab3 源码
├── lab4                    使用openmpi加速的多线程图像处理
│   ├── Makefile            lab4 Makefile
│   ├── main.cpp            lab4 源码
│   └── run.sh              openmpi 运行环境的包装,使用mpirun执行编译输出
├── lab5                    使用cuda加速的多线程图像处理
│   ├── Makefile            lab5 Makefile
│   └── main.cu             lab5 源码
├── project1                数独求解程序
│   ├── Makefile            lab6 Makefile
│   ├── parallel.cpp        使用openmp加速的并行数独求解程序源码
│   └── serial.cpp          串行数独求解程序源码
├── lib                     第三方库文件夹
│   └── CImg-2.3.3          CImg图像处理库
│       └── ...             CImg图像处理库内容
└── report                  报告文件夹
        └── ...             报告内容
\end{lstlisting}

\par 要正常的编译和运行,需要保证gcc、nvcc所在的目录在\$PATH中。

\par 要能够正确的编译并运行lab5,需要满足下列要求:
\begin{itemize}
    \item nvcc在\$PATH中
    \item PC上装有支持cuda的Nvidia显卡
    \item 正确版本的Nvidia驱动以正确安装并 **已加载**
    \item Nvidia显卡有足够的显存以及足够的Concurrent Thread数目以支持程序的运行。
\end{itemize}

\par 此外,lab2、lab3可以使用`-n <number>`来指定线程数,使用`-s <number>` 来指定图片分块大小 (推荐在8~256之间) ;lab4可以通过更改`run.sh`更改进程数,进程数必须大于1,但不可超过CPU物理内核数。


\vfill
{\tiny written by HuSixu \hfill powered by \XeLaTeX .}

\end{document}
