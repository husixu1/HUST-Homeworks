%! TEX program = xelatex
\documentclass{report}
% provides basic settings for ctex document
\usepackage[UTF8, heading=true]{ctex}
\usepackage{fancyhdr}
\usepackage{tocloft}
\usepackage[margin=1in]{geometry}
\usepackage{metalogo}                   % \XeLaTeX
\usepackage{float}                      % figure H flag
\usepackage{microtype}                  % break long words
\usepackage[hidelinks]{hyperref}
\usepackage{tabularx}
\usepackage{amsmath}
\usepackage{lmodern}                    % allow fonts to scale
\usepackage{placeins}
\usepackage{multirow}                   % multirow, multicolumn support
\usepackage{booktabs}                   % toprule, cmidrule support
\usepackage{caption}

% make chapter stay in the same page
%\makeatletter
%\renewcommand\chapter{\thispagestyle{plain}%
%\global\@topnum\z@
%\@afterindentfalse
%\secdef\@chapter\@schapter}
%\makeatother

\fancyhead{}
\renewcommand{\sectionmark}[1]{\markleft{#1}}
\renewcommand{\partmark}[1]{\markright{#1}}
\lhead{\tiny \leftmark}
\rhead{\tiny \rightmark}
% see $(texdoc ctex) for details
\ctexset{
    chapter = {
        name = {实验},
        format += \flushleft,
        number = \arabic{chapter},
    },
    section = {
        format += \flushleft,
    },
    appendix = {
        number = \Alph{chapter},
        name = {附录},
    },
}

\pagestyle{fancy}
%\setlength\cftaftertoctitleskip{2em}

% provides code input support
\usepackage{xparse}                     % newcommand multiple optional arguments
\usepackage{listings}                   % code
\usepackage{fontspec}
\usepackage{lmodern}

%\newfontfamily\codeF{Fira Code}

\setmonofont[
    Contextuals={Alternate},
    ItalicFont = Fira Code      % to avoid font warning
]{Fira Code}

% usage: \inputCode{[language] <path>}
% if language is not explicitly set, it's defaulted to c
\DeclareDocumentCommand{\inputCode}{ O{c} m }{
    {
        \lstinputlisting[
            basicstyle=\small\ttfamily,
            language={#1},
            tabsize=4,
            showstringspaces=false,
            breaklines=true,
            frame=shadowbox,
            framexleftmargin=10mm,
            rulesepcolor=\color{black},
            numbers=left,
            xleftmargin=4em,
        ]{#2}
    }
}

\DeclareDocumentCommand{\inputCodeSetLanguage}{ m }{
    \lstset{
        basicstyle=\small\ttfamily,
        language={#1},
        tabsize=4,
        showstringspaces=false,
        breaklines=true,
        frame=shadowbox,
        framexleftmargin=10mm,
        rulesepcolor=\color{black},
        numbers=left,
        xleftmargin=4em,
    }
}

\DeclareDocumentCommand{\inputCodeNoNumberSetLanguage}{ m }{
    \lstset{
        basicstyle=\small\ttfamily,
        language={#1},
        tabsize=4,
        showstringspaces=false,
        breaklines=true,
        frame=shadowbox,
        rulesepcolor=\color{black},
    }
}


\usepackage{xparse}
\usepackage{tcolorbox}
\usepackage{mdframed}

\tcbuselibrary{breakable}
\NewDocumentCommand{\exercise}{ m +m }{
    {
        \edef\originalParIndent{\the\parindent}
        \begin{tcolorbox}[breakable,arc=0mm,boxrule=0.4pt]
            \setlength{\parindent}{\originalParIndent}
            \noindent
            \textbf{\large Exercise #1}
            \indent
            #2
        \end{tcolorbox}
    }
}

% environment style of exercise, to feed special requirements
\NewDocumentEnvironment{exerciseEnv}{m}{
    \edef\originalParIndent{\the\parindent}
    \begin{tcolorbox}[breakable,arc=0mm,boxrule=0.4pt]
        \setlength{\parindent}{\originalParIndent}
        \noindent
        \textbf{\large Exercise #1}
        \indent
} {
    \end{tcolorbox}
}

\NewDocumentEnvironment{questionEnv}{}{
    \edef\originalParIndent{\the\parindent}
    \begin{tcolorbox}[breakable,arc=0mm,boxrule=0.4pt]
        \setlength{\parindent}{\originalParIndent}
        \noindent
        \textbf{\large Question}
        \indent
} {
    \end{tcolorbox}
}

% a raised rule
\NewDocumentCommand{\raisedrule}{ O{0em} m }{\leaders\hbox{\rule[#1]{1pt}{#2}}\hfill}

\NewDocumentEnvironment{exerciseSolution}{m}{
    {\noindent \textbf{\large Exercise #1 实验过程 \raisedrule[0.3em]{0.6pt}}}
} {
    \par
    {\noindent \textbf{\large \raisedrule[0.3em]{0.6pt} Exercise #1 实验过程}}
    \vspace{1em}
}

\NewDocumentEnvironment{answer}{}{
    {\noindent \textbf{\large Answer \raisedrule[0.3em]{0.6pt}}}
} {
    \par
    {\noindent \textbf{\large \raisedrule[0.3em]{0.6pt} Answer}}
    \vspace{1em}
}



%%%%%%%%%%%%%%%%%%%%%%%%%%%%%%%%
\graphicspath{{./res/}}

% report content %%%%%%%%%%%%
%%%%%%%%%%%%%%%%%%%%%%%%%%%%%
\begin{document}

% cover page
\begin{titlepage}
    \addtolength{\topmargin}{1cm}
    \centering
    \includegraphics[width=0.6\textwidth]{hust.jpg}\par
    \vspace{0.5cm}
    {\Huge \heiti 编译原理实验报告}\par
    \vspace{10cm}
    {
        \large
        \begin{tabular}{r m{8em}}
            \makebox[6em][s]{学生姓名}:& 胡思勖 \\ \cline{2-2}
            \makebox[6em][s]{学号}:& U201514898\\ \cline{2-2}
            \makebox[6em][s]{专业}:& 计算机科学与技术\\ \cline{2-2}
            \makebox[6em][s]{班级}:& 计卓1501\\ \cline{2-2}
            \makebox[6em][s]{指导教师}:& 徐丽萍\\ \cline{2-2}
        \end{tabular}
    }
    \vfill
    2018-06-23
\end{titlepage}

\setcounter{tocdepth}{1}
\pagenumbering{Roman}
\tableofcontents

\newpage
\pagenumbering{arabic}
\setcounter{page}{1}



% chapter 1
\chapter{Building a Cache Simulator}
\label{cha:building_a_cache_simulator}
\par 整个实验分为两个部分,在第一部分中,需要实现一个缓存模拟器。通过valgrind中的lackey工具\footnote{\url{http://valgrind.org/docs/manual/lk-manual.html}}可以得到一个程序几乎所有的内存访问情况,使用如下命令即可获得这些信息:
\inputCodeNoNumberSetLanguage{bash}
\begin{lstlisting}[numbers=none]
linux> valgrind --log-fd=1 --tool=lackey --trace-mem=yes <program>
\end{lstlisting}

\par 一个样例输出如下:
\begin{lstlisting}[numbers=none]
I 0400d7d4,8
 M 0421c7f0,4
 L 04f6b868,8
 S 7ff0005c8,8
\end{lstlisting}

\par 输出的格式为``操作\ 地址,大小'',I表示指令加载,L表示数据加载,S表示数据存储,M表示数据修改,而数据修改应该被当做一次数据加载加上一次数据存储。内存地址以十六进制的形式给出。

\section{实验要求}
\label{sec:shi_yan_yao_qiu_}

\par 此实验要求编写一个Cache模拟器,其输入为Valgrind输出的内存访问轨迹,输出为与csim-ref相同的统计数据。

\par 实验要求:
\begin{itemize}
    \item 模拟器必须在输入参数s、E、b设置为任意值时均能正确工作——即需要使用malloc函数(而不是代码中固定大小的值)来为模拟器中数据结构分配存储空间。
    \item 由于实验仅关心数据Cache的性能,因此模拟器应忽略所有指令cache访问(即轨迹中“I”起始的行)
    \item 假设内存访问的地址总是正确对齐的,即一次内存访问从不跨越块的边界——因此可忽略访问轨迹中给出的访问请求大小
    \item main函数最后必须调用printSummary函数输出结果,并如下传之以命中hit、缺失miss和淘汰/驱逐eviction的总数作为参数:
\end{itemize}

\par 在编写完成后,使用test-csim程序进行测试以及评分。Cache模拟器使用的策略应为LRU替换策略。

\section{实验设计}
\label{sec:shi_yan_she_ji_}

\subsection{总体设计}
\label{sub:zong_ti_she_ji_}

\par 需要修改的文件为csim.c。由于已经给出了框架,首先观察框架中的代码,其中包含以下函数:
\inputCodeSetLanguage{c}
\begin{lstlisting}
accessData(mem_addr_t addr)
freeCache()
initCache()
main(int argc,char * argv[])
printUsage(char * argv[])
replayTrace(char * trace_fn)
\end{lstlisting}

\par 根据函数名和注释可以得知,initCache和freeCache用于初始化和释放cache,printUsage用于打印帮助信息,而replayTrace则直接被主函数调用并调用accessData来模拟Cache的替换过程。而需要修改的函数就是accessData函数。

\par 接下来观察Cache是如何在C语言中被组织并模拟的。首先,文件中定义了如下的结构,来描述cache line,并定义其指针以及指针的指针分别为cache set和cache。
\begin{lstlisting}
typedef struct cache_line {
    char valid;
    mem_addr_t tag;
    unsigned long long int lru;
}
typedef cache_line_t *cache_set_t;
typedef cache_set_t *cache_t;
\end{lstlisting}

\par 而通过initCache中的初始化过程可以得知,文件中描述了一个如图\ref{fig:cacheStructure}所示的cache:

\begin{figure}[htb]
    \centering
    \includegraphics[width=0.72\linewidth]{cacheStructure.png}
    \caption{Cache 结构}
    \label{fig:cacheStructure}
\end{figure}
\FloatBarrier

\par 了解模拟cache在内存中的结构后,每一次的内存访问过程如下:首先判断此内存地址是否能够在对应位置cache hit,如果cache hit则将hit计数加一,否则判定为cache miss,将miss计数加一,然后判断是否需要替换。如果需要替换则按照LRU规则进行替换,然后将eviction计数加一。由于cache line是以数组的形式存在于内存中的,因此在实现LRU的队列结构的时候需要注意对于结构体中lru字段的操作。

\subsection{详细设计}
\label{sub:xiang_xi_she_ji_}
\par 接下来根据LRU的替换规则设计accessData中具体的cache替换流程。首先,根据传入的地址计算组索引以及tag,并根据组索引获得cache组。接下来对于组中的所有cache line进行遍历,若某一个cache line的tag字段与事先计算的tag字段相等,则增加一次cache hit计数,然后更新此cache set中所有cache line的lru字段。若遍历完成都没有cache hit,则一定出现了cache miss,此时将cache miss计数加一。但此时依然需要分类讨论:若此cache set未被填满,则不需要进行替换,直接将新的记录插入cache set中并更新所有cache line的lru字段,否则还需要进行cache替换。替换方法为:首先遍历次cache set中所有的cache line,找出其中lru值最大的cache line(存在时间最长的),将其替换为新的地址后更新其他所有cache line的lru字段。具体的替换流程如图\ref{fig:cacheSub}所示。

\begin{figure}[htb]
    \centering
    \includegraphics[width=0.7\linewidth]{cacheSub.png}
    \caption{cache更新流程}
    \label{fig:cacheSub}
\end{figure}

\par 在上述三种不同的情况中,三种lru的更新方法是不同的:在cache hit时,遍历所有cache line,将lru字段小于命中的cache的lru字段值,并且valid字段为1的所有cache line的lru值加一,然后将命中的cache line的lru值置为0。
\par 在cache miss但未发生cache eviction时,首先遍历所有cache line,找出第一个valid字为0的cache line,将这一cache line的valid字段置为1,lru字段置为0并将其他所有valid字段为1的cache line的lru值加一。
\par 若出现cache miss eviction,则遍历所有cache line,找出lru值最大的cache line,将其lru值置为0并将这一cache set中其他所有cache line的lru值加一。则这三种替换过程如\ref{fig:cacheSub123}所示。

\begin{figure}[htb]
    \centering
    \includegraphics[width=0.9\linewidth]{cacheSub123.png}
    \caption{三种情况下的cache替换过程}
    \label{fig:cacheSub123}
\end{figure}

\par 除了上述对于核心部分的设计之外,还需要考虑其他的部分:题目要求实现与csim-ref完全相同的功能,而csim-ref的参数可以指定-h、-v、-s、-E、-b、-t。其中-h在所给出的代码框架中已经得以实现,而-s、-E、-b、-t全部是为核心部分的cache模拟提供支持的。-v参数则需要添加额外的实现。使用-v参数得到的一个样例输出如下:
\inputCodeNoNumberSetLanguage{bash}
\begin{lstlisting}
L 10,1 miss
M 20,1 miss hit
L 22,1 hit
S 18,1 hit
L 110,1 miss eviction
L 210,1 miss eviction
M 12,1 miss eviction hit
hits:4 misses:5 evictions:3
\end{lstlisting}

\par 从输出中可以看出,-v对于除了I型指令之外的内存操作均进行了输出,输出格式与valgrind的输出格式类似,但在每行后面添加了cache hit/miss/eviction的情况。

\section{实验过程}
\label{sec:shi_yan_guo_cheng_}

\subsection{实验环境}
\label{sub:shi_yan_huan_jing_}
\begin{table}[htb]
    \centering
    \caption{实验环境配置}
    \label{tab:label}
    \begin{tabular}{r l}
        \toprule
        操作系统        & Archlinux x64 2018-04-11 更新\\
        编译器          & gcc 7.3.1 \\
        Makefile管理器  & gnu make 4.2.1 \\
        内存调试工具    & valgrind 3.13.0 \\
        版本管理工具    & git 2.17.0 \\
        \bottomrule
    \end{tabular}
\end{table}

\subsection{详细步骤}
\label{sub:shi_yan_guo_cheng_}

\par 有了上述设计后,代码的实现就较为容易了。首先,补全freeCache程序中的代码。根据initCache中的代码可知源程序是在cache这一二维数组的两个维度进行了分配,而从逻辑上可以推断出,在整个程序的运行过程中不需要对于内存进行新的分配,因此,在freeCache时只需要对应的将initCache中分配的内存逐个释放即可。实现的代码如下:
\inputCodeSetLanguage{c}
\begin{lstlisting}
void freeCache() {
    int i;
    for (i = 0; i < S; ++i)
        free(cache[i]);
    free(cache);
}
\end{lstlisting}

\par 接下来实现核心部分代码,也就是accessData模拟cache更新这一部分。首先通过如下代码计算组编号以及tag值,并根据组编号获得cache组。
\begin{lstlisting}
mem_addr_t set_index = (addr >> b) & set_index_mask;
mem_addr_t tag = addr >> (s + b);
cache_set_t cache_set = cache[set_index];
\end{lstlisting}

\par 然后对于cache set进行第一次遍历。在每一次循环中,对于是否命中进行判断,判断的依据为valid是否为1且tag是否与当前内存地址的tag相等。若命中则增加命中计数并进行上述替换过程1,然后直接在返回。此外,还需要注意对于-v参数的支持:如果verbosity flag为1,则需要输出``hit''提示。
\begin{lstlisting}
for (i = 0; i < E; ++i) {
        /* hit */
        if (cache_set[i].valid && cache_set[i].tag == tag) {
            if (verbosity)
                printf("hit ");
            ++hit_count;

            /* update entry whose lru is less than the current lru (newer) */
            for (int j = 0; j < E; ++j)
                if (cache_set[j].valid && cache_set[j].lru < cache_set[i].lru)
                    ++cache_set[j].lru;
            cache_set[i].lru = 0;
            return;
        }
    }
}
\end{lstlisting}

\par 如果上述循环执行完成而没有返回,表明没有发生cache hit,因此使用以下代码将miss技术加一,并在verbosity flag为1时打印``miss''提示。
\begin{lstlisting}
if (verbosity)
    printf("miss ");
++miss_count;
\end{lstlisting}

\par 接下来需要分情况讨论是否会发生cache eviction。经过考虑后发现,可以将不发生eviction情况下寻找最大lru条目的过程与发生eviction情况下寻找第一个invalid cache line的过程合并在同一个循环中,以提高cache模拟器的性能,合并后的循环如下:
\begin{lstlisting}
int j, maxIndex = 0;
unsigned long long maxLru = 0;
for (j = 0; j < E && cache_set[j].valid; ++j) {
    if (cache_set[j].lru >= maxLru) {
        maxLru = cache_set[j].lru;
            maxIndex = j;
        }
    }
}
\end{lstlisting}

\par 在上述循环结束后,通过$j$的值来判断是否发生了cache eviction。由于上述循环在遇到invalid cache line时会跳出,因此当$j$小于E时表明未发生cache eviction。也就是说,如果发生cache eviction,则$j==E$一定成立。因此通过以下代码进行cache的更新。
\begin{lstlisting}
if (j != E) {
    for (int k = 0; k < E; ++k)
        if (cache_set[k].valid)
            ++cache_set[k].lru;
    cache_set[j].lru = 0;
    cache_set[j].valid = 1;
    cache_set[j].tag = tag;
} else {
    if (verbosity)
        printf("eviction ");
    ++eviction_count;
    for (int k = 0; k < E; ++k)
        ++cache_set[k].lru;
    cache_set[maxIndex].lru = 0;
    cache_set[maxIndex].tag = tag;
}
\end{lstlisting}

\par 在上述代码中,$j\neq E$部分为不发生eviction的情况,$j==E$的部分为发生eviction的情况,分别按照章节\ref{sub:xiang_xi_she_ji_}中的方法进行cache的更新。

\par 在accessData中核心部分的代码完成后,需要完成replayTrace中的代码调用accessData完成对于内存访问过程中cache变化的模拟。使用fscanf读入每一行,然后根据访问类型进行对于accessData的调用:忽略I型访问,对于L型和S型访问调用accessData一次,而对于M型访问调用accessData两次,此外还需要注意verbosity flag对于输出的影响。代码如下:
\begin{lstlisting}
while (fscanf(trace_fp, " %c %llx,%d", buf, &addr, &len) > 0) {
    if (verbosity && buf[0] != 'I')
        printf("%c %llx,%d ", buf[0], addr, len);
    switch (buf[0]) {
        case 'I':
            break;
        case 'L':
        case 'S':
            accessData(addr);
            break;
        case 'M':
            accessData(addr);
            accessData(addr);
            break;
        default:
            break;
    }
    if (verbosity && buf[0] != 'I')
        putchar('\n');
}
\end{lstlisting}

\par 至此,所有需要填写的代码已补充完成。

\subsection{测试与分析}
\label{sub:jie_guo_fen_xi_}

\par 对于完成的代码进行测试:首先使用make对代码进行编译,然后直接运行./test-csim命令。实验的测试程序给出的测试样例如表\ref{tab:example}所示,其输出结果如图\ref{fig:result1}所示。

\begin{center}
    \captionof{table}{测试样例}
    \label{tab:example}
    \begin{longtable}{r c c c c c c}
        \toprule
        \multicolumn{1}{c}{\textbf{测试文件}} &
        \multicolumn{1}{c}{\textbf{组索引位数 s}} &
        \multicolumn{1}{c}{\textbf{关联度 E}} &
        \multicolumn{1}{c}{\textbf{块偏移位数 b}} &
        \multicolumn{1}{c}{\textbf{hit}} &
        \multicolumn{1}{c}{\textbf{miss}} &
        \multicolumn{1}{c}{\textbf{eviction}}            \\
        \cmidrule(lr){1-1} \cmidrule(lr){2-4} \cmidrule(lr){5-7}
        yi2.trace   & 1 & 1 & 1 & 9      & 8     & 6     \\
        yi.trace    & 4 & 2 & 4 & 4      & 5     & 2     \\
        dave.trace  & 2 & 1 & 4 & 2      & 3     & 1     \\
        trans.trace & 2 & 1 & 3 & 167    & 71    & 67    \\
        trans.trace & 2 & 2 & 3 & 201    & 37    & 29    \\
        trans.trace & 2 & 4 & 3 & 212    & 26    & 10    \\
        trans.trace & 5 & 1 & 5 & 213    & 7     & 0     \\
        long.trace  & 5 & 1 & 5 & 265189 & 21775 & 21743 \\
        \bottomrule
    \end{longtable}
\end{center}

\begin{figure}[htb]
    \centering
    \includegraphics[width=0.8\linewidth]{result1.png}
    \caption{test-csim输出结果}
    \label{fig:result1}
\end{figure}

\par 从图中可以看出,所有的测试通过。接下来对于-v选项进行测试。首先对于yi.trace的进行测试,然后对比处理其他trace文件的输出与csim-ref处理相同文件的输出。运行的结果如\ref{fig:result2}所示。

\begin{figure}[htb]
    \centering
    \includegraphics[width=0.95\linewidth]{result2.png}
    \caption{对于-v选项的测试}
    \label{fig:result2}
\end{figure}

\par 可以看出,csim的输出与csim-ref的输出完全相同。至此,所有的测试完成,csim能够完全复现csim-ref的功能。




% chapater 2
\chapter{基于pthread的形态学图像处理}
\section{实验目的与要求}
\begin{itemize}
    \item 掌握使用pthread的基本的并行编程设计方法以及调优方法;
    \item 掌握并行编程中基本的数据分块以及任务分解的方法。
    \item 使用pthread实现并行的形态学图像处理。
    \item 简要分析以及总结处理的结果。
\end{itemize}

\section{算法描述}
\par 使用多个线程对于一个图像进行蚀刻以及膨胀的算法如下,算法为一个线程的流程,而有多个这样的线程同时进行。
\begin{simpleAlgorithm}{pthread并行处理算法(一个线程)}
    \Procedure{PthreadParallel}{$blocks$}
    \While{true}
        \State lock\((blocks)\)
        \State get first block \(blk\) from \(blocks\)
        \If{\(blocks\).empty()}
            \State unlock\((blocks)\)
            \State \Return
        \EndIf
        \State unlock\((blocks)\)
        \State \Call{ErodeAndDilate}{$blk, kernel_e, kernel_d$}
    \EndWhile
    \EndProcedure
\end{simpleAlgorithm}
\par 算法中,\(blocks\)参数为一个工作队列,队列中的工作为原预处理过后的图片的子图片。在每个线程的每个循环中,首先锁住队列,从队列中获取一个子图片\(blk\)、解锁队列然后使用上一章中的ErodeAndDilate过程进行处理。如果\(blocks\)中没有子图片,说明处理完成,则此线程退出。
\par 主线程的流程如图\ref{fig:pthreadMain}所示。在进行预处理过后启动多个线程,然后等待所有线程竞争子图像、处理然后结束即可,最后保存处理的结果即可。
\begin{figure}[htpb]
    \centering
    \includegraphics[width=0.95\linewidth]{pthreadMain.png}
    \caption{主线程流程}
    \label{fig:pthreadMain}
\end{figure}

\par 由于是使用pthread的并行算法,每一个线程处理一个部分,因此首先需要将数据分块(即分为算法中的\(blocks\))。分块方式如图\ref{fig:partition}所示。每块大小一样,在边缘部分如果块大小不符则按照原图的边缘进行裁减。因此,在进行处理时需要对于边缘部分进行考虑。
\begin{figure}[htpb]
    \centering
    \includegraphics[width=0.76\linewidth]{partition.png}
    \caption{分块方法}
    \label{fig:partition}
\end{figure}

\section{实验方案}
\par 所有的开发与运行环境见附录\ref{cha:env},表\ref{tab:env},此后实验的开发与运行环境均相同,不再赘述。根据算法描述、分块方法以及主线程的流程编写程序并运行,然后观察结果并与串行的程序比较。经过多轮的比较以及参数调试后得出一个较好的效果。

\section{实验结果与分析}
\par 功能上,程序处理后的图片与串行处理后的图片一致,此处不再给出。4线程,分块大小为128的情况下程序的运行时间如图\ref{fig:pthreadOutput}所示。在4个线程的情况下,运行三次的平均运行时间为11.7s,相比于串行算法,程序的加速比为\(44.2\div 11.7 = 3.77\),已经十分接近理想加速比4。
\begin{figure}[htpb]
    \centering
    \includegraphics[width=0.9\linewidth]{pthreadOutput.png}
    \caption{pthread程序运行时间}
    \label{fig:pthreadOutput}
\end{figure}

\par 经过8组、每组3次的测试,加速比随线程变化的曲线如图\ref{fig:pthreadTrend}所示。可以看出,在线程数为1\textasciitilde 4时加速比随着线程数几乎呈线性变化,而在线程数为1时加速比为1.006,overhead所占用的时间几乎可以不计。在线程数达到4时由于物理内核已经被占满,因此后面加速比不再增加,随着线程数量的进一步增大,由于线程调度的开销,因此程序的加速比不再增加,反而有所下降。
\begin{figure}[htpb]
    \centering
    \includegraphics[width=0.8\linewidth]{pthreadTrend.png}
    \caption{加速比随线程数变化}
    \label{fig:pthreadTrend}
\end{figure}

\par 对于分块大小而言,加速比随着分块大小的变化如图\ref{fig:pthreadTrend2}所示,在分块大小较小时,加速比随着分块大小的变化并不大,只在分块大小过小时由于线程调度导致一点性能开销。当分块大小大于原图的一半时总时间则取决于分到最大分块线程所用的时间,因此在这个区间内性能随分块大小呈下降趋势。
\begin{figure}[htpb]
    \centering
    \includegraphics[width=0.8\linewidth]{pthreadTrend2.png}
    \caption{加速比随分块大小变化}
    \label{fig:pthreadTrend2}
\end{figure}




% chapater 3
\chapter{User Environment}
\label{cha:user_environment}

\section{User Environments and Exception Handling}
\par inc/env.h中包含了基本的环境定义。在kern/env.c中,可以看到内核维护了3个全局变量来存储环境。
\begin{itemize}
    \item struct Env *envs = NULL; //ALL environments
    \item struct Env *curenv = NULL; The current env
    \item static struct Env *env\_free\_list; //Free environment list
\end{itemize}
\par jOS开始运行以后,env指针将指向一个存放系统各种环境的Env结构体数组。jOS内核最大支持NENV个同时活动的环境。jOS内核使用env\_free\_list维护所有不同的Env结构体,类似于空闲链表。内核使用curenv来表示当前运行的环境。内核启动前这个变量是NULL。

\subsection{Environment State}
\par Env结构体在inc/env中定义:
\inputCodeSetLanguage{c}
\begin{lstlisting}
struct Env {
    struct Trapframe env_tf;	// Saved registers
    struct Env *env_link;		// Next free Env
    envid_t env_id;			    // Unique environment identifier
    envid_t env_parent_id;		// env_id of this env's parent
    enum EnvType env_type;		// Indicates special system environments
    unsigned env_status;		// Status of the environment
    uint32_t env_runs;		    // Number of times environment has run

    // Address space
    pde_t *env_pgdir;		    // Kernel virtual address of page dir
};
\end{lstlisting}
\par 其中:
\begin{itemize}
    \item env\_tf:定义在inc/trap.h中,用于存放环境停止运行时寄存器的值。切换为内核模式的时候也会保存寄存器的值。
    \item env\_link:指向env\_free\_list中的下一个Env,env\_free\_list空闲链表的第一个env环境。
    \item env\_id:内核储存env\_id环境的父用户环境id。
    \item env\_type:用来特定环境。
    \item env\_status:这个变量可能为以下几种值:
        \begin{itemize}
            \item ENV\_FREE:这个Env是不活跃的,也就是说在env\_free\_list中。
            \item ENV\_RUNNABLE:这个Env正在等待被处理器运行。
            \item ENV\_RUNNING:这个Env结构体代表了正在运行的环境。
            \item ENV\_NOT\_RUNNABLE:当前环境是活跃的但是不准备运行。比如等待其他环境进行进程间通信。
            \item ENV\_DYING:Env是一个僵尸环境,这个环境下一次进入内核的时候会被释放。
        \end{itemize}
    \item env\_pgdir:这个变量存放这个环境的页目录的虚拟地址。
\end{itemize}
\par 类似于Unix,一个jOS环境中结合了``线程''和``地址空间''的概念。线程是由保护寄存器定义的,而地址空间是由env\_pgdir指向的页目录和页表定义。

\subsection{Allocating the Environments Array}
\par 在lab2中修改了mem\_init()内为pages数组分配了空间,而现在需要进一步修改mem\_init()来为Env分配一个相似的结构envs。
\exercise{1}{
    \par 修改kern/pmap.c中的mem\_init()来为envs分配空间并建立映射。这个数组应该正好包含NENV个Env结构。并且这个envs应该映射到用户制度的UENVS,这样用户进程可以读取。可以使用check\_kern\_pgdir()来检查代码是否正确。
}
\begin{exerciseSolution}{1}
    \par 首先是分配数组,在分配pages的代码后添加为envs分配空间的代码:
    \inputCodeSetLanguage{c}
    \begin{lstlisting}
envs = (struct Env*)boot_alloc(NENV*sizeof(struct Env));
memset(envs, 0, NENV*sizeof(struct Env));
    \end{lstlisting}
    \par 在分配完内存空间之后,接下来准备映射。因此在对于pages的映射之后对envs进行映射,添加如下代码:
    \begin{lstlisting}
boot_map_region(kern_pgdir, UENVS, PTSIZE, PADDR(envs), PTE_U);
    \end{lstlisting}

    \par 修改完成以后,重新编译运行,结果如图\ref{fig:lab3/exercise1_1}所示。可以看到,显示check\_kern\_pgdir() succeeded!,也就是说exercise1实验成功。
    \begin{figure}[htb]
        \centering
        \includegraphics[width=0.8\linewidth]{lab3/exercise1_1.png}
        \caption{修改mem\_init后的运行结果}
        \label{fig:lab3/exercise1_1}
    \end{figure}
    \FloatBarrier
\end{exerciseSolution}

\subsection{Creating and Running Environments}
\par 现在需要完善kern/env.c使之能够运行一个用户环境。由于没有文件系统,因此必须将内核设置为能够加载内核中的静态二进制程序映像文件。
\par Lab3里面的GNUMakefile文件在obj/user目录下生成了一系列二进制文件。通过kern/Makefrag能够将这些二进制文件直接链接到可执行文件中。通过链接器中的-b binary选项能够使这些文件被作为二进制文件链接到内核之后。
\exercise{2}{
    \par 在env.c中,完成以下函数:
    \begin{itemize}
        \item env\_init()
            \par 初始化所有在envs数组中的Env结构,并将其加入env\_free\_list中。此外还需要调用env\_init\_percput来配置段式内存管理硬件来将所有的分段分为0级(内核)以及3级(用户)。
        \item env\_setup\_vm()
            \par 分配页目录,初始化用户环境地址空间中和内核相关的部分。
        \item region\_alloc()
            \par 为用户环境分配物理空间。
        \item load\_icode()
            \par 像boot loader一样分析一个ELF文件,并将它的内容加载到用户环境下。
        \item env\_create()
            \par 使用env\_alloc和load\_icode函数分配空间并加载一个ELF文件到用户环境中。
        \item env\_run()
            \par 在用户模式下开始一个用户环境。
    \end{itemize}
}
\begin{exerciseSolution}{2}
    \par 对于env\_init函数而言,遍历envs数组中的Env结构体,把每一个Env的end\_id置0。实现的env\_init的代码如下:
    \begin{lstlisting}
void env_init(void) {
    int counter;
    env_free_list = NULL;
    for (counter = NENV - 1; counter >= 0; --counter) {
        envs[counter].env_id = 0;
        envs[counter].env_status = ENV_FREE;
        envs[counter].env_link = env_free_list;
        env_free_list = &envs[counter];
    }
    // Per-CPU part of the initialization
    env_init_percpu();
}
    \end{lstlisting}

    \par 然后填写env\_setup\_vm部分。env\_setup\_vm的函数的作用是初始化新的用户环境页目录表,但是只设置夜幕里表中和内核相关的页目录,而不映射用户目录。因此可以使用kern\_pgdir来设置env\_pgdir中的内容。最终补充的代码如下:
    \begin{lstlisting}
++p->pp_ref;
e->env_pgdir = (pde_t *)page2kva(p);
memcpy(e->env_pgdir, kern_pgdir, PGSIZE);
    \end{lstlisting}

    \par 接下来补充为用户环境分配len字节的空间的函数,然后映射到环境中的虚拟地址va,根据提示,va向下对齐,va+len向上对齐。
    \begin{lstlisting}
static void region_alloc(struct Env *e, void *va, size_t len) {
    struct PageInfo *page = NULL;
    va = ROUNDDOWN(va, PGSIZE);
    void *end = (void *)ROUNDUP(va + len, PGSIZE);
    for (; va < end; va += PGSIZE) {
        if (!(page = page_alloc(ALLOC_ZERO)))
            panic("region_alloc: alloc failed.");
        if (page_insert(e->env_pgdir, page, va, PTE_U | PTE_W))
            panic("region_alloc: page mapping failed.");
    }
}
    \end{lstlisting}

    \par load\_icode需要加载ELF二进制到用户内存。参考boot/main.c中的boot loader加载内核到内存,首先验证ELF文件的合法性,然后加载ph->
p\_type = ELF\_PROG\_LOAD的字段,在加载前需要注意使用lcr3切换到用户态的页目录,否则不能够正确的加载到用户内存空间。在加载完成并将多余位清零后,映射初始栈的一个页,最终完成的代码如下:
    \begin{lstlisting}
static void load_icode(struct Env *e, uint8_t *binary) {
    struct Elf *elf_header = (struct Elf *)binary;
    if (elf_header->e_magic != ELF_MAGIC)
        panic("load_icode: illegal ELF format.");
    lcr3(PADDR(e->env_pgdir));
    struct Proghdr *ph = (struct Proghdr *)((uint8_t *)(elf_header) + elf_header->e_phoff);
    struct Proghdr *eph = ph + elf_header->e_phnum;
    for (; ph < eph; ++ph) {
        if (ph->p_type == ELF_PROG_LOAD) {
            region_alloc(e, (void *)ph->p_va, ph->p_memsz);
            memmove((void *)ph->p_pa, binary + ph->p_offset, ph->p_filesz);
            memset((void *)(ph->p_pa + ph->p_filesz), 0, ph->p_memsz - ph->p_filesz);
        }
    }
    e->env_tf.tf_eip = elf_header->e_entry;
    lcr3(PADDR(kern_pgdir));
    region_alloc(e, (void *)(USTACKTOP - PGSIZE), PGSIZE);
}
    \end{lstlisting}

    \par 对于env\_create首先使用env\_alloc创建一个env,然后调用load\_icode来加载elf二进制镜像,最后设置env\_type。值得注意的是env的父id应该设置为0,其实现如下:
    \begin{lstlisting}
void env_create(uint8_t *binary, enum EnvType type) {
    struct Env *environment;
    if(env_alloc(&environment, 0))
        panic("env_create: env_alloc failed.");
    load_icode(environment, binary);
    environment->env_type = type;
}
    \end{lstlisting}

    \par 对于env\_run而言,首先切换判断当前环境是否为空,环境状态是否为ENV\_RUNNING,如果是则将环境设置为ENV\_RUNNABLE,然后将curenv设置为当前环境。设置状态ENV\_RUNNING,更新env\_runs计数器后切换到它的地址空间。使用env\_pop\_tf换源环境寄存器然后进入用户模式。实现的代码如下:
    \begin{lstlisting}
void env_run(struct Env *e) {
    if(curenv && curenv->env_status == ENV_RUNNING)
        curenv->env_status = ENV_RUNNABLE;
    curenv = e;
    curenv->env_status = ENV_RUNNING;
    ++curenv->env_runs;
    lcr3(PADDR(curenv->env_pgdir));
    env_pop_tf(&curenv->env_tf);
}
    \end{lstlisting}
\end{exerciseSolution}

\par 用户环境的代码被调用前,操作系统一共按顺序执行了以下几个函数:
\begin{itemize}
    \item start (kern/entry.S)
    \item i386\_init (kern/init.c)
        \begin{itemize}
            \item cons\_init
            \item mem\_init
            \item env\_init
            \item trap\_init (still incomplete at this point)
            \item env\_create
            \item env\_run
                \begin{itemize}
                    \item env\_pop\_tf
                \end{itemize}
        \end{itemize}
\end{itemize}
\par 完成上述函数的代码后重新编译运行,系统会进入用户空间并且开始执行hello程序,直到系统调用int指令。这个指令不能成功执行,因为jOS还没有设置相关硬件来实现从用户态向内核态转换的功能。当CPU发现它不能处理这种中断时会触发一个异常,然后发现这个异常也无法处理,直到产生第三个异常,但仍旧不能解决,因此将其叫做"triple fault"。我们可以使用调试器检查我们是否进入了用户模式。使用make qemu-gdb并在env\_pop\_tf处设置一个断点,然后单步执行,处理器会在执行完iret指令以后进入用户模式。该进入用户模式的第一条指令是一个cmp指令。然后使用b *0x...设置一个在obj/user/hello.asm中的断点中的sys\_cputs函数的int \$0x30处。这个int指令是一个系统调用,用来向控制台输出一个字符。如果你的程序不能运行到int指令说明程序有错误。
\par 按照上述过程进行调试,程序停止在了int \$0x30处,如图\ref{fig:lab3/exercise2_1}所示。说明程序功能正常。
\begin{figure}[htb]
    \centering
    \includegraphics[width=0.9\linewidth]{lab3/exercise2_1.png}
    \caption{程序停止在int \$0x30处}
    \label{fig:lab3/exercise2_1}
\end{figure}

\subsection{Handling Interrupts and Exceptions}
\par 现在需要一个异常处理及系统调用处理机制来使系统从用户态切换到内核态。
\exercise{3}{
    \par 阅读\textit{Chapter 9, Exceptions and Interrupts }\footnote{\url{https://pdos.csail.mit.edu/6.828/2017/readings/i386/c09.htm}}
}
%\begin{exerciseSolution}{3}
%    \par 从文中可以知道,中断分为可屏蔽中断以及不可屏蔽中断;异常分为处理器检测异常以及程序触发的异常。通过NMI、IF、RF以及修改SS可以使能或屏蔽中断。中断描述符表储存中断处理程序的入口地址,而中断的处理流程如图\ref{fig:lab3/exercise3_1}所示,通过IDT与GDT共同决定用于处理的程序。
%    \begin{figure}[htb]
%        \centering
%        \includegraphics[width=0.6\linewidth]{lab3/exercise3_1.png}
%        \caption{中断处理流程}
%        \label{fig:lab3/exercise3_1}
%    \end{figure}
%\end{exerciseSolution}

\subsection{Basics of Protected Control Transfer}
\par 异常和中断都是保护控制转移,让处理器从用户模式切换到内核模式。这样用户代码不会对内核造成任何影响。在intel处理器中,中断通常是外部设备引起的异步的保护控制转移,而异常则是由当前运行的代码引起的同步的保护控制转移。
\par 为了能够保证这些控制转移真的能被保护,处理器的中断/异常机制通常为用户态代码无权选择内核代码的执行起点,处理器只有在某些条件下才能进入内核态。在x86上,有2中机制配合来提供这种保护:
\begin{enumerate}
    \item 中断向量表:
        \par 处理器保证撞断和异常只能导致内核进入一些预先定好的入口。x86处理器可以有最多256个不同的中断和异常,而每一个都对应一个唯一的中断向量。一个中断向量的值是根据中断的来源决定的。CPU将使用这个向量作为中断向量表的索引,而这个表又是内核设置的。通过表项处理器会加载:
        \begin{itemize}
            \item 加载到EIP寄存器的值,也就是指向处理这种类型异常的内核代码指针。
            \item 加载到CS寄存器的值,包含特权级别0\textasciitilde 1。
        \end{itemize}
    \item 任务状态段:
        \par 处理器需要存放中断异常发生之前的旧的处理器状态,包括原EIP和CS值,以在中断处理之后能够还原到之前的状态。保存的这个位置必须要受到保护,不能随意被修改。
        \par 因此,处理西在处理中断时会导致特权级别由用户转级为内核级,将堆切换到内核内存中。处理器将SS, ESP, EFLAGS, CS, EIP和可选的错误码压入堆栈中,然后从中断描述符中加载CS和EIP,设置ESP和SS指向新的堆栈。
\end{enumerate}

\subsection{Types of Exceptions and Interrupts}
\par 所有的x86处理器内部产生的议程向量是0\textasciitilde 31之间的整数,也映射到了IDT的0\textasciitilde 31项。大于31的项只被软件中断所使用,也就是说可以被int触发,或者是异步的硬件中断。
\par 这一节将要扩展jOS的功能使之能够处理0\textasciitilde 31号的内部异常。下一节会让jOS处理48号软件中断。

\subsection{An Example}
\par 在这个例子中,假设处理器遇到了除0的问题。
\begin{enumerate}
    \item 处理器切换到TSS的SS0和ESP0对应的堆栈,在jOS中,这两个字段是GD\_KG和KSTACKTOP。
    \item 处理器将异常参数压入内核堆栈,并放在KSTACKTOP中。
        \inputCodeSetLanguage{bash}
        \begin{lstlisting}[numbers=none]
+--------------------+ KSTACKTOP
| 0x00000 | old SS   |     " - 4
|      old ESP       |     " - 8
|     old EFLAGS     |     " - 12
| 0x00000 | old CS   |     " - 16
|      old EIP       |     " - 20 <---- ESP
+--------------------+
        \end{lstlisting}
   \item 由于处理器错误在x86上是0号中断向量,因此去读IDT的第0项并设置CS:EIP指向中断处理程序。
   \item 处理函数接过控制权并处理异常。
\end{enumerate}
\par 对于确定类型的x86异常,除了上面标准的5个压栈元素外还有一个错误码。当处理器将错误码压栈时,栈是这样的:
\begin{lstlisting}[numbers=none]
+--------------------+ KSTACKTOP
| 0x00000 | old SS   |     " - 4
|      old ESP       |     " - 8
|     old EFLAGS     |     " - 12
| 0x00000 | old CS   |     " - 16
|      old EIP       |     " - 20
|     error code     |     " - 24 <---- ESP
+--------------------+
\end{lstlisting}

\subsection{Nested Exceptions and Interrupts}
\par 处理器在内核模式和用户模式都可以处理异常和中断。但是当内核从用户态进入内核态的时候,x86处理器会在压入旧的寄存器之前自动切换栈并通过IDT触发异常处理。如果当中断或异常发生时处理器已经在内核态了,那么CPU会在同一个栈上压入更多的值。这样,内核就能够处理嵌套中断;了。如果处理器已经在内核模式且正在处理嵌套异常,就不会保存SS和ESP寄存器,因此堆栈如下:
\begin{lstlisting}[numbers=none]
+--------------------+ <---- old ESP
|     old EFLAGS     |     " - 4
| 0x00000 | old CS   |     " - 8
|      old EIP       |     " - 12
+--------------------+
\end{lstlisting}
\par 如果处理器在内核模式处理异常,但栈空间不足,不能将旧的状态压入堆栈,那么处理器之后就不能恢复,只能重启。内核应该被设计为不允许这种事情发生。

\subsection{Setting Up the IDT}
\par 现在可以设置IDT表并处理JOS的内部异常了(中断向量0\textasciitilde 31)。最终需要实现的代码效果如下:
\inputCodeSetLanguage{bash}
\begin{lstlisting}[numbers=none]
       IDT              trapentry.S       trap.c
+----------------+
|   &handler1    |---> handler1:        trap (struct Trapframe *tf)
|                |        // do stuff    {
|                |        call trap        // handle the exception/interruput
|                |        // ...         }
+----------------+
|   &handler2    |---> handler2:
|                |       // do stuff
|                |       call trap
|                |       // ...
+----------------+
        ...
+----------------+
|   &handlerX    |---> handlerX:
|                |        // do stuff
|                |        call trap
|                |        // ...
+----------------+
\end{lstlisting}
\par 每一个中断或者异常结构都有它的中断处理函数,定义在trapentry.S中。trap\_init()初始化IDT表。每个处理函数都应该构建一个在Trapframe堆栈上的结构体,并调用trap()函数指向它。trap()则处理异常/中断。

\exercise{4}{
    \par 编辑trap.S以及trap.c并实现上述功能。对于每一个定义在inc/trap.h中的trap,都应该有一个函数入口应该被加在trapentry.S中。应该提供一个\_alltraps供TRAPHANDLER宏引用。要初始化idt表需要修改trap\_init函数,使表中的每一项指向定义在 trapentry.S 中的入口指针。实现的\_alltraps函数应该:
    \begin{enumerate}
        \item 将值压入堆栈,使堆栈看起来像一个Trapframe
        \item 加载GD\_KD进入\%ds以及\%es
        \item 使用pusl \%esp给Trapframe传递指针,作为trap()的参数
        \item 调用trap
    \end{enumerate}
}
\begin{exerciseSolution}{4}
    \par 首先,trapentry.S 中的宏定义TRAPHANDLER以及TRAPHANDLER\_NOEC定义了发生中断或异常时用于初始处理的函数。因为有些中断有错误码,有些没有,因此需要两个函数。通过参考\textit{80386 Programmer’s Manual 9.10 Error Code Summary}\footnote{\url{https://pdos.csail.mit.edu/6.828/2016/readings/i386/s09_10.htm}}可以知道哪些中断有错误码。因此trapentry.S修改如下:

\inputCodeSetLanguage{[x86masm]Assembler}
\begin{lstlisting}
TRAPHANDLER_NOEC(handler_divide,  T_DIVIDE)
TRAPHANDLER_NOEC(handler_debug,   T_DEBUG)
TRAPHANDLER_NOEC(handler_nmi,     T_NMI)
TRAPHANDLER_NOEC(handler_brkpt,   T_BRKPT)
TRAPHANDLER_NOEC(handler_oflow,   T_OFLOW)
TRAPHANDLER_NOEC(handler_bound,   T_BOUND)
TRAPHANDLER_NOEC(handler_illop,   T_ILLOP)
TRAPHANDLER_NOEC(handler_device,  T_DEVICE)
TRAPHANDLER_NOEC(handler_simderr, T_SIMDERR)
TRAPHANDLER_NOEC(handler_fperr,   T_FPERR)
TRAPHANDLER_NOEC(handler_mchk,    T_MCHK)
TRAPHANDLER_NOEC(handler_syscall, T_SYSCALL)
TRAPHANDLER(handler_dblflt, T_DBLFLT)
TRAPHANDLER(handler_tss,    T_TSS)
TRAPHANDLER(handler_segnp,  T_SEGNP)
TRAPHANDLER(handler_stack,  T_STACK)
TRAPHANDLER(handler_gpflt,  T_GPFLT)
TRAPHANDLER(handler_pgflt,  T_PGFLT)
TRAPHANDLER(handler_align,  T_ALIGN)

_alltraps:
    pushl %ds
    pushl %es
    pushal
    movw $GD_KD, %eax
    movw %ax, %ds
    movw %ax, %es
    pushl %esp
    call trap
\end{lstlisting}
\par 然后在trap.c中完成trap\_init,对于系统的IDT表进行初始化:
\inputCodeSetLanguage{c}
\begin{lstlisting}
void handler_divide();
void handler_debug();
void handler_nmi();
void handler_brkpt();
void handler_oflow();
void handler_bound();
void handler_illop();
void handler_device();
void handler_simderr();
void handler_fperr();
void handler_mchk();
void handler_syscall();
void handler_dblflt();
void handler_tss();
void handler_segnp();
void handler_stack();
void handler_gpflt();
void handler_pgflt();
void handler_align();

void
trap_init(void)
{
    extern struct Segdesc gdt[];

    SETGATE(idt[T_DIVIDE],  0, GD_KT, handler_divide,  0);
    SETGATE(idt[T_DEBUG],   0, GD_KT, handler_debug,   0);
    SETGATE(idt[T_NMI],     0, GD_KT, handler_nmi,     0);
    SETGATE(idt[T_BRKPT],   0, GD_KT, handler_brkpt,   3);
    SETGATE(idt[T_OFLOW],   0, GD_KT, handler_oflow,   0);
    SETGATE(idt[T_BOUND],   0, GD_KT, handler_bound,   0);
    SETGATE(idt[T_ILLOP],   0, GD_KT, handler_illop,   0);
    SETGATE(idt[T_DEVICE],  0, GD_KT, handler_device,  0);
    SETGATE(idt[T_SIMDERR], 0, GD_KT, handler_simderr, 0);
    SETGATE(idt[T_FPERR],   0, GD_KT, handler_fperr,   0);
    SETGATE(idt[T_MCHK],    0, GD_KT, handler_mchk,    0);
    SETGATE(idt[T_SYSCALL], 0, GD_KT, handler_syscall, 3);
    SETGATE(idt[T_DBLFLT],  0, GD_KT, handler_dblflt,  0);
    SETGATE(idt[T_TSS],     0, GD_KT, handler_tss,     0);
    SETGATE(idt[T_SEGNP],   0, GD_KT, handler_segnp,   0);
    SETGATE(idt[T_STACK],   0, GD_KT, handler_stack,   0);
    SETGATE(idt[T_GPFLT],   0, GD_KT, handler_gpflt,   0);
    SETGATE(idt[T_PGFLT],   0, GD_KT, handler_pgflt,   0);
    SETGATE(idt[T_ALIGN],   0, GD_KT, handler_align,   0);

    // Per-CPU setup
    trap_init_percpu();
}
\end{lstlisting}
\par 完成后使用make grade进行测试,可以发现divzero,softint以及badsegment测试通过,如图\ref{fig:lab3/exercise4_1}所示,说明IDT初始化正确。
\begin{figure}[htb]
    \centering
    \includegraphics[width=0.8\linewidth]{lab3/exercise4_1.png}
    \caption{使用make grade进行测试}
    \label{fig:lab3/exercise4_1}
\end{figure}
\FloatBarrier
\end{exerciseSolution}

\begin{questionEnv}
    \begin{enumerate}
        \item 对于每一个中断/异常都设置一个独立的处理函数的意义是什么?
        \item 你做了让user/softint正确执行的工作吗?grade script希望它产生一个general protection falt(trap 13),但是softint中为int \$14。为什么产生了中断向量13?如果系统允许int \$14调用kernel page fault处理函数?
    \end{enumerate}
\end{questionEnv}
\begin{answer}
    \begin{enumerate}
        \item 因为不同的中断可能需要不同的处理方式,比如有些中断需要返回,有些中断则不需要,还有些中断需要做额外的工作。
        \item 应为当先系统在用户态,而int为特权级别为0的指令,此时不能直接调用int指令,会引发general protection exception。如果允许int \$14处理,那么会导致用户态程序可能得到0级特权,造成保护失效。
    \end{enumerate}
\end{answer}

\section{Page Faults, Breakpoints Exceptions, and System Calls}
\subsection{Handling Page Faults}
\par 当处理器产生一个缺页异常时,它会将因此缺页异常的线性地址(虚拟的)存入CR2中。在trap.c中我们提供了一个特殊的函数page\_fault\_handler()用于处理缺页异常。
\exercise{5}{
    \par 修改trap\_dispatch函数使系统能够把缺页异常分发到page\_fault\_handler上。修改完成后运行make grade应该可以成功通过faultread, faultreadkernel, faultwrite, 以及faultwritekernel检查。
}
\begin{exerciseSolution}{5}
    \par trap\_dispatch是一个分发函数,通过Trapframe指针tf中的tf\_trapno来判断这个中断是什么中断。而在这一个exercise中,如果中断是缺页中断则调用page\_fault\_handler函数。考虑到之后可能要添加的中断类型,在trap\_dispatch中添加的代码如下:
    \inputCodeSetLanguage{c}
    \begin{lstlisting}
switch(tf->tf_trapno){
    case T_PGFLT:
        page_fault_handler(tf);
        break;
}
    \end{lstlisting}
    \par 重新编译运行后,通过了题目描述中的4个测试,如图\ref{fig:lab3/exercise5_1}所示。说明这部分代码实现成功。
    \begin{figure}[htb]
        \centering
        \includegraphics[width=0.8\linewidth]{lab3/exercise5_1.png}
        \caption{重新编译运行后的make grade输出}
        \label{fig:lab3/exercise5_1}
    \end{figure}
\end{exerciseSolution}

\subsection{The Breakpoint Exception}
\par 异常编号为3的断点异常能够让调试器给程序加上断点,也就是将要加断点的语句用一个int3指令替换,然后在执行到int3时触发中断。在jOS中需要将这个中断变为任何用户环境都能调用的伪系统调用。
\exercise{6}{
    \par 修改trap\_dispatch,使断点异常发生时能够调用kernel monitor。修改完成后重新make grade应该能够通过breakpoint测试。
}
\begin{exerciseSolution}{6}
    \par 与exercise 5类似,但是这里处理的是T\_BRKPT。调用kernel monitor需要使用kern/monitor.c中的monitor函数。修改后的trap\_dispatch中的前半部分内容如下:
    \inputCodeSetLanguage{c}
    \begin{lstlisting}
switch(tf->tf_trapno){
    case T_PGFLT:
        page_fault_handler(tf);
        break;
    case T_BRKPT:
        monitor(tf);
        break;
}
    \end{lstlisting}
    \par 修改完成后,成功通过breakpoint测试,如图\ref{fig:lab3/exercise6_1}所示。
    \begin{figure}[htb]
        \centering
        \includegraphics[width=0.8\linewidth]{lab3/exercise6_1.png}
        \caption{程序通过breakpoint测试输出}
        \label{fig:lab3/exercise6_1}
    \end{figure}
\end{exerciseSolution}

\begin{questionEnv}
    \begin{enumerate}
        \setcounter{enumi}{2}
        \item breakpoint exeception测试用例会征程一个breakpoint 异常或者general protection 错误,依赖于如何初始化IDT中的breakpoint entry。为什么?要怎么做才能让breakpoint exception正常工作?怎样的错误设置会导致触发general protection error?
        \item 这些机制有什么意义?尤其是对于user/softint中的测试程序而言?
    \end{enumerate}
\end{questionEnv}
\begin{answer}
    \begin{enumerate}
        \setcounter{enumi}{2}
        \item 如果在IDT中设置breakpoint exception时将DPL字段设置为0则会触发breakpoint exception,设置为3则会触发general protection exception。DPL字段为段描述符优先级,如果当前程序为用户态但是尝试调用内核态的指令的时候就会触发general protection exception。只有当前程序的优先级小于或等于段描述符优先级才能触发正确的breakpoint exception。
        \item 这些机制保证用户环境不能随意访问内核态的代码和内存,保护内核不受用户程序的破坏。
    \end{enumerate}
\end{answer}

\subsection{System calls}
\par 用户程序会要求内核通过系统调用的方式帮其完成一些任务。当用户程序触发系统调用时,处理器进入内核态并保存用户的处理状态。内核处理完成后返回用户程序,但具体的细节随系统的不同而不同。
\par jOS使用int来处理系统调用。特别的,使用int \$0x30作为系统调用中断。常量T\_SYSCALL就是0x30。0x30不能被外部硬件产生,因此没有任何歧义。
\par 应用程序会把系统调用和参数放到寄存器中,通过这种方法内核就不需要查询用户程序的堆栈了。系统调用号存放到\%eax中,参数则存放在 \%edx, \%ecx, \%ebx, \%edi, 和 \%esi中。返回值存放到\%eax@s中。lib/syscall.c中已有了触发系统调用的方法。
\exercise{7}{
    \par 通过编辑kern/trapentry.S以及kern/trap.c的trap\_init(),给T\_SYSCALL添加一个中断向量处理函数。同时trap\_dispatch也需要被修改,通过调用syscall的方法来处理系统调用。最后,需要在kern/syscall.c中首先实现syscall函数。如果系统调用号不合法,需要syscall返回-E\_INVAL。
    \par 通过make run-hello运行user/hello,qemu应该打印处hello, world,并触发一个page fault。并且make grade应该能够通过testbss测试。
}
\begin{exerciseSolution}{7}
    \par 在用户态执行系统调用时,首先产生了中断30,因此在kern/trapentry.S中添加一个处理函数声明TRAPHANDLER\_NOEC(handler\_syscall, T\_SYSCALL),并在trap\_init中添加handler\_syscall的声明以及在trap\_init中通过SETGATE(idt[T\_SYSCALL], 0, GD\_KT, t\_syscall, 3);将其注册到IDT,这些在exercise 5中已经完成,此时系统已经能够正确捕捉int 30了。
    \par 观察lib/syscall.c的syscall,发现其就是执行了int指令并取回了返回值,而对于这条int指令的处理,则是在kern/syscall.c中进行的。首先,完成kern/trap.c中对于中断的分发,也就是在switch中加入如下几行:
    \inputCodeSetLanguage{c}
    \begin{lstlisting}
case T_SYSCALL:
    tf->tf_regs.reg_eax = syscall(
            tf->tf_regs.reg_eax,
            tf->tf_regs.reg_edx,
            tf->tf_regs.reg_ecx,
            tf->tf_regs.reg_ebx,
            tf->tf_regs.reg_edi,
            tf->tf_regs.reg_esi);
    break;
    \end{lstlisting}
    \par 然后在kern/syscall.c中完成对于syscall的实现,从而完成对于整个int指令的调用:
    \begin{lstlisting}
int32_t syscall(uint32_t syscallno, uint32_t a1,
        uint32_t a2, uint32_t a3, uint32_t a4, uint32_t a5) {
    switch (syscallno) {
        case SYS_cputs:
            sys_cputs((char *)a1, a2);
            return 0;
        case SYS_cgetc:
            return sys_cgetc();
        case SYS_getenvid:
            return sys_getenvid();
        case SYS_env_destroy:
            return sys_env_destroy(a1);
        default:
            return -E_INVAL;
    }
}
    \end{lstlisting}
    \par 填写完成后,重新编译并运行qemu,输出hello world并触发缺页中断。如图\ref{fig:lab3/exercise7_1}所示。
    \begin{figure}[htb]
        \centering
        \includegraphics[width=0.8\linewidth]{lab3/exercise7_1.png}
        \caption{输出hello world并触发缺页中断}
        \label{fig:lab3/exercise7_1}
    \end{figure}
    \par 运行make grade,成功通过testbss测试,如图\ref{fig:lab3/exercise7_2}所示,说明系统调用实现成功。
    \begin{figure}[htb]
        \centering
        \includegraphics[width=0.8\linewidth]{lab3/exercise7_2.png}
        \caption{成功通过testbss测试}
        \label{fig:lab3/exercise7_2}
    \end{figure}
\end{exerciseSolution}

\subsection{User-mode startup}
\par 用户模式开始运行的地方是lib/entry.S。在该文件中首先进行一些设置,然后调用libmain。应该修改libmain()来初始化全局指针thisenvz指向envs数组中的Env结构体。
\par 然后libmain调用main,也就是user/hello.c中被调用的函数。在之前的实验中发现hello.c只能打印hello world并报出page fault异常,而其原因就是this->env\_id语句。如果正确初始化了thisenv就不会报错了。
\exercise{8}{
    \par 补全用户库中的代码并启动内核,user/hello应该打印出hello, world然后打印i am environment 00001000。通过调用sys\_env\_destroy(),user/hello会尝试退出。由于内核仅仅支持一个用户环境,因此它应该显示用户环境已被销毁的信息,然后退回kernel monitor。完成后make grade应该能够通过hello test测试。
}
\begin{exerciseSolution}{8}
    \par 在libmain中,修thisenv让其指向env当前环境的env即可。使用sys\_getenvid来获得当前的环境id。因此将libmain中的thisenv修改为如下即可。
    \inputCodeSetLanguage{c}
    \begin{lstlisting}
thisenv = envs + ENVX(sys_getenvid());
    \end{lstlisting}
    \par 重新编译运行,并运行make grade,输出如图\ref{fig:lab3/exercise8_1}所示,此时能够通过hello测试。
    \begin{figure}[htb]
        \centering
        \includegraphics[width=0.8\linewidth]{lab3/exercise8_1.png}
        \caption{程序通过hello测试}
        \label{fig:lab3/exercise8_1}
    \end{figure}
\end{exerciseSolution}

\subsection{Page faults and memory protection}
\par 内存保护是操作系统的一个非常重要的特性,能够保证bug不能够损坏其他的程序或者操作系统。操作系统通常依赖于硬件来完成这一功能。操作系统能够让硬件知道那笑虚拟地址是有效的。当程序尝试访问一个无效地址或者越权操作时就会触发异常。如果异常是可以修复的,那么就会修复异常并继续运行程序,否则就不会继续运行。
\par 在许多操作系统中,内核在初始情况下只会分配一个内核堆栈。如果程序想要访问这个堆栈之外的堆栈空间,就会触发异常,内核会自动分配页个程序然后继续让程序运行。
\par 但是系统调用需要存在问题,大部分系统系统调用接口让用户传递一个指向用户缓冲区的指针给内核,但是:
\begin{enumerate}
    \item 内核中的page fault比用户中的page fault严重。如果内核出现page fault,那么这是内核bug,而且异常处理会中断内核的执行。但是当内核解引用用户程序的时候,它需要一种方法标记这些page fault确实是由用户引起的。
    \item 内核通常比用户程序由更高级别的权限。用户程序可能会传递一个内核可读写但是用户不行的指针。此时内核不能对其进行解引用,否则可能泄露内核信息。
\end{enumerate}

\begin{exerciseEnv}{9}
    \par 修改kern/trap.c,使其能够检测内核模式下的page fault发生并发出kernel panic。阅读kern/pmap.c中的use\_mem\_assert并实现其中的use\_mem\_check。修改ker/syscall.c检查输入参数。
    \par 启动内核并运行user/buggyhello。环境应该被摧毁,但是不应该出现kernel panic。应该能够看到:
    \inputCodeSetLanguage{bash}
    \begin{lstlisting}[numbers=none]
[00001000] user_mem_check assertion failure for va00000001
[00001000] free env 00001000
Destroyed the only environment - nothing more to do!
    \end{lstlisting}
    \par 最后,修改kern/kdebug.c中的debuginfo\_eip来运行user\_mem\_check检查use, stabs和stabstr。如果现在运行user/breakpoint,应该能够从内核监视器中运行backtrace来检查在page fault之前检查lib/libmain.c。是什么导致了page fault?
\end{exerciseEnv}

\begin{exerciseSolution}{9}
    \par 首先检查page fault是否在内核模式,即检查Trap Frame中的tf\_cs,在page\_fault\_handler中添加的代码如下:
    \inputCodeSetLanguage{c}
    \begin{lstlisting}
if(tf->tf_cs == GD_KT)
    panic("page_fault_handler: kernel page fault");
    \end{lstlisting}
    \par pmap.c中的user\_mem\_check。阅读user\_mem\_assert, 可以发现它调用了user\_mem\_check。然而user\_mem\_check的功能当前永辉态程序是否有对于$[va, va+len)$的perm|PTE\_P的访问权限。因此在user\_mem\_check中查看用户态程序中的页表项,然后检查其perm|PTE\_P。最终实现的程序如下:
    \begin{lstlisting}
int user_mem_check(struct Env *env, const void *va, size_t len, int perm) {
    uint32_t start = (uint32_t)ROUNDDOWN(va, PGSIZE);
    uint32_t end = (uint32_t)ROUNDUP(va + len, PGSIZE);
    pte_t *page;
    for (; start < end; start += PGSIZE) {
        page = pgdir_walk(env->env_pgdir, (void *)start, 0);
        if (!page || start > ULIM || ((uint32_t)(*page) & perm) != perm ) {
            if (start <= (uint32_t)va)
                user_mem_check_addr = (uintptr_t)va;
            else
                user_mem_check_addr = (uintptr_t)start;
            return -E_FAULT;
        }
    }
    return 0;
}
    \end{lstlisting}
    \par 接下来对于kern/syscall.c进行补全。通过观察发现需要补全的是sys\_cputs函数。通过注释可以发现需要用户程序检查用户对于虚拟地址空间$[s, s+len)$是否具有访问权限,而这个则可以用上面实现的user\_mem\_assert实现。这个函数补全后如下:
    \begin{lstlisting}
static void sys_cputs(const char *s, size_t len) {
    user_mem_assert(curenv, s, len, 0);
    cprintf("%.*s", len, s);
}
    \end{lstlisting}
    \par 最后修改kern/kdegbug.c中的debuginfo\_eip。添加如下代码:
    \begin{lstlisting}
if(user_mem_check(curenv, usd, sizeof(struct UserStabData), PTE_U))
    return -1;
    \end{lstlisting}
    \par 运行make run-breakpoint,显示如图\ref{fig:lab3/exercise9_1}所示。可以看到,输入backtrace能够显示backtrace之前能够追踪进入libmain.c
    \begin{figure}[htb]
        \centering
        \includegraphics[width=0.7\linewidth]{lab3/exercise9_1.png}
        \caption{运行user/breakpoint后的结果}
        \label{fig:lab3/exercise9_1}
    \end{figure}
    \par 通过gdb进行追踪,发现是由于执行mon\_backtrace时进行追踪时到达了用户栈顶,然后在打印参数时访问的第六个参数超过了用户栈的大小,如图\ref{fig:lab3/exercise9_2}所示。
    \begin{figure}[htb]
        \centering
        \includegraphics[width=0.9\linewidth]{lab3/exercise9_2.png}
        \caption{使用gdb进行追踪}
        \label{fig:lab3/exercise9_2}
    \end{figure}
    \par 最后,使用make grade 进行测试,发现能够正常通过所有测试,如图\ref{fig:lab3/exercise9_3}所示。
    \begin{figure}[htb]
        \centering
        \includegraphics[width=0.6\linewidth]{lab3/exercise9_3.png}
        \caption{make grade通过所有测试}
        \label{fig:lab3/exercise9_3}
    \end{figure}
    \FloatBarrier
\end{exerciseSolution}

\begin{exerciseEnv}{10}
    \par 启动你的内核并运行user/evilhello。环境应该被摧毁并且内核不应该panic。你应该能够看到:
    \inputCodeSetLanguage{bash}
    \begin{lstlisting}[numbers=none]
[00000000] new env 00001000
...
[00001000] user_mem_check assertion failure for va f010000c
[00001000] free env 00001000
    \end{lstlisting}
\end{exerciseEnv}
\begin{exerciseSolution}{10}
    \par 运行evilhello,结果如图\ref{fig:lab3/exercise10_1}所示。环境被摧毁且没有发生kernel panic。
    \begin{figure}[htb]
        \centering
        \includegraphics[width=0.6\linewidth]{lab3/exercise10_1.png}
        \caption{evilhello运行结果}
        \label{fig:lab3/exercise10_1}
    \end{figure}
    \FloatBarrier
\end{exerciseSolution}




\chapter{目标代码生成}
\label{cha:mu_biao_dai_ma_sheng_cheng_}

\section{实验设计}
\label{sec:shi_yan_she_ji_4}
\par 在生成了TAC后,剩下的部分就是目标代码生成了。在这一部分中,程序将由能力生成可以运行的MIPS指令。
\par 由于前面已经生成了三地址码,生成对应的mips指令只需将一种线性结构转换为另一种线性结构,这一部通过CodeGenerator类中的generateFinalCode完成。其中的核心部分在于寄存器的分配。可供分配的寄存器数量有限,而TAC中的临时变量一般远远超出可以分配的寄存器数量,因此需要知道那些变量不再使用,便于将其所绑定的寄存器空出给其他的变量使用。
\par 要完成这一点,需要对程序进行数据流分析,而其中的第一步则是构建程序的流图。在本实验中,使用一个ControlFlowGraph作为存储流图的数据结构,对于流图这样一个有向无环图,ControlFlowGrap采用双向邻接表的数据结构表示。对于个节点的出度使用一个矩阵,入度使用另一个,并分别建立ForwardFlow类以及BackwardFlow类作为原图的遮罩,便于图的前向遍历和方向遍历。
\par 生成最终代码的大体框架分为3步:
\begin{enumerate}
    \item 遍历CodeGenerator中的TAC指令,生成CFG。
    \item 遍历CFG,生成活跃变量表。
    \item 遍历CodeGenerator中的TAC指令,结合活跃变量表,完成最终的寄存器分配以及代码生成。
\end{enumerate}

\inputCodeSetLanguage{c++}
\par 这些步骤均在generateFinalCode中完成。活跃变量表采用\lstinline|list<map<Location,Instruction>>|的结构实现,其中,外层的list为源程序中不同的函数,每个函数有自己的CFG, 而内层的map则是对于cfg进行分析后的活跃变量表,它将变量(Location类,变量使用变量地址表示)映射到\textbf{最后一次使用它的指令}。这样,在后续的遍历过程中,只需要查找所使用的变量是否在活跃变量表中,如果在,则比对当前的指令是否与其最后一次使用的指令相同,如果相同,则说明这是最后一个使用这个变量的指令。此变量已经可以被丢弃,此后不会再被使用了。
\par 第三步执行的正是这样的工作:遍历TAC指令并决定是否丢弃用到的变量。为了减少程序的耦合度,不因为丢弃变量而干扰其他类,使用上一章中提到的DiscardValue这一伪TAC指令生成目标指令,这样解绑寄存器的策略就决定与最终用于生成目标指令的类而不是CodeGenerator这一生成TAC指令的类了。
\par 与从AST到TAC类似,为了便于程序的控制,减小各个模块之间的耦合度,采用Mips类进行最终的代码生成。由于产生的TAC指令与LLVM后端不兼容,要使用LLVM生成则过于复杂,因此使用此类直接硬编码生成MIPS指令。由于将TAC生成与最终代码的生成分离,因此可以轻易的将Mips类替换成其他类,用以生成x86, arm等一系列平台的目标代码。对于本实验中的Mips类而言,其通过不同的函数将TAC类映射到对应的Mips代码的生成方法,而Instruction类的各个子类通过虚函数重载了generateSpecific方法用于调用生成与自身TAC类相关的Mips指令,因此在CodeGenerator中只调用每个TAC指令的generateSpecific方法以及丢弃指令的generateSpecific即可生成最终的Mips代码。由于未作特别的代码优化,在Mips类中,除了特殊寄存器(如sp)保留之外,其余的空闲寄存器均是随机分配的,以达到均匀使用所有寄存器的效果。

\section{实验步骤}
\label{sec:shi_yan_bu_zou_4}
\par 作为程序流图的基础,首先构建ControlFlowGraph类,然后对于其中的mapLabels以及mapEdges方法进行编写,对于普通TAC指令以及可能造成跳转的四种指令:LCall、ACall、IfZ以及Goto进行映射,从而构建CFG。
\par 在CFG类实现完成后,实现CodeGenerator中的generateFinalCode方法,实现上述框架中的三个步骤,即生成CFG、生成活跃变量表以及寄存器分配和生成最终代码。
\par 最后,在AST的主节点生成TAC后调用generateFinal方法,构造出一个完整的编译器。

\section{实验结果及分析}
\label{sec:shi_yan_jie_guo_ji_fen_xi_4}
\par 在编译器构建完成后,继续采用附录\ref{cha:ce_shi_yong_decafdai_ma_}中的sort程序进行测试。将程序输入编译器,得到的汇编文件部分如图\ref{fig:asm}所示。
\begin{figure}[htpb]
    \centering
    \includegraphics[width=0.74\linewidth]{asm.png}
    \caption{生成的汇编文件(部分)}
    \label{fig:asm}
\end{figure}

\par 由于之前在计算机组成原理课堂上实现的Mips CPU只能支持部分指令且不支持Ascii I/O,因此最终选择在Mips模拟平台Mars上进行模拟。模拟结果如图\ref{fig:simulate}所示。测试中,对于3, 6, 4, 2, 7, 12, 1, 0, 5, 10这10个数进行排序。从图中可以看出,排序结果正确,可以说明Decaf程序正确的完成了它的功能,编译器功能正常。
\begin{figure}[htpb]
    \centering
    \includegraphics[width=0.8\linewidth]{simulate.png}
    \caption{Mars模拟编译后的程序运行结果}
    \label{fig:simulate}
\end{figure}

\chapter{实验心得}
\label{cha:shi_yan_xin_de_}
\par 在本次实验中,我借助flex,bison等工具从零开始制造了一个Decaf编译器,在这个过程中我收获了许多。首先,是对于编译器工作原理的理解,从前端到后端,从大体的流程框架到小的细节,我更加深入的理解了编译器是如何工作的,各个数据结构之间是如何转换与协同运作的。通过flex与bison的编写,我加深了对于课堂上所讲的源程序如何转换词汇表的理解
\par 除了编译原理有关的知识外,我还在编写的过程中对于C++的使用有了更深入一步的了解。经过反复的思考与设计,如何利用一个语言的特性来高效的设计编译器等。最重要的是,在这一个过程中,我体会到了如何完整的设计一个系统,使之可以无缝接合,正常工作,同又不至于耦合度过高,并具有一定的可扩展性。
\par 这次实验并不是一个十分完美的实验,还有许多可扩展的空间,如代码优化,寄存器优化,以及生成LLVM可识别的TAC等等。在时间有限的情况下,选择需要需要实现的重点内容对我来说也是一次十分宝贵的经验。这些所有的经验在将来的学习过程中也会由很大的帮助。

\appendix
{\let\clearpage\relax \chapter{程序使用方法}}
\label{cha:cheng_xu_shi_yong_fang_fa_}
\par 要编译本程序,需要在Linux环境下进行,并保证环境中安装有表\ref{tab:environment} 所示的同版本或更高版本的软件。在Makefile所在目录下运行make就可以自动编译了。
\par 生成的二进制文件为decafcc,dcc为使用bash脚本对其的包装。在使用时,推荐使用dcc这一包装。使用命令为\lstinline|./dcc <file>|,file为用于测试的decaf文件的路径。
\par 使用上述命令会生成\lstinline|tokens.out, ast.out, scope.out, tac.out, a.out|这5个文件,分别为词法分析结果,语法分析结果,符号表以及作用域分析结果,三地址码中间代码以及目标代码。使用cat命令可以直接在终端显示这些文件的内容。由于部分文件使用了linux颜色转义字符\footnote{\url{https://misc.flogisoft.com/bash/tip_colors_and_formatting}},因此使用文本编辑器打开极有可能不能正确显示,推荐在现代终端下使用cat命令进行输出。使用Mars 4.5及以上的版本打开a.out文件可以进行目标代码的模拟运行。

\chapter{测试用Decaf代码}
\label{cha:ce_shi_yong_decafdai_ma_}
\par 在本报告中使用的测试用decaf文件为一排序程序,程序如下。使用者也可以编写符合Decaf语法规范的程序进行测试。
\inputCodeSetLanguage{c++}
\begin{lstlisting}
int[] ReadArray() {
    int i;
    int num;
    int [] arr;
    int numScores;

    Print("How many numbers to sort? ");
    numScores = ReadInteger();
    arr = NewArray(numScores, int);
    i = 0;
    while (i < arr.length()) {
        Print("Enter next number: ");
        num = ReadInteger();
        arr[i] = num;
        i = i + 1;
    }
    return arr;
}

void Sort(int []arr) {
    int i;
    int j;
    int val;

    i = 1;
    while (i < arr.length()) {
        j = i - 1;
        val = arr[i];
        while (j >= 0) {
            if (val >= arr[j])
                break;
            arr[j + 1] = arr[j];
            j = j - 1;
        }
        arr[j + 1] = val;
        i = i + 1;
    }
}

void PrintArray(int []arr) {
    int i;
    i = 0;
    Print("Sorted results: ");
    while (i < arr.length()) {
        Print(arr[i], " ");
        i = i + 1;
    }
    Print("\n");
}


void main() {
    int[] arr;

    Print("\nThis program will read numbers and print them sorted\n");
    arr = ReadArray();
    Sort(arr);
    PrintArray(arr);
}
\end{lstlisting}

{\let\clearpage\relax \chapter{编译器源代码}}
\label{cha:bian_yi_qi_yuan_dai_ma_}

\par 各个文件的内容及作用如表\ref{tab:file}所示。
\begin{center}
    \begin{longtable}{r l r l}
        \caption{caption}
        \label{tab:file} \\

        \toprule
        \multicolumn{1}{c}{\textbf{文件}} &
        \multicolumn{1}{c}{\textbf{作用}} &
        \multicolumn{1}{c}{\textbf{文件}} &
        \multicolumn{1}{c}{\textbf{作用}} \\
        \cmidrule(lr){1-1} \cmidrule(lr){2-2}
        \cmidrule(lr){3-3} \cmidrule(lr){4-4}
        \endfirsthead

        \toprule
        \multicolumn{1}{c}{\textbf{文件}} &
        \multicolumn{1}{c}{\textbf{作用}} &
        \multicolumn{1}{c}{\textbf{文件}} &
        \multicolumn{1}{c}{\textbf{作用}} \\
        \cmidrule(lr){1-1} \cmidrule(lr){2-2}
        \cmidrule(lr){3-3} \cmidrule(lr){4-4}
        \endhead
        Doxyfile       & 文档生成配置文件       & ast\_stmt.cpp & 语法树语句结点实现\\
        Makefile       & Make文件               & ast\_stmt.h   & 语法树语句结点声明\\
        ast.cpp        & 语法树基类实现         & ast\_type.cpp & 语法树类型结点实现\\
        ast.h          & 语法树基类声明         & ast\_type.h   & 语法树类型结点声明\\
        ast\_decl.cpp  & 语法树声明节点实现     & codegen.cpp   & TAC代码生成器实现\\
        ast\_decl.h    & 语法树声明节点声明     & codegen.h     & TAC代码生成器声明\\
        ast\_expr.cpp  & 语法树表达式节点实现   & dcc           & 可执行文件decafcc包装\\
        ast\_expr.h    & 语法树表达式节点声明   & defs.asm      & decaf链接库\\
        errors.cpp     & 错误处理实现           & mips.cpp      & Mips指令生成器实现\\
        errors.h       & 错误处理声明           & mips.h        & Mips指令生成器声明\\
        flow.cpp       & CFG实现                & parser.h      & 语法分析器导出符号\\
        flow.h         & CFG声明                & parser.y      & 语法分析器bison文件\\
        hash.h         & 哈希表声明及实现       & printer.cpp   & 输出管理器实现\\
        list.h         & 链表声明及实现         & printer.h     & 输出管理器声明\\
        main.cpp       & 程序入口               & scanner.h     & 词法分析器导出符号\\
        scope.cpp      & 作用域声明             & scanner.l     & 词法分析器flex文件\\
        scope.h        & 作用域实现             & utility.h     & 工具类声明\\
        tac.cpp        & 三地址码实现           & utility.cpp   & 工具类实现\\
        tac.h          & 三地址码声明           & tags          & 索引标签\\
        \bottomrule
    \end{longtable}
\end{center}

\NewDocumentCommand{\appendixCode}{v m}{
    \noindent\par #1
    \inputCodeSetLanguage{#2}
    \lstinputlisting{../src/#1}
}
\par 部分关键源码如下(全部源码过于冗长,此处不再全部放出,详见源文件):
\appendixCode{scanner.l}{c++}
\appendixCode{parser.y}{c++}
\appendixCode{ast.h}{c++}
\appendixCode{codegen.h}{c++}
\appendixCode{mips.h}{c++}
\appendixCode{dcc}{bash}
\appendixCode{Makefile}{make}

\addcontentsline{toc}{chapter}{参考文献}
\nocite{*}
\bibliographystyle{plain}
\bibliography{report}

\vfill
{\tiny written by HuSixu \hfill powered by \XeLaTeX .}
\end{document}
