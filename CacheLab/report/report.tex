%! TEX program = xelatex
\documentclass{report}
% provides basic settings for ctex document
\usepackage[UTF8, heading=true]{ctex}
\usepackage{fancyhdr}
\usepackage{tocloft}
\usepackage[margin=1in]{geometry}
\usepackage{metalogo}                   % \XeLaTeX
\usepackage{float}                      % figure H flag
\usepackage{microtype}                  % break long words
\usepackage[hidelinks]{hyperref}
\usepackage{tabularx}
\usepackage{amsmath}
\usepackage{lmodern}                    % allow fonts to scale
\usepackage{placeins}
\usepackage{multirow}                   % multirow, multicolumn support
\usepackage{booktabs}                   % toprule, cmidrule support
\usepackage{caption}

% make chapter stay in the same page
%\makeatletter
%\renewcommand\chapter{\thispagestyle{plain}%
%\global\@topnum\z@
%\@afterindentfalse
%\secdef\@chapter\@schapter}
%\makeatother

\fancyhead{}
\renewcommand{\sectionmark}[1]{\markleft{#1}}
\renewcommand{\partmark}[1]{\markright{#1}}
\lhead{\tiny \leftmark}
\rhead{\tiny \rightmark}
% see $(texdoc ctex) for details
\ctexset{
    chapter = {
        name = {实验},
        format += \flushleft,
        number = \arabic{chapter},
    },
    section = {
        format += \flushleft,
    },
    appendix = {
        number = \Alph{chapter},
        name = {附录},
    },
}

\pagestyle{fancy}
%\setlength\cftaftertoctitleskip{2em}

% provides code input support
\usepackage{xparse}                     % newcommand multiple optional arguments
\usepackage{listings}                   % code
\usepackage{fontspec}
\usepackage{lmodern}

%\newfontfamily\codeF{Fira Code}

\setmonofont[
    Contextuals={Alternate},
    ItalicFont = Fira Code      % to avoid font warning
]{Fira Code}

% usage: \inputCode{[language] <path>}
% if language is not explicitly set, it's defaulted to c
\DeclareDocumentCommand{\inputCode}{ O{c} m }{
    {
        \lstinputlisting[
            basicstyle=\small\ttfamily,
            language={#1},
            tabsize=4,
            showstringspaces=false,
            breaklines=true,
            frame=shadowbox,
            framexleftmargin=10mm,
            rulesepcolor=\color{black},
            numbers=left,
            xleftmargin=4em,
        ]{#2}
    }
}

\DeclareDocumentCommand{\inputCodeSetLanguage}{ m }{
    \lstset{
        basicstyle=\small\ttfamily,
        language={#1},
        tabsize=4,
        showstringspaces=false,
        breaklines=true,
        frame=shadowbox,
        framexleftmargin=10mm,
        rulesepcolor=\color{black},
        numbers=left,
        xleftmargin=4em,
    }
}

\DeclareDocumentCommand{\inputCodeNoNumberSetLanguage}{ m }{
    \lstset{
        basicstyle=\small\ttfamily,
        language={#1},
        tabsize=4,
        showstringspaces=false,
        breaklines=true,
        frame=shadowbox,
        rulesepcolor=\color{black},
    }
}



%%%%%%%%%%%%%%%%%%%%%%%%%%%%%%%%
\graphicspath{{./res/}}

% report content %%%%%%%%%%%%
%%%%%%%%%%%%%%%%%%%%%%%%%%%%%
\begin{document}

% cover page
\begin{titlepage}
    \addtolength{\topmargin}{1cm}
    \centering
    \includegraphics[width=0.6\textwidth]{hust.jpg}\par
    \vspace{0.5cm}
    {\Huge \heiti 编译原理实验报告}\par
    \vspace{10cm}
    {
        \large
        \begin{tabular}{r m{8em}}
            \makebox[6em][s]{学生姓名}:& 胡思勖 \\ \cline{2-2}
            \makebox[6em][s]{学号}:& U201514898\\ \cline{2-2}
            \makebox[6em][s]{专业}:& 计算机科学与技术\\ \cline{2-2}
            \makebox[6em][s]{班级}:& 计卓1501\\ \cline{2-2}
            \makebox[6em][s]{指导教师}:& 徐丽萍\\ \cline{2-2}
        \end{tabular}
    }
    \vfill
    2018-06-23
\end{titlepage}

\setcounter{tocdepth}{1}
\pagenumbering{Roman}
\tableofcontents

\newpage
\pagenumbering{arabic}
\setcounter{page}{1}



% chapter 1
\chapter{Building a Cache Simulator}
\label{cha:building_a_cache_simulator}
\par 整个实验分为两个部分,在第一部分中,需要实现一个缓存模拟器。通过valgrind中的lackey工具\footnote{\url{http://valgrind.org/docs/manual/lk-manual.html}}可以得到一个程序几乎所有的内存访问情况,使用如下命令即可获得这些信息:
\inputCodeNoNumberSetLanguage{bash}
\begin{lstlisting}[numbers=none]
linux> valgrind --log-fd=1 --tool=lackey --trace-mem=yes <program>
\end{lstlisting}

\par 一个样例输出如下:
\begin{lstlisting}[numbers=none]
I 0400d7d4,8
 M 0421c7f0,4
 L 04f6b868,8
 S 7ff0005c8,8
\end{lstlisting}

\par 输出的格式为``操作\ 地址,大小'',I表示指令加载,L表示数据加载,S表示数据存储,M表示数据修改,而数据修改应该被当做一次数据加载加上一次数据存储。内存地址以十六进制的形式给出。

\section{实验要求}
\label{sec:shi_yan_yao_qiu_}

\par 此实验要求编写一个Cache模拟器,其输入为Valgrind输出的内存访问轨迹,输出为与csim-ref相同的统计数据。

\par 实验要求:
\begin{itemize}
    \item 模拟器必须在输入参数s、E、b设置为任意值时均能正确工作——即需要使用malloc函数(而不是代码中固定大小的值)来为模拟器中数据结构分配存储空间。
    \item 由于实验仅关心数据Cache的性能,因此模拟器应忽略所有指令cache访问(即轨迹中“I”起始的行)
    \item 假设内存访问的地址总是正确对齐的,即一次内存访问从不跨越块的边界——因此可忽略访问轨迹中给出的访问请求大小
    \item main函数最后必须调用printSummary函数输出结果,并如下传之以命中hit、缺失miss和淘汰/驱逐eviction的总数作为参数:
\end{itemize}

\par 在编写完成后,使用test-csim程序进行测试以及评分。Cache模拟器使用的策略应为LRU替换策略。

\section{实验设计}
\label{sec:shi_yan_she_ji_}

\subsection{总体设计}
\label{sub:zong_ti_she_ji_}

\par 需要修改的文件为csim.c。由于已经给出了框架,首先观察框架中的代码,其中包含以下函数:
\inputCodeSetLanguage{c}
\begin{lstlisting}
accessData(mem_addr_t addr)
freeCache()
initCache()
main(int argc,char * argv[])
printUsage(char * argv[])
replayTrace(char * trace_fn)
\end{lstlisting}

\par 根据函数名和注释可以得知,initCache和freeCache用于初始化和释放cache,printUsage用于打印帮助信息,而replayTrace则直接被主函数调用并调用accessData来模拟Cache的替换过程。而需要修改的函数就是accessData函数。

\par 接下来观察Cache是如何在C语言中被组织并模拟的。首先,文件中定义了如下的结构,来描述cache line,并定义其指针以及指针的指针分别为cache set和cache。
\begin{lstlisting}
typedef struct cache_line {
    char valid;
    mem_addr_t tag;
    unsigned long long int lru;
}
typedef cache_line_t *cache_set_t;
typedef cache_set_t *cache_t;
\end{lstlisting}

\par 而通过initCache中的初始化过程可以得知,文件中描述了一个如图\ref{fig:cacheStructure}所示的cache:

\begin{figure}[htb]
    \centering
    \includegraphics[width=0.72\linewidth]{cacheStructure.png}
    \caption{Cache 结构}
    \label{fig:cacheStructure}
\end{figure}
\FloatBarrier

\par 了解模拟cache在内存中的结构后,每一次的内存访问过程如下:首先判断此内存地址是否能够在对应位置cache hit,如果cache hit则将hit计数加一,否则判定为cache miss,将miss计数加一,然后判断是否需要替换。如果需要替换则按照LRU规则进行替换,然后将eviction计数加一。由于cache line是以数组的形式存在于内存中的,因此在实现LRU的队列结构的时候需要注意对于结构体中lru字段的操作。

\subsection{详细设计}
\label{sub:xiang_xi_she_ji_}
\par 接下来根据LRU的替换规则设计accessData中具体的cache替换流程。首先,根据传入的地址计算组索引以及tag,并根据组索引获得cache组。接下来对于组中的所有cache line进行遍历,若某一个cache line的tag字段与事先计算的tag字段相等,则增加一次cache hit计数,然后更新此cache set中所有cache line的lru字段。若遍历完成都没有cache hit,则一定出现了cache miss,此时将cache miss计数加一。但此时依然需要分类讨论:若此cache set未被填满,则不需要进行替换,直接将新的记录插入cache set中并更新所有cache line的lru字段,否则还需要进行cache替换。替换方法为:首先遍历次cache set中所有的cache line,找出其中lru值最大的cache line(存在时间最长的),将其替换为新的地址后更新其他所有cache line的lru字段。具体的替换流程如图\ref{fig:cacheSub}所示。

\begin{figure}[htb]
    \centering
    \includegraphics[width=0.7\linewidth]{cacheSub.png}
    \caption{cache更新流程}
    \label{fig:cacheSub}
\end{figure}

\par 在上述三种不同的情况中,三种lru的更新方法是不同的:在cache hit时,遍历所有cache line,将lru字段小于命中的cache的lru字段值,并且valid字段为1的所有cache line的lru值加一,然后将命中的cache line的lru值置为0。
\par 在cache miss但未发生cache eviction时,首先遍历所有cache line,找出第一个valid字为0的cache line,将这一cache line的valid字段置为1,lru字段置为0并将其他所有valid字段为1的cache line的lru值加一。
\par 若出现cache miss eviction,则遍历所有cache line,找出lru值最大的cache line,将其lru值置为0并将这一cache set中其他所有cache line的lru值加一。则这三种替换过程如\ref{fig:cacheSub123}所示。

\begin{figure}[htb]
    \centering
    \includegraphics[width=0.9\linewidth]{cacheSub123.png}
    \caption{三种情况下的cache替换过程}
    \label{fig:cacheSub123}
\end{figure}

\par 除了上述对于核心部分的设计之外,还需要考虑其他的部分:题目要求实现与csim-ref完全相同的功能,而csim-ref的参数可以指定-h、-v、-s、-E、-b、-t。其中-h在所给出的代码框架中已经得以实现,而-s、-E、-b、-t全部是为核心部分的cache模拟提供支持的。-v参数则需要添加额外的实现。使用-v参数得到的一个样例输出如下:
\inputCodeNoNumberSetLanguage{bash}
\begin{lstlisting}
L 10,1 miss
M 20,1 miss hit
L 22,1 hit
S 18,1 hit
L 110,1 miss eviction
L 210,1 miss eviction
M 12,1 miss eviction hit
hits:4 misses:5 evictions:3
\end{lstlisting}

\par 从输出中可以看出,-v对于除了I型指令之外的内存操作均进行了输出,输出格式与valgrind的输出格式类似,但在每行后面添加了cache hit/miss/eviction的情况。

\section{实验过程}
\label{sec:shi_yan_guo_cheng_}

\subsection{实验环境}
\label{sub:shi_yan_huan_jing_}
\begin{table}[htb]
    \centering
    \caption{实验环境配置}
    \label{tab:label}
    \begin{tabular}{r l}
        \toprule
        操作系统        & Archlinux x64 2018-04-11 更新\\
        编译器          & gcc 7.3.1 \\
        Makefile管理器  & gnu make 4.2.1 \\
        内存调试工具    & valgrind 3.13.0 \\
        版本管理工具    & git 2.17.0 \\
        \bottomrule
    \end{tabular}
\end{table}

\subsection{详细步骤}
\label{sub:shi_yan_guo_cheng_}

\par 有了上述设计后,代码的实现就较为容易了。首先,补全freeCache程序中的代码。根据initCache中的代码可知源程序是在cache这一二维数组的两个维度进行了分配,而从逻辑上可以推断出,在整个程序的运行过程中不需要对于内存进行新的分配,因此,在freeCache时只需要对应的将initCache中分配的内存逐个释放即可。实现的代码如下:
\inputCodeSetLanguage{c}
\begin{lstlisting}
void freeCache() {
    int i;
    for (i = 0; i < S; ++i)
        free(cache[i]);
    free(cache);
}
\end{lstlisting}

\par 接下来实现核心部分代码,也就是accessData模拟cache更新这一部分。首先通过如下代码计算组编号以及tag值,并根据组编号获得cache组。
\begin{lstlisting}
mem_addr_t set_index = (addr >> b) & set_index_mask;
mem_addr_t tag = addr >> (s + b);
cache_set_t cache_set = cache[set_index];
\end{lstlisting}

\par 然后对于cache set进行第一次遍历。在每一次循环中,对于是否命中进行判断,判断的依据为valid是否为1且tag是否与当前内存地址的tag相等。若命中则增加命中计数并进行上述替换过程1,然后直接在返回。此外,还需要注意对于-v参数的支持:如果verbosity flag为1,则需要输出``hit''提示。
\begin{lstlisting}
for (i = 0; i < E; ++i) {
        /* hit */
        if (cache_set[i].valid && cache_set[i].tag == tag) {
            if (verbosity)
                printf("hit ");
            ++hit_count;

            /* update entry whose lru is less than the current lru (newer) */
            for (int j = 0; j < E; ++j)
                if (cache_set[j].valid && cache_set[j].lru < cache_set[i].lru)
                    ++cache_set[j].lru;
            cache_set[i].lru = 0;
            return;
        }
    }
}
\end{lstlisting}

\par 如果上述循环执行完成而没有返回,表明没有发生cache hit,因此使用以下代码将miss技术加一,并在verbosity flag为1时打印``miss''提示。
\begin{lstlisting}
if (verbosity)
    printf("miss ");
++miss_count;
\end{lstlisting}

\par 接下来需要分情况讨论是否会发生cache eviction。经过考虑后发现,可以将不发生eviction情况下寻找最大lru条目的过程与发生eviction情况下寻找第一个invalid cache line的过程合并在同一个循环中,以提高cache模拟器的性能,合并后的循环如下:
\begin{lstlisting}
int j, maxIndex = 0;
unsigned long long maxLru = 0;
for (j = 0; j < E && cache_set[j].valid; ++j) {
    if (cache_set[j].lru >= maxLru) {
        maxLru = cache_set[j].lru;
            maxIndex = j;
        }
    }
}
\end{lstlisting}

\par 在上述循环结束后,通过$j$的值来判断是否发生了cache eviction。由于上述循环在遇到invalid cache line时会跳出,因此当$j$小于E时表明未发生cache eviction。也就是说,如果发生cache eviction,则$j==E$一定成立。因此通过以下代码进行cache的更新。
\begin{lstlisting}
if (j != E) {
    for (int k = 0; k < E; ++k)
        if (cache_set[k].valid)
            ++cache_set[k].lru;
    cache_set[j].lru = 0;
    cache_set[j].valid = 1;
    cache_set[j].tag = tag;
} else {
    if (verbosity)
        printf("eviction ");
    ++eviction_count;
    for (int k = 0; k < E; ++k)
        ++cache_set[k].lru;
    cache_set[maxIndex].lru = 0;
    cache_set[maxIndex].tag = tag;
}
\end{lstlisting}

\par 在上述代码中,$j\neq E$部分为不发生eviction的情况,$j==E$的部分为发生eviction的情况,分别按照章节\ref{sub:xiang_xi_she_ji_}中的方法进行cache的更新。

\par 在accessData中核心部分的代码完成后,需要完成replayTrace中的代码调用accessData完成对于内存访问过程中cache变化的模拟。使用fscanf读入每一行,然后根据访问类型进行对于accessData的调用:忽略I型访问,对于L型和S型访问调用accessData一次,而对于M型访问调用accessData两次,此外还需要注意verbosity flag对于输出的影响。代码如下:
\begin{lstlisting}
while (fscanf(trace_fp, " %c %llx,%d", buf, &addr, &len) > 0) {
    if (verbosity && buf[0] != 'I')
        printf("%c %llx,%d ", buf[0], addr, len);
    switch (buf[0]) {
        case 'I':
            break;
        case 'L':
        case 'S':
            accessData(addr);
            break;
        case 'M':
            accessData(addr);
            accessData(addr);
            break;
        default:
            break;
    }
    if (verbosity && buf[0] != 'I')
        putchar('\n');
}
\end{lstlisting}

\par 至此,所有需要填写的代码已补充完成。

\subsection{测试与分析}
\label{sub:jie_guo_fen_xi_}

\par 对于完成的代码进行测试:首先使用make对代码进行编译,然后直接运行./test-csim命令。实验的测试程序给出的测试样例如表\ref{tab:example}所示,其输出结果如图\ref{fig:result1}所示。

\begin{center}
    \captionof{table}{测试样例}
    \label{tab:example}
    \begin{longtable}{r c c c c c c}
        \toprule
        \multicolumn{1}{c}{\textbf{测试文件}} &
        \multicolumn{1}{c}{\textbf{组索引位数 s}} &
        \multicolumn{1}{c}{\textbf{关联度 E}} &
        \multicolumn{1}{c}{\textbf{块偏移位数 b}} &
        \multicolumn{1}{c}{\textbf{hit}} &
        \multicolumn{1}{c}{\textbf{miss}} &
        \multicolumn{1}{c}{\textbf{eviction}}            \\
        \cmidrule(lr){1-1} \cmidrule(lr){2-4} \cmidrule(lr){5-7}
        yi2.trace   & 1 & 1 & 1 & 9      & 8     & 6     \\
        yi.trace    & 4 & 2 & 4 & 4      & 5     & 2     \\
        dave.trace  & 2 & 1 & 4 & 2      & 3     & 1     \\
        trans.trace & 2 & 1 & 3 & 167    & 71    & 67    \\
        trans.trace & 2 & 2 & 3 & 201    & 37    & 29    \\
        trans.trace & 2 & 4 & 3 & 212    & 26    & 10    \\
        trans.trace & 5 & 1 & 5 & 213    & 7     & 0     \\
        long.trace  & 5 & 1 & 5 & 265189 & 21775 & 21743 \\
        \bottomrule
    \end{longtable}
\end{center}

\begin{figure}[htb]
    \centering
    \includegraphics[width=0.8\linewidth]{result1.png}
    \caption{test-csim输出结果}
    \label{fig:result1}
\end{figure}

\par 从图中可以看出,所有的测试通过。接下来对于-v选项进行测试。首先对于yi.trace的进行测试,然后对比处理其他trace文件的输出与csim-ref处理相同文件的输出。运行的结果如\ref{fig:result2}所示。

\begin{figure}[htb]
    \centering
    \includegraphics[width=0.95\linewidth]{result2.png}
    \caption{对于-v选项的测试}
    \label{fig:result2}
\end{figure}

\par 可以看出,csim的输出与csim-ref的输出完全相同。至此,所有的测试完成,csim能够完全复现csim-ref的功能。




% chapter 2
\chapter{基于pthread的形态学图像处理}
\section{实验目的与要求}
\begin{itemize}
    \item 掌握使用pthread的基本的并行编程设计方法以及调优方法;
    \item 掌握并行编程中基本的数据分块以及任务分解的方法。
    \item 使用pthread实现并行的形态学图像处理。
    \item 简要分析以及总结处理的结果。
\end{itemize}

\section{算法描述}
\par 使用多个线程对于一个图像进行蚀刻以及膨胀的算法如下,算法为一个线程的流程,而有多个这样的线程同时进行。
\begin{simpleAlgorithm}{pthread并行处理算法(一个线程)}
    \Procedure{PthreadParallel}{$blocks$}
    \While{true}
        \State lock\((blocks)\)
        \State get first block \(blk\) from \(blocks\)
        \If{\(blocks\).empty()}
            \State unlock\((blocks)\)
            \State \Return
        \EndIf
        \State unlock\((blocks)\)
        \State \Call{ErodeAndDilate}{$blk, kernel_e, kernel_d$}
    \EndWhile
    \EndProcedure
\end{simpleAlgorithm}
\par 算法中,\(blocks\)参数为一个工作队列,队列中的工作为原预处理过后的图片的子图片。在每个线程的每个循环中,首先锁住队列,从队列中获取一个子图片\(blk\)、解锁队列然后使用上一章中的ErodeAndDilate过程进行处理。如果\(blocks\)中没有子图片,说明处理完成,则此线程退出。
\par 主线程的流程如图\ref{fig:pthreadMain}所示。在进行预处理过后启动多个线程,然后等待所有线程竞争子图像、处理然后结束即可,最后保存处理的结果即可。
\begin{figure}[htpb]
    \centering
    \includegraphics[width=0.95\linewidth]{pthreadMain.png}
    \caption{主线程流程}
    \label{fig:pthreadMain}
\end{figure}

\par 由于是使用pthread的并行算法,每一个线程处理一个部分,因此首先需要将数据分块(即分为算法中的\(blocks\))。分块方式如图\ref{fig:partition}所示。每块大小一样,在边缘部分如果块大小不符则按照原图的边缘进行裁减。因此,在进行处理时需要对于边缘部分进行考虑。
\begin{figure}[htpb]
    \centering
    \includegraphics[width=0.76\linewidth]{partition.png}
    \caption{分块方法}
    \label{fig:partition}
\end{figure}

\section{实验方案}
\par 所有的开发与运行环境见附录\ref{cha:env},表\ref{tab:env},此后实验的开发与运行环境均相同,不再赘述。根据算法描述、分块方法以及主线程的流程编写程序并运行,然后观察结果并与串行的程序比较。经过多轮的比较以及参数调试后得出一个较好的效果。

\section{实验结果与分析}
\par 功能上,程序处理后的图片与串行处理后的图片一致,此处不再给出。4线程,分块大小为128的情况下程序的运行时间如图\ref{fig:pthreadOutput}所示。在4个线程的情况下,运行三次的平均运行时间为11.7s,相比于串行算法,程序的加速比为\(44.2\div 11.7 = 3.77\),已经十分接近理想加速比4。
\begin{figure}[htpb]
    \centering
    \includegraphics[width=0.9\linewidth]{pthreadOutput.png}
    \caption{pthread程序运行时间}
    \label{fig:pthreadOutput}
\end{figure}

\par 经过8组、每组3次的测试,加速比随线程变化的曲线如图\ref{fig:pthreadTrend}所示。可以看出,在线程数为1\textasciitilde 4时加速比随着线程数几乎呈线性变化,而在线程数为1时加速比为1.006,overhead所占用的时间几乎可以不计。在线程数达到4时由于物理内核已经被占满,因此后面加速比不再增加,随着线程数量的进一步增大,由于线程调度的开销,因此程序的加速比不再增加,反而有所下降。
\begin{figure}[htpb]
    \centering
    \includegraphics[width=0.8\linewidth]{pthreadTrend.png}
    \caption{加速比随线程数变化}
    \label{fig:pthreadTrend}
\end{figure}

\par 对于分块大小而言,加速比随着分块大小的变化如图\ref{fig:pthreadTrend2}所示,在分块大小较小时,加速比随着分块大小的变化并不大,只在分块大小过小时由于线程调度导致一点性能开销。当分块大小大于原图的一半时总时间则取决于分到最大分块线程所用的时间,因此在这个区间内性能随分块大小呈下降趋势。
\begin{figure}[htpb]
    \centering
    \includegraphics[width=0.8\linewidth]{pthreadTrend2.png}
    \caption{加速比随分块大小变化}
    \label{fig:pthreadTrend2}
\end{figure}




\chapter*{心得体会}
\addcontentsline{toc}{chapter}{心得体会}
\label{cha:xin_de_ti_hui_}

\par 通过这次实验,我对于cache的结构,运作方式以及如何更高效的写出cache friendly的代码有了更为深入的了解。通过对于cache的模拟,我对于cache的组成方式,与内存地址的对应关系以及cache组相连的工作方式有了更为深入的认识。
\par 在对于错误代码的调试过程中,我尝试使用了各种方法来结构性的显示cache的内容以及替换的过程,这也给了我有关cache工作方式更为直观的印象。
\par 第二个实验则更令我印象深刻,它从简单的矩阵变换入手,展示了对于程序员而言透明的cache是如何影响代码性能的,以及程序员应该如何优化代码使之能够在cache层面有更好的工作效率。通过逐步深入的实验,对于cache的理解也逐步加深,并在实验查阅资料的过程中了解到了cache优化在包括科学计算等多个领域的应用,这不仅开拓了我的视野,也为今后的学习工作奠定了基础。

{\let\clearpage\relax \chapter*{源代码}}
\addcontentsline{toc}{chapter}{源代码}
\label{cha:yuan_dai_ma_}

\inputCodeSetLanguage{c}
\noindent \textbf{csim.c}
\lstinputlisting{../src/csim.c}

\noindent \textbf{trans.c}
\lstinputlisting{../src/trans.c}

\vfill
{\tiny written by HuSixu \hfill powered by \XeLaTeX .}
\end{document}

%%%%%%%%%%%%%%%%%%%%%%%%%%%%%%%%%%%%
%%%%%%%%%%%%%%%%%%%%%%%%%%%%%%%%%%%%
%%%%%%%%%%%%%%%%%%%%%%%%%%%%%%%%%%%%
% in this report i learned:
% how to use a tcolorbox(see conf/report_settings.tex)
% how to use a siderule(see conf/report_settings.tex)
% how to use long argument in xparse (+m, +o, ..., see cont/report_settings.tex)
% how to use latex and avoid warning(see conf/ctex_wrapper.tex)
% how to use footnote in a box and let footnote show at the bottom of the page (use footnotemark and footnotetext, do not use footmisc package)

